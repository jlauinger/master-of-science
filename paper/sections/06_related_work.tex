\section{Related Work}
\label{sec:rw}

% Analyses in Go / Go and Security
%\textbf{Analyses in Go}
Previous research on Go mostly concentrated on issues related to its concurrency model including the channel implementation~\cite{tu2019,dilley2019,giunti2020,gabet2020,lange2017,bodden2016information}.
The work by Wang et al.~\cite{wang2020} suggests an improvement of the existing escape analysis in Go which we also discussed in our paper. 

% Unsafe in other languages
%\textbf{Unsafe usages in memory safe languages}

%While previous research on Go did not answer the question on the influence of \unsafe{} code, related studies exist for other languages. 
Moreover, the usage of \unsafe{} in other languages has already been studied to varying degrees. 
For Java, Mastrangelo et al.~\cite{mastrangelo2015} identified that 25\% of the analyzed artefacts depend on the Java \unsafe{} library.
The different JVM crash patterns caused by those usages are analyzed by Huang et al.~\cite{huang2019}.
Recently, two studies analyzed \unsafe{} usages in Rust projects and identified that \unsafe{} is widely used to improve performance or to reuse existing code~\cite{qin2020,evans2020}.
Furthermore, work was presented on how to ensure memory safety while using \unsafe{} in Rust~\cite{hussain2018Fidelius}.
Lehmann et al.~\cite{lehmann-everything-2020} studied to which extent \unsafe{} programs compiled to WebAssembly can lead to vulnerabilities within the virtual machine environment. %and it is possible to compile Go code to WebAssembly. 
%
% Shorter version 
%Recently, related studies were published that focus on unsafe code and its usage in Rust~\cite{qin2020,evans2020} and Java~\cite{mastrangelo2015,huang2019}.
%Lehmann et al.~\cite{lehmann-everything-2020} studied to which extend unsafe programs can lead to vulnerabilities in WebAssembly. %and it is possible to compile Go code to WebAssembly. 
%
% Memory vulnerabilities and C/C++
For C/C++, non memory-safe languages, research exists on how to support at least partial memory safety~\cite{burow2018CUP, nagarkatte2009SoftBound} and work on identifying vulnerabilities by program analyses~\cite{song2019sok}.
A comprehensive study on memory-management-related vulnerabilities, like the ones we discussed earlier, and their mitigations is presented in earlier work~\cite{szekeres2013sok}.

%Dependencies
Concerning project dependencies, it is difficult to count the dependencies that matter the most, e.g., by excluding test dependencies~\cite{pashchenko2018}.
A common problem is that dependencies are often updated slowly, keeping old bugs alive, although measures such as automated pull requests exist to mitigate this problem~\cite{derr2017keep, mirhosseini2017, lauinger2017}.


