\section{Discussion}
\label{sec:discussion}

Looking at the study results, we see that \unsafe{} is used consistently and wide-spread in the most popular open-source Go projects.
One might argue that the found patterns only are minor annoyances, not that severe or would require a manual case-to-case inspection. 
The general exploitability of various of these patterns was discussed in Section~\ref{sec:appr} with a link to proof-of-concept exploits that we developed, clearly showing that it is indeed possible to use the memory corruptions to ones advantage.
Furthermore, we already submitted \numberPRs{} pull requests to fix more than \numberBugsFixed{} bugs that we found using \toolSA{}.
So far, \numberPRsMerged{} have already been reviewed and accepted by the corresponding project maintainers.

%Since we showed actual exploit vectors and bugs with code snippets that are commonly used in popular real-world applications, we think that research on \unsafe{} is clearly not only an academic issue, but has direct and important consequences on the actual software market.
%Audit efforts should be made up to an import depth of around 4.


Potential internal threats to validity for our study include bias towards bigger projects because those might be over-represented in the manually labeled data set. 
External threats include a bias towards more active projects with many developers because we selected a subset of the most-starred open-source projects on GitHub. 
Also we only considered projects that use the Go module system, about a third of the top 500 projects are not modularized yet.
Further, we could have missed projects from a special domain not having that many stars which might have other usage scenarios for \unsafe{} Go.
Nevertheless, one can argue that the biggest projects also have professional developers, higher standards and code gets more reviewed, thus, code quality should be higher.

