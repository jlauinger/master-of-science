\section{Discussion}
\label{sec:discussion}

Looking at the study results, we see that \unsafe{} is used consistently and wide-spread in the most popular open-source Go projects.
Since we showed actual exploit vectors and bugs with code snippets that are commonly used in popular real-world applications, we think that research on \unsafe{} is clearly not only an academic issue, but has direct and important consequences on the actual software market.
Audit efforts should be made up to an import depth of around 4.

We submitted \numberPRs{} pull requests to fix more than \numberBugsFixed{} bugs that we found using \toolSA{}.
\numberPRsMerged{} had already been accepted by the submission of this paper.

Potential internal threats to validity for our study include bias towards bigger projects because those might be over-represented in the manually labeled data set. External threats include bias towards more active projects with many developers because we selected a subset of the most-starred open-source projects. Different project domains might use a different distribution and intention with \unsafe{} Go.


