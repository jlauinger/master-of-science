The Go programming language, although aiming to provide memory and thread safety through a strict type system enforced at compile-time, also offers ways to circumvent this safety net through using the unsafe package.
It allows arbitrary type casts and pointer arithmetic.
There are legitimate use cases, e.g. low-level memory operations needed in the Go runtime, or increased efficiency due
to avoiding unnecessary allocations.
However, such uses must be done with caution to avoid introducing vulnerabilities common to C programs, like buffer
overflows and use-after-free.
In this work, we contribute a large-scale quantitative study on the usage of unsafe in the \projsAnalyzed{} most popular Go projects on GitHub, including an analysis of \checkNum{1,400} code samples on how unsafe is used and for
what purpose.
We find that \checkNum{5.5\%} of packages used transitively by the projects contain unsafe usages. \checkNum{38 \%}
of projects contain at least one usage that is not part of the standard library, and \checkNum{91 \%} of projects
contain at least one \unsafe{} in the project or one of its transitive dependencies.
Based on the usage patterns found, we present dangers and possible exploit vectors.
Finally, we present two novel tools for Go developers: \toolUsage{} to count unsafe usages in a package and its
dependencies, and \toolSA{} to identify dangerous usage patterns that were previously undetected by existing tools.
\todo{refactor to new storyline}