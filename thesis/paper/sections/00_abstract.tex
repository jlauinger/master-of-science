The Go programming language aims to provide memory and thread safety through compile-time measures such as a strict type system that prevents invalid memory accesses and data races. However, it also offers a way of circumventing this safety net through using the \unsafe{} package.
This package allows arbitrary type casts and pointer arithmetic.
There are legitimate use cases, but such uses must be done with extreme caution to avoid introducing vulnerabilities common to C programs, like buffer overflows and use-after-free.
Furthermore, usages of \unsafe{} may be present not only in a project's first-party source code, but can be introduced through third-party dependencies as well.
In this work, we present \toolUsage{}, a novel tool for Go developers to count \unsafe{} usages in a package and its dependencies.
This tool enables developers to focus auditing efforts on the packages that actually contain \unsafe{} usages, therefore increasing cost and time efficiency.
Using \toolUsage{}, we contribute a large-scale quantitative study on the usage of \unsafe{} in the \projsAnalyzed{} most popular open-source Go projects on GitHub, including a manual analysis of \numberCodeSnippets{} code samples on how \unsafe{} is used and for what purpose.
We find that \percentagePackagesWithUnsafe{} of packages used transitively by the projects contain \unsafe{} usages. 
\percentageProjectsWithUnsafe{} of projects contain at least one usage that is not part of the standard library, and \percentageProjectsAndDependenciesUnsafe{} of projects contain at least one \unsafe{} in the project or one of its transitive dependencies.
The analyzed projects use \unsafe{} primarily for efficiency reasons and to create functionality of generics, which are not available in current versions of Go.
Based on the usage patterns found, we present dangers and possible exploit vectors. Finally, we present \toolSA{}, a novel static analysis tool to identify two dangerous and common usage patterns that were previously undetected with existing tools.
