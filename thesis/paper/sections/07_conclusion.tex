\section{Conclusion}
\label{sec:concl}

In this paper, we presented two novel tools to help Go developers write safer code with respect to \unsafe{} Go.
\toolUsage{} identifies and counts \unsafe{} usages not only within a package, but also in its transitive dependencies, therefore providing effective tools to focus audit efforts on the code locations that are the most dangerous, and raising awareness into how \unsafe{} code is included into a project.
\toolSA{} is a new static code analysis tool that helps developers identify two dangerous code patterns that were previously uncaught with existing tools.
We presented novel evidence of how to exploit common unsafe code patterns, and submitted fixes to \numberBugsFixed{} bugs.
We conducted a large-scale study of \packagesAnalyzed{} in \projsAnalyzed{} top-starred open-source Go projects and found that \percentagePackagesWithUnsafe{} of packages used transitively by the projects contain \unsafe{} usages. 
\percentageProjectsWithUnsafe{} of projects contain at least one usage that is not part of the standard library, and \percentageProjectsAndDependenciesUnsafe{} of projects contain at least one \unsafe{} in the project or one of its transitive dependencies.
Finally, we presented a new data set of manually labeled code snippets, providing insight into how and for what purpose \unsafe{} is used by developers.

Based on our contributions, future work might look into supervised learning algorithms using our labeled data set to train classifiers that can identify the purpose and domain of \unsafe{} usages by looking at new code examples.
As Smith et al. show, avoiding security vulnerabilities and fixing bugs can be made easier for developers by integrating code analysis tools into code editors and IDEs \cite{smith2020}.
Thus, future work could build plugins for common IDEs that integrate our tools \toolUsage{} and \toolSA{}.
