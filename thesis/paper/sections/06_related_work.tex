\section{Related Work}
\label{sec:rw}

% Analyses in Go / Go and Security
There are many previous works on the security of other Go features like channels and concurrency~\cite{tu2019,dilley2019,giunti2020,gabet2020,lange2017}, and a static code optimization approach~\cite{wang2020} that  actively breaks the escape analysis as also described in our paper.
Security issues can be analyzed in project code like we did, or in a project's commit history~\cite{zhou2017}.
% Unsafe in other languages
Recently, related studies were published that focus on unsafe code and its usage in Rust~\cite{qin2020,evans2020} and Java~\cite{mastrangelo2015,huang2019}.
Lehmann et al~\cite{lehmann-everything-2020} studied to which extend unsafe programs can lead to vulnerabilities in WebAssembly. %and it is possible to compile Go code to WebAssembly. 
In contrast to analyzing the avoidance of memory safety, there are many previous works on how to support at least partial memory safety in C/C++ code ~\cite{burow2018CUP} \aw{Add others}.

% Memory vulnerabilities
A comprehensive study on vulnerabilities in memory, like the ones we discussed earlier, and their mitigation's is presented in earlier work~\cite{szekeres2013sok}.


%Dependencies
Concerning project dependencies, it is difficult to count the dependencies that matter the most, e.g. by excluding test dependencies~\cite{pashchenko2018}.
A common problem is that dependencies are often updated slowly, keeping old bugs alive, although measures such as automated pull requests exist to mitigate this problem~\cite{kula2017, mirhosseini2017}.
