\section{Methodology and Implementation}
\label{sec:impl}

The goal of our study is to understand how \unsafe{}~is used in open-source projects and the implications on the security of Go applications. 
To achieve this aim, we answer the following research questions:

\begin{enumerate}[label={RQ\arabic*},leftmargin=*] 
    \item How many projects use \unsafe{} in Go code in their application? \label{rq:prevalApp}
    \item How many projects introduce \unsafe{} through their dependencies, and from which registries? \label{rq:prevalDeps}
    \item How deep in the import stack are the most imported \unsafe{} code packages? \label{rq:depsDepth}
    \item Which \unsafe{} keywords are used most? \label{rq:distTypes}
    \item What \unsafe{} operations are used in practice, and what is their goal? \label{rq:purpose}
    \item Are there problems arising from the use of \unsafe{} that can lead to exploitable vulnerabilities? \label{rq:probsUnsafe}
\end{enumerate}

Later ref to the RQ with the label \ref{rq:probsUnsafe}.

\subsection{Data Set Creation}

To answer our research question, we create a data set of open-source Go code available on GitHub.
As our research is focused on projects, we decide to crawl the \initalProjs{} most-stared Go projects available on GitHub. 
To further understand the influence of the dependencies, we selected the applications supporting \textit{go modules}.
With the introduction of Go \checkNum{1.13}, \textit{go modules} are the way to include dependencies within a Go application with the help of the Go toolchain. 
Unfortunately, \withoutModules{} of the projects did not yet support go modules and we had to exclude them.
We further, had to remove \notCompiled{} projects as we couldn't compile those.
As a result, we end up with \projsAnalyzed{} top-rated Go projects collected from GitHub. 

\aw{Perhaps, we want to include some star stats or similar stats to show that we have relevant Go projects.}


\begin{figure*}[!t]
    \begin{tikzpicture}
        % Define tikz styles
        \tikzset{metaGroup/.style={
            draw = gray, thin, rounded corners,
        }}
        \tikzset{inner/.style={
            font=\tiny,
        }}
    
        % create nodes
        \node (mark) {\includegraphics[width=11mm]{gfx/figures/mark.png}};
        \node (checkout) [right=of mark, yshift=10mm, inner] {\initalProjs};
        \node (goModules) [right= of mark, inner] {- \withoutModules{}};
        \node (compile) [right=of mark, yshift=-10mm, inner] {- \notCompiled{}};
        \node (projects) [fit=(checkout)(goModules)(compile)] {};
        \node (dataSet) [fit=(projects)(mark), metaGroup, label=Data set creation] {};
        
        \node (analysis) [right=of dataSet] {analysis};
        
        % create connections between components
        \draw [->] (mark) -- (projects);
        \draw [->, thin, font=\tiny] (checkout) -- node[anchor=east] {no go modules} (goModules);
        \draw [->, thin, font=\tiny] (goModules) -- node[anchor=east] {not build} (compile);
        
        \draw [->] (dataSet) -- node[anchor=south] {\projsAnalyzed} (analysis);
    \end{tikzpicture}
    \caption{Overview of our Methodology and Empirical Results.}
    \label{fig:overview}
\end{figure*}

