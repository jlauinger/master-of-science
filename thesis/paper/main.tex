\documentclass[conference,compsoc]{IEEEtran}

% *** CITATION PACKAGES ***
%
\ifCLASSOPTIONcompsoc
  % IEEE Computer Society needs nocompress option
  % requires cite.sty v4.0 or later (November 2003)
  \usepackage[nocompress]{cite}
\else
  % normal IEEE
  \usepackage{cite}
\fi

% correct bad hyphenation here
\hyphenation{op-tical net-works semi-conduc-tor}


\usepackage{url}
\usepackage{adjustbox}
\usepackage{graphicx}
\usepackage{caption}
\usepackage{subcaption}
\usepackage{xcolor}
\usepackage{colortbl}
\usepackage{multirow}
\usepackage[hidelinks]{hyperref}
\usepackage[nolist]{acronym}
\usepackage{amssymb}
\usepackage{blindtext}
\usepackage{enumitem}
\usepackage{makecell}
\usepackage{listing}
\usepackage{tabularx}
\usepackage{enumitem} % Add labels for RQs defined in enumerated list
\usepackage{amsmath}
\usepackage{balance}

%% Tikz
\usepackage{tikz}
\usetikzlibrary{positioning, fit} % positioning enables relative positions in tikz, fit enable to create groups of nodes
\usepackage{tikzscale}
\usepackage{pgfplots}
\DeclareUnicodeCharacter{2212}{−}
\usepgfplotslibrary{groupplots,dateplot}
\usetikzlibrary{patterns,shapes.arrows}
\pgfplotsset{compat=newest}

%% Listings
\RequirePackage{listings}
\lstdefinelanguage{Golang}%
    {morekeywords=[1]{package,import,func,type,struct,return,defer,panic,%
    recover,select,var,const,iota,},%
    morekeywords=[2]{string,uint,uint8,uint16,uint32,uint64,int,int8,int16,%
    int32,int64,bool,float32,float64,complex64,complex128,byte,rune,uintptr,%
    error,interface},%
    morekeywords=[3]{map,slice,make,new,nil,len,cap,copy,close,true,false,%
    delete,append,real,imag,complex,chan,},%
    morekeywords=[4]{for,break,continue,range,go,goto,switch,case,fallthrough,if,%
    else,default,},%
    morekeywords=[5]{Println,Printf,Error,Print,},%
    sensitive=true,%
    morecomment=[l]{//},%
    morecomment=[s]{/*}{*/},%
    morestring=[b]',%
    morestring=[b]",%
    morestring=[s]{`}{`},%
}
\lstset{
    frame=none,
    rulecolor=\color{red},
    basicstyle=\footnotesize,
    keywordstyle=\color{blue},
    numbers=left,
    numbersep=5pt,
    showstringspaces=false,
    stringstyle=\color{darkgreen},
    tabsize=4,
    language=Golang,
    xleftmargin=17pt,
    framexleftmargin=27pt,
    framexrightmargin=5pt,
    framexbottommargin=4pt,
    breaklines=true,
    postbreak=\mbox{\textcolor{red}{$\hookrightarrow$}\space},
}

\definecolor{verylightgray}{RGB}{240,240,240}
\definecolor{darkgreen}{RGB}{20, 150, 20}
\definecolor{fuchsia}{rgb}{1.0, 0.0, 1.0}
\newcommand{\toolUsage}{\textit{go-geiger}}
\newcommand{\toolSA}{\textit{go-safer}}
\newcommand{\unsafe}{\textit{unsafe}}

\newcommand{\projsAnalyzed}{\checkNum{343}}
\newcommand{\initalProjs}{\checkNum{500}}
\newcommand{\withoutModules}{\checkNum{150}}
\newcommand{\notCompiled}{\checkNum{7}}


\begin{document}
%
% paper title
% Titles are generally capitalized except for words such as a, an, and, as,
% at, but, by, for, in, nor, of, on, or, the, to and up, which are usually
% not capitalized unless they are the first or last word of the title.
% Linebreaks \\ can be used within to get better formatting as desired.
% Do not put math or special symbols in the title.
\title{A Study of Prevalence, Purpose, and Dangers\\ of Unsafe Go Code in the Wild}


% conference papers do not typically use \thanks and this command
% is locked out in conference mode. If really needed, such as for
% the acknowledgment of grants, issue a \IEEEoverridecommandlockouts
% after \documentclass

% for over three affiliations, or if they all won't fit within the width
% of the page (and note that there is less available width in this regard for
% compsoc conferences compared to traditional conferences), use this
% alternative format:
% 
\author{
\IEEEauthorblockN{
A A\IEEEauthorrefmark{2},
B B\IEEEauthorrefmark{1},
C C\IEEEauthorrefmark{1}, 
Mira Mezini\IEEEauthorrefmark{1}
}
\IEEEauthorblockA{
Technische Universität Darmstadt, D-64289 Darmstadt, Germany\\
}
\IEEEauthorblockA{\IEEEauthorrefmark{1}
E-mail: \{A, B , C, mezini\}@cs.tu-darmstadt.de \\
}
\IEEEauthorblockA{\IEEEauthorrefmark{2}
E-mail: A@cs.tu-darmstadt.de \\
}
}

% use for special paper notices
%\IEEEspecialpapernotice{(Invited Paper)}

% make the title area
\maketitle

% As a general rule, do not put math, special symbols or citations
% in the abstract
\begin{abstract}
The Go programming language aims to provide memory and thread safety through compile-time measures such as a strict type system that prevents invalid memory accesses and data races. However, it also offers a way of circumventing this safety net through using the \unsafe{} package.
This package allows arbitrary type casts and pointer arithmetic.
There are legitimate use cases, but such uses must be done with extreme caution to avoid introducing vulnerabilities common to C programs, like buffer overflows and use-after-free.
Furthermore, usages of \unsafe{} may be present not only in a project's first-party source code, but can be introduced through third-party dependencies as well.
In this work, we present \toolUsage{}, a novel tool for Go developers to count \unsafe{} usages in a package and its dependencies.
This tool enables developers to focus auditing efforts on the packages that actually contain \unsafe{} usages, therefore increasing cost and time efficiency.
Using \toolUsage{}, we contribute a large-scale quantitative study on the usage of \unsafe{} in the \projsAnalyzed{} most popular open-source Go projects on GitHub, including a manual analysis of \numberCodeSnippets{} code samples on how \unsafe{} is used and for what purpose.
We find that \percentagePackagesWithUnsafe{} of packages used transitively by the projects contain \unsafe{} usages. 
\percentageProjectsWithUnsafe{} of projects contain at least one usage that is not part of the standard library, and \percentageProjectsAndDependenciesUnsafe{} of projects contain at least one \unsafe{} in the project or one of its transitive dependencies.
The analyzed projects use \unsafe{} primarily for efficiency reasons and to create functionality of generics, which are not available in current versions of Go.
Based on the usage patterns found, we present dangers and possible exploit vectors. Finally, we present \toolSA{}, a novel static analysis tool to identify two dangerous and common usage patterns that were previously undetected with existing tools.

\end{abstract}

% no keywords

% For peer review papers, you can put extra information on the cover
% page as needed:
% \ifCLASSOPTIONpeerreview
% \begin{center} \bfseries EDICS Category: 3-BBND \end{center}
% \fi
%
% For peerreview papers, this IEEEtran command inserts a page break and
% creates the second title. It will be ignored for other modes.
\IEEEpeerreviewmaketitle


\section{Introduction}
\label{sec:intro}


The adoption of memory-safe languages for all kinds of different applications has been increasing significantly in the last decade. 
While environments and languages such as Java, Rust, Nim or Google's Go try to eliminate many bug classes through their language design and/or runtime, they also provide, to varying degrees, escape hatches to perform potentially unsafe operations if explicitly requested.
These might be used for optimization purposes, for convenience of the programmer, to directly access hardware or to circumvent limitations of the programming language.
Often enough unsafe code blocks are indirectly introduced through 3rd party libraries \cite{evans2020} and thus not directly obvious to the application developer.
Not knowing about the dangers introduced through external dependencies can have severe consequences.
Thus, developers and administrators need ways to quickly evaluate the potential risks introduced by other code.

For this paper we took a deeper look into Google's Go programming language and the usage of unsafe code blocks within the most popular software projects using it. 
During this work, we developed two specific tools for developers and security analysts. The first one, called \toolUsage{}, analyzes a project including its dependencies for general usage of unsafe and in which context.
The second one, called \toolSA{}, helps during development by providing meaningful hints to potentially dangerous usages of unsafe and how to avoid them. 
We analyzed the top \projsAnalyzed{} Go projects on GitHub to see how often unsafe is used in the wild. \aw{I would already mention some of our results here, e.g., We found that AB~\% of the applications make use of unsafe either in the application code or in a dependency.} 
From there we identified reasons for introducing unsafe in the source code in the first place and also whether it was really necessary.
Additionally, we provide insights into the dangers and possible exploit vectors to some of the patterns we found in the wild. 
Thus, we show the severe nature of these bugs.
Through the course of this work we made over \checkNum{14} pull requests to analyzed projects and libraries, fixing over \checkNum{60} individual potentially dangerous unsafe usages.

In this paper, we present the following contributions:

\begin{itemize}
\item We present to the best of our knowledge the first large-scale quantitative evaluation on the usage of unsafe in the \projsAnalyzed{} most popular Go projects on GitHub
\item We present a novel data set with \checkNum{1,400} labeled occurrences of unsafe code in various contexts of Go projects, providing insight into what is being done and for what purpose
\item We present novel evidence on how to exploit Go unsafe usages in the wild
\item We present a novel tool, \toolUsage{}, for detecting and scoring the presence of unsafe in Go projects and their dependencies
\item We present a novel static code analysis tool, \toolSA{}, to aid developers in identifying potentially problematic unsafe usage in their code base that were previously uncaught with existing tools
\end{itemize}

The paper is organized as follows:
Section~\ref{sec:background} gives a short introduction into unsafe Go.
Section~\ref{sec:rw} discusses related work. 
In Section~\ref{sec:impl}, we present our methodology and data selection for the large-scale study, as well as our approach to collect and label data.
Section~\ref{sec:eval} presents the results of our work.
Section~\ref{sec:exploits} presents potential exploit vectors to some unsafe patterns found in the wild.
In Section~\ref{sec:tools}, we present our novel tools \toolUsage{} and \toolSA{}.
Section~\ref{sec:discussion} discusses our approach, including potential threads to validity.
Section~\ref{sec:concl} concludes the paper and outlines areas of future work.

\section{Go's \unsafe{}~package}
\label{sec:background}



The Go programming language, like other memory-safe languages, provides the \unsafe{}~package\footnote{\url{https://golang.org/pkg/unsafe}}, which offers 
(a) the functions \textit{Sizeof}, \textit{Alignof}, and \textit{Offsetof} that are evaluated at compile time and provide access into memory alignment details of Go data types that are otherwise inaccessible, %unnecessary to know, and thus, inaccessible.
and (b) a pointer type, \textit{unsafe.Pointer} that allows developers to circumvent restrictions of regular pointer types.

One can cast any pointer to/from \textit{unsafe.Pointer}, thus enabling casts between completely arbitrary types, as illustrated in  
%
%In the remainders of this section, we discuss two example use cases of the \unsafe{} package in practice. 
Listing~\ref{lst:unsafe-ex-in-place-cast}.
%shows the usage of the \unsafe{} package to cast between arbitrary types.
In this example, \textit{in.Items} is assigned to a new type (\textit{out.Items}) in line 3 without reallocation for efficiency reasons.\footnote{This code was taken from the Kubernetes \textit{k8s.io/apiserver} module with minor adjustments.} 
Furthermore, casts between \textit{unsafe.Pointer} and \textit{uintptr} are also enabled, mainly for pointer arithmetic.
A \textit{uintptr} is an integer type with a length sufficient to store memory addresses. 
However, it is not a pointer type, hence, not treated as a reference.
%The first \textit{unsafe.Pointer} rule allows casts between completely arbitrary types, and the latter one allows the use of pointer arithmetic.
%The usage of the \unsafe{}~package removes the safety net provided by the Go type system and compiler, and brings developers down to the flexibility and danger of the pointers in C.
%
Listing~\ref{lst:unsafe-ex-escape-analysis} presents an example of casts involving \textit{uintptr}. 
%The code is taken from the \textit{modern-go/reflect2} module.
In Line 2, the \textit{unsafe.Pointer} is converted to \textit{uintptr}.
Thus, the memory address is stored within a non-reference type.
Hence, the back-conversion in Line 3 causes the \textit{unsafe.Pointer} to be hidden for the \textit{escape analysis (EA)} that Go's garbage collector uses 
%manages the memory allocations and tries to identify memory which can be freed up.
%For this task, it uses \textit{escape analysis (EA)} 
to determine whether a pointer is local to a function and can be stored in the corresponding stack frame, 
or whether it can \textit{escape} the function and needs to be stored on the heap \cite{wang2020}. 
Since \textit{uintptr} values are not pointer types, storing the address of a pointer in a variable of such a type and then converting it back causes the \textit{EA} to miss the chain of references to the underlying value in memory. 
Therefore, Go will assume a value does not escape when it actually does, and may place it on the stack.
Correctly used it can improve efficiency because deallocation is faster on the stack than on the heap~\cite{wang2020}.
However, used incorrectly it can cause security problems as shown later in Section~\ref{sec:appr}.

\begin{lstlisting}[language=Golang, label=lst:unsafe-ex-in-place-cast, caption=In-place cast using the \unsafe{} package
from the Kubernetes \textit{k8s.io/apiserver} module with minor changes.
,float, belowskip=-1.5em]
func autoConvert(in *PolicyList, out *audit.PolicyList) error {
	// [...]
	out.Items = *(*[]audit.Policy)(unsafe.Pointer(&in.Items))
	return nil
}
\end{lstlisting}

\begin{lstlisting}[language=Golang, label=lst:unsafe-ex-escape-analysis, caption=Hiding a value from escape analysis from the \textit{modern-go/reflect2} module.
, float, belowskip=-1.5em]
func NoEscape(p unsafe.Pointer) unsafe.Pointer {
	x := uintptr(p)
	return unsafe.Pointer(x ^ 0)
}
\end{lstlisting}


% \subsection{Go Dependency Management}

%Old way: packages, Go Path

%New way: modules, registries, \textit{go.mod} file.

%Package cache, versions, bad reproducibility, relatively high error rates for dependency resolution.


\section{Tools for a Safer Go World}
\label{sec:appr}

%This section describes problems related to \unsafe{} that we identified, as well as information on how to exploit these issues in the wild.
%In the following, we present our two novel tools, \toolUsage{} and \toolSA{}, that aid in locating, evaluating and fixing potentially dangerous \unsafe{} usages.
In the following, we present our two novel tools: 
\new{\toolUsage{} aids to provide an overview of all different types of \unsafe{} usages %and their context, e.g., an assignment, 
of a project and the dependencies. 
Thus, it supports auditing a project and perhaps selecting dependencies more carefully. 
In contrast, \toolSA{} is a linter, which can be easily integrated into a developers workflow and can identify known potentially dangerous \unsafe{} patterns.}
%\toolUsage{} \new{to aid the detection of all \unsafe{} usages, and \toolSA{} to identify known potentially dangerous \unsafe{} patterns.}




%% ---------------------------------------------------


\subsection{\toolUsage{}: Identification of Unsafe Usage}
\label{sec:appr:toolUsage}

This section presents \toolUsage{}\footnote{\url{https://github.com/jlauinger/go-geiger}}, a novel tool to identify and quantify usages of \unsafe{} in a Go project and in its dependencies. 
%, which is available on GitHub\footnote{\url{https://github.com/jlauinger/go-geiger}}.
Its development was inspired by \textit{cargo geiger}\footnote{\url{https://github.com/rust-secure-code/cargo-geiger}}, a similar tool for detecting unsafe code blocks in Rust programs.

%\subsubsection*{Approach}

Figure~\ref{fig:geiger-architecture} shows an overview of the architecture of \toolUsage{}.
We use the standard parsing infrastructure provided by Go to identify and parse packages including their dependencies based on user input.
%In particular, we use the \textit{packages.Load} function to parse the sources of all packages requested for analysis including their transitive dependencies.
Then, we analyze the AST, %using the standard \textit{ast.Inspect} function.
which enables us to identify different usages of \unsafe{} and their context as described in the next paragraph.
Finally, we arrange the packages requested for analysis and their dependencies in a tree structure, sum up \unsafe{} usages for each package individually, and calculate a cumulative score including dependencies.
We perform a deduplication if the same package is transitively imported more than once.
The \unsafe{} dependency tree, usage counts, as well as identified code snippets, are presented to the user.

%\subsubsection*{Implementation}

We detect all usages of methods and fields from the \textit{unsafe} package, specifically: \textit{Pointer}, \textit{Sizeof}, \textit{Offsetof}, and \textit{Alignof}.
Furthermore, because they often are used in unsafe operations, we also count occurrences of \textit{SliceHeader} and \textit{StringHeader} from the \textit{reflect} package, and \textit{uintptr}.
All of these usages are referred to as \unsafe{} usages in this paper.
%The first six \unsafe{} types are detected by finding selector expression nodes with matching field names, while \textit{uintptr} usages are found by inspecting identifier nodes in the AST.
Additionally, we determine the context in which the \unsafe{} usage is found, i.e., 
the type of statement that includes the \unsafe{} usage.
In \toolUsage{} we distinguish between assignments (including definitions of composite literals and return statements), calls to functions, function parameter declarations, general variable definitions, or other not further specified usages.
We determine the context by looking up in the AST starting from the node representing the \unsafe{} usage, and identifying the type of the parent node.
%For example, if the nearest relevant ancestor in the AST is an \textit{AssignStmt} node, then the context is determined as assignment.

%% ---------------------------------------------------



\subsection{\toolSA{}: An Unsafe-Focused Linter}
\label{sec:appr:toolSA}

%This section presents \toolSA{}, a novel static code analysis tool to find dangerous \unsafe{} usage patterns that were previously uncaught with existing tools. 
Through the usage of \toolUsage{} in real-world code and our manual analysis in Section~\ref{sec:eval}, we found \unsafe{} code patterns that were not covered by existing linters such as \textit{go vet}. 
To automatically give advice for some of these patterns we designed \toolSA{}\footnote{\url{https://github.com/jlauinger/go-safer}}.
It is meant for assistance during manual audits and also for integration in build chains during development.
%All source code for \toolSA{} is made publicly available\footnote{\url{https://github.com/jlauinger/go-safer}}.
Avoiding the \unsafe{} usage patterns that \toolSA{} detects, prevents the garbage collector race and escape analysis flaw vulnerabilities that we discussed in Section~\ref{sec:appr:vulnerabilites}.

%\subsubsection*{Approach}

\begin{figure}[htp!]
    %\vspace{2mm}
    \centering
    \includegraphics[width=0.7\textwidth]{assets/figures/chapter5/go-safer-architecture.pdf}
    \caption{Architecture of the \toolSafer{} static code analysis tool}
    \label{fig:safer-architecture}
    %\vspace{-14pt}
\end{figure}


Figure~\ref{fig:safer-architecture} shows an overview of the architecture of \toolSA{}.
%It was built on top of the existing infrastructure provided by the \textit{go vet} tool.
First, it uses \textit{go vet} to build a list of packages to be analyzed and parses their sources.
Then, a number of static code analyzers, called \textit{passes}, run.
Our analyses depend on existing passes to acquire the abstract syntax tree (AST) and control flow graph (CFG).
Two separate analyses are run by \toolSA{}: the \textit{sliceheader} and the \textit{structcast} passes. % discovers incorrect string and slice casts as shown in Listing~\ref{lst:string-to-bytes}.
%The \textit{structcast} pass finds unsafe casts between different struct types that include architecture-dependent field sizes and, therefore, might create a security risk when ported to other platforms.
%Describing potential exploit vectors for this second type of problem is beyond the scope of this paper, thus, we added an example of this to the public repository\textsuperscript{\ref{fn:poc}} mentioned in the last section.

%\subsubsection*{Implementation}

The \textit{sliceheader} pass discovers incorrect string and slice casts as shown in Listing~\ref{lst:string-to-bytes}.
It finds composite literals and assignments in the AST.
Then, for each it checks whether the type of the %literal or assignment 
receiver is \textit{reflect.StringHeader}, \textit{reflect.SliceHeader}, or some derived type with the same signature.
For assignments, the analysis pass then finds the last node in the CFG where the receiver object's value is defined, and checks if it is derived correctly by casting a string/slice.
If it can not infer with certainty that the assignment receiver object was created by a cast, then \toolSA{} issues a warning.

The \textit{structcast} pass discovers instances of in-place casts between different struct types that include architecture-dependent field sizes. 
This can create a security risk when ported to other platforms because \unsafe{} casts can lead to misaligned fields, and thus, memory access outside a value's bounds on some platforms, allowing the same exploit vectors as a buffer overflow does.
The pass finds struct cast instances that involve \textit{unsafe.Pointer} in the AST.
Then, it compares the struct types and checks if they contain an unequal amount of fields with types \textit{int}, \textit{uint}, or \textit{uintptr}, which are the architecture-dependent types supported by Go.
If the numbers do not match, \toolSA{} issues a warning.


\section{A Study of Go's \unsafe{} Usages in the Wild}
\label{sec:eval}

%The goal of our real-world study is to understand how \unsafe{} is used in open-source projects, and what the implications of this are on the security of Go applications. 

We designed and performed a study in the wild to answer the following research questions:

\begin{enumerate}[leftmargin=*,label={RQ\arabic*}]
    \item How prevalent is \unsafe{} in Go projects? \label{rq:prevalApp}
    \item How deep are \unsafe{} code packages buried in the dependency tree? \label{rq:depsDepth}
    \item Which \unsafe{} keywords are used most? \label{rq:distTypes}
    \item Which \unsafe{} operations are used in practice, and for what purpose? \label{rq:purpose}
\end{enumerate}

%Figure~\ref{fig:study-overview} provides an overview of our study methodology.
%\begin{figure}[ht]
    \includegraphics[width=\textwidth]{assets/figures/study-methodology.pdf}
    \caption{Overview of our Study Methodology}
    \label{fig:study-methodology}
\end{figure}


In the following, we first describe our evaluation data set and then provide in-depth analyses of \unsafe{} usage in the wild using \toolUsage{}.
% of how prevalent \unsafe{} is in the wild, in which way and why it is used in our test data set.
Our evaluation scripts as well as the results are available online\footnote{\url{https://github.com/stg-tud/unsafe_go_study_results}}.
%for further research.


%% included here for manual positioning one page earlier. Belongs to next section, reposition if needed
\begin{figure*}[!t]
    \centering
    \includegraphics[width=\textwidth]{gfx/figures/unsafe-import-depth.png}
    \caption{Import Depth of Unsafe Packages. Answers~\ref{rq:depsDepth}: unsafe packages are around \averageUnsafeImportDepth{} hops away (sd=\stdUnsafeImportDepth), thus manageable to find manually.}
    \label{fig:unsafe-import-depth}
\end{figure*}




%% ---------------------------------------------------

\subsection{Data Set}

%For our evaluation, we created a data set of open-source Go code available on GitHub.
As our research is focused on open-source projects, we crawled the \initalProjs{} most-starred Go projects available on GitHub. 
To further understand the influence of dependencies, we then selected the applications supporting \textit{go modules}.
With the introduction of Go \checkNum{1.13}, \textit{go modules}\footnote{\url{https://blog.golang.org/using-go-modules}} are the official way to include dependencies.
Unfortunately, \withoutModules{} of the projects did not yet support Go modules.
Thus, we excluded them from our set.
Furthermore, \notCompiled{} projects that did not compile were removed.
As a result, we ended up with \projsAnalyzed{} top-rated Go projects. % collected from GitHub.
These have between 72,988 and 3,075 stars, with an average of 7,860. % and median of 5,345. %, thus, this evaluation focuses on very popular projects.
%and save all findings into CSV files.



%% ---------------------------------------------------

\subsection{Unsafe Usages in Projects and Dependencies}

We used the Go tool chain to identify the root module of each project. 
This is the module defined by the top-level \textit{go.mod} file in the project.
Then we enumerated the dependencies of the project, and build the dependency tree.
For each package, we use \toolUsage{} to generate CSV reports of the found \unsafe{} usages.
Through these analyses we answer the research questions of how many projects use \unsafe{} either in their own code or dependencies (\ref{rq:prevalApp}), and how deep in the dependency tree are the most \unsafe{} code usages (\ref{rq:depsDepth}). 
By selecting only results from the project root modules, we can easily find out how many applications contain a first-hand use of \unsafe{} code.
Our data shows that \checkNum{131} (\checkNum{38.19\%}) projects have at least one \unsafe{} usage within the project code itself.
By looking closer at the imported packages, we see that \checkNum{3,388} of \checkNum{62,025} (\checkNum{5.46\%}) transitively imported packages use \unsafe{}. 
%Through filtering the imported packages, we find that \checkNum{33} of \checkNum{186} (\checkNum{17.74\%}) used standard library packages contain \unsafe{}.
%There are \checkNum{299} (\checkNum{87.17\%}) projects that have at least one direct dependency to a package that has $\geq 1$ unsafe dependency, however for this number we counted only packages belonging to the project's root module as first-party project code. \jl{Might be wrong numbers}
%If a project is split into several modules that all should be logically viewed as first-party project code, they will inflate this number.
There are \checkNum{312} (\checkNum{90.96\%}) projects that have at least one non-standard-library dependency with \unsafe{} usages somewhere in their dependency tree.
Since all projects include the Go runtime, which uses \unsafe{}, counting it as an \unsafe{} dependency would mean that \checkNum{100\%} of projects transitively include \unsafe{}.
We consider this to be less meaningful, as we assume the Go standard library is well audited and safer to use.

%\begin{tcolorbox}[boxsep=1pt, enlarge top by=5pt, title=Answer to \ref{rq:prevalDeps}]
\begin{tcolorbox}[boxsep=1pt, enlarge top by=5pt, title=Answer to \ref{rq:prevalApp}]
About \checkNum{38\%} of projects directly contain \unsafe{} usages.
Furthermore, about \checkNum{91\%} of projects transitively import at least one dependency that contains \unsafe{}.
\end{tcolorbox}

% Using this tree, we can identify the import depth as minimum depth in the tree for each package.
Figure~\ref{fig:unsafe-import-depth} shows the number of packages with at least one \unsafe{} usage by their depth in the dependency tree for every project on its own as a heatmap, alongside the distribution for all projects combined as bars on the left side.
It is evident that most packages with \unsafe{} are imported early in the dependency tree with an average depth of \averageUnsafeImportDepth{}~and a standard deviation of \stdUnsafeImportDepth{}.
This number is very similar to the overall average depth of imported packages (\averageGeneralImportDepth{}). %, regardless of whether they contain usages of \unsafe{}.
While the packages containing \unsafe{} can be manually found and evaluated, this process requires significant resources to handle the increasing number of packages introduced through each dependency. 
For developers only the first level of dependencies, the ones they added themselves, are really obvious.
On this level, \levelOneImportedUnsafePackagesCount{} out of \ImportedUnsafePackagesCount{} imported packages (\levelOneImportedUnsafePackagesShare{}) contain \unsafe{}.

\begin{tcolorbox}[boxsep=1pt, enlarge top by=5pt, title=Answer to \ref{rq:depsDepth}]
Most imported packages containing \unsafe{} usages are found around a depth of \checkNum{3} in the dependency tree.
\end{tcolorbox}



%% ---------------------------------------------------

\subsection{Types and Purpose of Unsafe in Practice}

This section answers \ref{rq:distTypes}) and \ref{rq:purpose}.
%This section answers the research questions which \unsafe{} keywords are used most (\ref{rq:distTypes}), as well as which \unsafe{} operations are used in practice for what purpose (\ref{rq:purpose}).
%
Figure~\ref{fig:unsafe-tokens-distribution} shows the distribution of the different \unsafe{} types in our data set.
Packages that are imported in different versions by the projects are counted once per version, as they might contain different \unsafe{} usages and coexist in the wild.
%We found various different usages of \unsafe{} in the analyzed projects, 
In our data set \textit{uintptr} and \textit{unsafe.Pointer} are used about equally often and by far the most common with almost 100,000 findings. 
Next, \textit{unsafe.Sizeof} is still used a bit ($\sim 3,700$), while the other \unsafe{} types are rarely used~($< 1,000$).

\begin{tcolorbox}[boxsep=1pt, enlarge top by=5pt, title=Answer to \ref{rq:distTypes}]
In the wild, \textit{uintptr} and \textit{unsafe.Pointer} are orders of magnitude more common than other \unsafe{} usages.
\end{tcolorbox}

\begin{figure}[!t]
    \vspace{-12pt}
    \centering
    \includegraphics[width=0.43\textwidth]{gfx/figures/distribution-unsafe-types-pdf.pdf}
    \caption{Distribution of different types of \unsafe{} tokens}
    \label{fig:unsafe-tokens-distribution}
\end{figure}

To learn about the purpose and context in which \unsafe{} is used, we needed to manually analyze code.
Thus, we selected the top \checkNum{10} projects (Table~\ref{tbl:dataset-projects}) with the most \unsafe{} usages in non-standard library packages.
%These projects including the Git revision analyzed plus some additional data are shown in Table~\ref{tbl:dataset-projects}.
From these projects and all their transitive dependencies, we randomly sampled \checkNum{400} code snippets that were found in the \textit{standard library (std)} and \checkNum{1,000} snippets from the remaining packages (\textit{app}).
We define standard library code as all packages that are part of the Go standard library or the \textit{golang.org/x/sys} module, as the \textit{syscall} standard library package is deprecated in favor of this module\footnote{\url{https://golang.org/pkg/syscall}}.
We split the snippets into two groups to analyze if there is a difference between the official standard library and non-standard library code regarding the usage of \unsafe{}.
Then, we identify class labels in two dimensions: what is being done, and for what purpose. 
Finally, we manually analyze all \checkNum{1,400} code snippets and label them accordingly.
The results of this process are shown in Table~\ref{tbl:dataset-classes}.

\begin{table}[htp!]
    \centering
    \caption{Projects selected for labeled data set}
    \label{tbl:dataset-projects}
    \begin{tabular}{llrrl}
        {} & \textbf{Name} &  \textbf{Stars} &  \textbf{Forks} &    \textbf{Revision} \\ \hline
        \rowcolor{verylightgray}
        1  &         kubernetes/kubernetes &  66,512 &  23,806 &  \texttt{fb9e1946b0} \\
        2  &                 elastic/beats &   8,852 &   3,207 &  \texttt{df6f2169c5} \\
        \rowcolor{verylightgray}
        3  &             gorgonia/gorgonia &   3,373 &    301 &  \texttt{5fb5944d4a} \\
        4  &              weaveworks/scope &   4,354 &    554 &  \texttt{bf90d56f0c} \\
        \rowcolor{verylightgray}
        5  &  mattermost/mattermost-server &  18,277 &   4,157 &  \texttt{e83cc7357c} \\
        6  &               rancher/rancher &  14,344 &   1,758 &  \texttt{56a464049e} \\
        \rowcolor{verylightgray}
        7  &                 cilium/cilium &   5,501 &    626 &  \texttt{9b0ae85b5f} \\
        8  &                     rook/rook &   7,208 &   1,472 &  \texttt{ff90fa7098} \\
        \rowcolor{verylightgray}
        9  &             containers/libpod &   4,549 &    539 &  \texttt{e8818ced80} \\
        10 &                       xo/usql &   5,871 &    195 &  \texttt{bdff722f7b} \\
    \end{tabular}
\end{table}
\begin{table*}[!t]
    \centering
    \caption[Labeled unsafe.Pointer usages in application code and standard library samples]%
    {Labeled unsafe.Pointer usages in application code and standard library samples \newline \tiny ~ \newline \small
        \underline{eff}: efficiency, \underline{gen}: generics, \underline{ser}: (de)serialization,
        \underline{inev}: inevitable use, \underline{SR}: safer reflections, \underline{LC}: layout control,
        \underline{EA}: hide from escape analysis, \underline{UU}: unused, \underline{cgo}: CGo mechanics,
        \underline{no GC}: avoid garbage collector, \underline{typ}: types implementation,
        \underline{mem}: memory management \newline \tiny ~}
    \label{tbl:dataset-classes}
    \begin{adjustbox}{max width=\textwidth}
    \begin{tabular}{r|cc|cc|cc|cc|cc|cc|cc|cc|cc|cc|cc|cc|cc}
                          & \multicolumn{2}{c|}{eff} & \multicolumn{2}{c|}{gen} & \multicolumn{2}{c|}{ser} & \multicolumn{2}{c|}{inev} & \multicolumn{2}{c|}{SR} & \multicolumn{2}{c|}{LC} & \multicolumn{2}{c|}{EA} & \multicolumn{2}{c|}{UU} & \multicolumn{2}{c|}{cgo} & \multicolumn{2}{c|}{no GC} & \multicolumn{2}{c|}{typ} & \multicolumn{2}{c|}{mem} & \multicolumn{2}{c}{total} \\ \hline
                          &  app &  std &  app &  std &  app &  std &  app &  std &  app &  std &  app &  std &  app &  std &  app &  std &  app &  std &   app &  std &  app &  std &  app &  std &   app &  std \\ \hline
 conversion-struct-struct &  408 &    4 &   42 &      &    3 &    6 &    1 &      &    1 &      &      &      &      &    2 &      &      &    5 &    2 &       &      &      &   30 &      &    4 &   460 &   48 \\
\rowcolor{verylightgray}
  conversion-struct-basic &   90 &    2 &   27 &      &    2 &    2 &      &    2 &      &      &    2 &    1 &      &      &      &      &    2 &    1 &       &      &      &    1 &      &    7 &   123 &   16 \\
        conversion-header &   37 &    1 &    2 &      &    1 &      &      &      &      &      &      &      &      &      &      &      &      &      &       &      &      &    2 &      &    1 &    40 &    4 \\
\rowcolor{verylightgray}
  conversion-struct-bytes &   26 &    5 &    2 &      &   73 &    6 &      &      &      &      &    2 &    1 &      &      &      &      &    1 &      &       &      &      &    1 &      &      &   104 &   13 \\
       conversion-pointer &   14 &   20 &    8 &      &      &      &    1 &      &    1 &      &      &      &      &      &      &      &   13 &    1 &       &      &      &    9 &      &    3 &    37 &   33 \\
\rowcolor{verylightgray}
     direct-memory-access &    2 &    1 &   10 &      &    1 &      &    1 &      &      &      &    5 &      &      &      &      &      &      &      &       &      &      &    4 &      &    8 &    19 &   13 \\
       pointer-arithmetic &    7 &    4 &    1 &      &      &      &      &      &    1 &      &    7 &    2 &    1 &    2 &      &      &      &    1 &       &      &      &    8 &      &    9 &    17 &   26 \\
\rowcolor{verylightgray}
               definition &    6 &      &    4 &    1 &      &      &      &      &   20 &    1 &    1 &      &      &      &    1 &      &    4 &    4 &       &      &      &    8 &      &   12 &    36 &   26 \\
                 delegate &    3 &      &   58 &    1 &      &      &   33 &   59 &      &      &      &      &      &    5 &      &      &    9 &    1 &       &      &      &    2 &      &    6 &   103 &   74 \\
\rowcolor{verylightgray}
          type-reflection &      &      &   25 &      &      &      &      &      &    3 &      &      &      &      &      &      &      &      &      &       &      &      &    1 &      &      &    28 &    1 \\
                  syscall &      &      &      &      &      &      &      &      &      &      &      &      &      &      &      &      &      &      &    17 &  138 &      &      &      &      &    17 &  138 \\
\rowcolor{verylightgray}
                   unused &      &      &      &      &      &      &      &      &      &      &      &      &      &      &   16 &    8 &      &      &       &      &      &      &      &      &    16 &    8 \\ \hline
                        total &  593 &   37 &  179 &    2 &   80 &   14 &   36 &   61 &   26 &    1 &   17 &    4 &    1 &    9 &   17 &    8 &   34 &   10 &    17 &  138 &    0 &   66 &    0 &   50 &  1000 &  400 \\
\end{tabular}
    \end{adjustbox}
\end{table*}

In the following, we outline the identified usage type classes describing what is being done in code.
The most prevalent are cast operations from arbitrary types to other structs, basic Go types such as integers, slice/string headers, byte slices, or raw \textit{unsafe.Pointer} values. 
%The most prevalent are \textit{cast-struct}, \textit{cast-basic}, \textit{cast-header}, \textit{cast-bytes}, and \textit{cast-pointer}, which are all cast operations from arbitrary types to other arbitrary structs, basic Go types such as integers, slice or string headers, byte slices, or raw \textit{unsafe.Pointer} values.
The \textit{memory-access} class is applied where \textit{unsafe.Pointer} values are dereferenced, used to manipulate corresponding memory or for comparison with another address.
\textit{Pointer-arithmetic} denotes usages of \unsafe{} to do some form of arithmetic manipulation of addresses, such as advancing an array.
\textit{Definition} groups usages where a field or method of type \textit{unsafe.Pointer} is declared for later usage.
\textit{Delegate} are instances where \unsafe{} is only needed in a function to pass it along to another function requiring a parameter of type \textit{unsafe.Pointer}. 
Thus, the need to use \unsafe{} is actually located elsewhere.
\textit{Syscall} are calls using the Go \textit{syscall} package or \textit{golang.org/x/sys} module.
As the name suggests, \textit{unused} is a class of occurrences that are not actually used in the analyzed code, e.g. dead code or unused parameters.

Our identified purpose classes, providing hints on why \unsafe{} is used, are described in the following.
\textit{Efficiency} includes cases where \unsafe{} is used to improve time or space efficiency.
The \textit{serialization} class contains (un)marshalling and (de)serialization operations such as in-place casts from complex types to bytes.
\textit{Generics} applies when \unsafe{} is used to build functionality that would otherwise be solved with generics if they were available in Go.
Samples in the \textit{avoid garbage collection} class are used to tell the Go compiler to not free a value while it is used, e.g., by a function written in assembly.
The \textit{atomic operations} class contains usages of the \textit{atomic} API which expects \unsafe{} for some functions.
The \textit{foreign function interface (FFI)} class contains interoperability with C code (CGo), and calling  functions that expect their parameters as \unsafe{} pointers.
\textit{Hide from escape analysis} includes the pattern described earlier (Listing~\ref{lst:unsafe-ex-escape-analysis}) to break the escape analysis chain.
The \textit{memory layout control} class contains code used for low-level memory management.
\textit{Types} snippets are used by the standard library to implement the Go type system.
\textit{Reflect} includes instances of type reflection and re-implementations of some types of the \textit{reflect} package, e.g. using \textit{unsafe.Pointer} instead of \textit{uintptr} for slice headers.
Again, \textit{unused} is a class of unused occurrences.

%Among the \checkNum{1,400} labeled snippets, \checkNum{683} are located in automatically generated code.
%It might be safe to assume that generated code is less dangerous, however, as bugs in the code generator can have very serious effects of scale, we included them in the study nonetheless.

Improving the efficiency of Go code is the most prevalent motivation to use \unsafe{} in the wild covering \checkNum{58\%} in application code, whereas it is only used for this purpose in \checkNum{5\%} of the cases in std. 
From these, \checkNum{97\%} resp. \checkNum{80\%} are achieved by casting different types. 
The second biggest reason to use \unsafe{} in app is to perform some form of (de)serialization, accounting for \checkNum{28\%}.
%Interestingly, we observe efficiency improvements for \checkNum{4\%} and \checkNum{58\%} of the usages for the standard library and the remaining libraries, respectively.
For the standard library, the most relevant motivation is avoiding garbage collection with \checkNum{35\%}, whereas this is only used in \checkNum{2\%} of the usages in the app sample.
Furthermore, in std type \checkNum{18\%}, FFI \checkNum{15\%} and memory layout \checkNum{13\%} related \unsafe{} usages are rather common.
Both subsets share that hiding from escape analysis with \checkNum{0.1\%} (\textit{app}) and \checkNum{2\%} (\textit{std}) and using \unsafe{} for reflection with \checkNum{1\%} (both) are rare.
%Another interesting observation is that there seem to be two main motivations which occur only for the standard- or 3rd party libraries respectively. 
%Only the standard library makes use of \unsafe{} to implement types being \checkNum{5\%} of the analyzed code snippets. 
%All usages (\checkNum{2\%}) to solve the missing generics functionality in Go are within 3rd party libraries. 
Implementation of generics functionality which is currently missing in Go is only done in few samples (\checkNum{2\%}), although some of the finding in the serialization class could alternatively be achieved with generics as well.

%It is obvious that most of the \unsafe{} found in the wild is used to improve efficiency by casting different types in place, or for fast (un)marshalling operations that would otherwise need either reflection or support for generics.
%Furthermore, \unsafe{} is required to call some functions that expect it, such as atomic operations or certain \textit{FFI} functions.
%The standard library makes extensive use of \unsafe{} to implement types and memory management.
%Hiding values from escape analysis on purpose is done rarely.

\begin{tcolorbox}[boxsep=1pt, enlarge top by=5pt, title=Answer to \ref{rq:purpose}]
\checkNum{More than half} of the \unsafe{} usages in projects and 3rd party libraries are to improve efficiency via \unsafe{} casts.
In the Go standard library, \checkNum{every third} use of \unsafe{} is to avoid garbage collection. 
\end{tcolorbox}


\subsection{Vulnerable Usages}
\label{sec:discussion}
Looking at the study results, we see that \unsafe{} is used consistently and wide-spread in the most popular open-source Go projects.
One might argue that the patterns found by \toolUsage{} are only minor annoyances, not severe or would require a manual case-to-case inspection. 
The exploitability of several of these patterns was discussed in Section~\ref{sec:appr} with a link to proof-of-concept exploits that we developed, clearly showing that it is indeed possible to use the memory corruptions to one's advantage.
However, not all \unsafe{} usages contain a vulnerability. 
As already discussed, we implemented more specific checks for some of our patterns known to be problematic in \toolSA{}.
The application of \toolSA{} to our data set revealed more than \numberBugsFixed{} bugs in different projects.
Based on the results, we submitted so far \numberPRs{} pull requests to fix the found bugs. %, which focuses on a subset of vulnerable code patterns.
By now, \numberPRsMerged{} have already been reviewed, acknowledged, and accepted by the corresponding project maintainers.
\section{Threats to Validity}
\label{sec:threatsToValidity}

%Since we showed actual exploit vectors and bugs with code snippets that are commonly used in popular real-world applications, we think that research on \unsafe{} is clearly not only an academic issue, but has direct and important consequences on the actual software market.
%Audit efforts should be made up to an import depth of around 4.


Potential internal threats to validity for our study include bias towards bigger projects because those might be over-represented in the manually labeled data set. 
External threats include a bias towards more active projects with many developers because we selected a subset of the most-starred open-source projects on GitHub. 
Also we only considered projects that use the Go module system and about a third of the top 500 projects are not covered by the analysis yet.
Further, we could have missed projects from a special domain not having that many stars which might have other usage scenarios for \unsafe{} Go.
Nevertheless, one can argue that the biggest projects also have professional developers, higher standards and code gets more reviewed, thus, code quality should be higher.


\section{Related Work}
\label{sec:rw}

% Analyses in Go / Go and Security
%\textbf{Analyses in Go}
Previous research on Go mostly concentrated on issues related to its concurrency model including the channel implementation~\cite{tu2019,dilley2019,giunti2020,gabet2020,lange2017,bodden2016information}.
The work by Wang et al.~\cite{wang2020} suggests an improvement of the existing escape analysis in Go which we also discussed in our paper. 

% Unsafe in other languages
%\textbf{Unsafe usages in memory safe languages}

%While previous research on Go did not answer the question on the influence of \unsafe{} code, related studies exist for other languages. 
Moreover, the usage of \unsafe{} in other languages has already been studied to varying degrees. 
For Java, Mastrangelo et al.~\cite{mastrangelo2015} identified that 25\% of the analyzed artefacts depend on the Java \unsafe{} library.
The different JVM crash patterns caused by those usages are analyzed by Huang et al.~\cite{huang2019}.
Recently, two studies analyzed \unsafe{} usages in Rust projects and identified that \unsafe{} is widely used to improve performance or to reuse existing code~\cite{qin2020,evans2020}.
Furthermore, work was presented on how to ensure memory safety while using \unsafe{} in Rust~\cite{hussain2018Fidelius}.
Lehmann et al.~\cite{lehmann-everything-2020} studied to which extent \unsafe{} programs compiled to WebAssembly can lead to vulnerabilities within the virtual machine environment. %and it is possible to compile Go code to WebAssembly. 
%
% Shorter version 
%Recently, related studies were published that focus on unsafe code and its usage in Rust~\cite{qin2020,evans2020} and Java~\cite{mastrangelo2015,huang2019}.
%Lehmann et al.~\cite{lehmann-everything-2020} studied to which extend unsafe programs can lead to vulnerabilities in WebAssembly. %and it is possible to compile Go code to WebAssembly. 
%
% Memory vulnerabilities and C/C++
For C/C++, non memory-safe languages, research exists on how to support at least partial memory safety~\cite{burow2018CUP, nagarkatte2009SoftBound} and work on identifying vulnerabilities by program analyses~\cite{song2019sok}.
A comprehensive study on memory management related vulnerabilities, like the ones we discussed earlier, and their mitigations is presented in earlier work~\cite{szekeres2013sok}.

%Dependencies
Concerning project dependencies, it is difficult to count the dependencies that matter the most, e.g. by excluding test dependencies~\cite{pashchenko2018}.
A common problem is that dependencies are often updated slowly, keeping old bugs alive, although measures such as automated pull requests exist to mitigate this problem~\cite{derr2017keep, mirhosseini2017, lauinger2017}.



\section{Conclusion}
\label{sec:concl}

In this paper, we presented two novel tools to help Go developers write safer code with respect to \unsafe{} Go.
\toolUsage{} identifies and counts \unsafe{} usages not only within a package, but also in its transitive dependencies, therefore providing effective tools to focus audit efforts on the code locations that are the most dangerous, and raising awareness into how \unsafe{} code is included into a project.
\toolSA{} is a new static code analysis tool that helps developers identify two dangerous code patterns that were previously uncaught with existing tools.
We presented novel evidence of how to exploit common unsafe code patterns, and submitted fixes to \numberBugsFixed{} bugs.
We conducted a large-scale study of \packagesAnalyzed{} in \projsAnalyzed{} top-starred open-source Go projects and found that \percentagePackagesWithUnsafe{} of packages used transitively by the projects contain \unsafe{} usages. 
\percentageProjectsWithUnsafe{} of projects contain at least one usage that is not part of the standard library, and \percentageProjectsAndDependenciesUnsafe{} of projects contain at least one \unsafe{} in the project or one of its transitive dependencies.
Finally, we presented a new data set of manually labeled code snippets, providing insight into how and for what purpose \unsafe{} is used by developers.

Based on our contributions, future work might look into supervised learning algorithms using our labeled data set to train classifiers that can identify the purpose and domain of \unsafe{} usages by looking at new code examples.
As Smith et al. show, avoiding security vulnerabilities and fixing bugs can be made easier for developers by integrating code analysis tools into code editors and IDEs \cite{smith2020}.
Thus, future work could build plugins for common IDEs that integrate our tools \toolUsage{} and \toolSA{}.


% conference papers do not normally have an appendix

% use section* for acknowledgment
\ifCLASSOPTIONcompsoc
  % The Computer Society usually uses the plural form
  \section*{Acknowledgments}
\else
  % regular IEEE prefers the singular form
  \section*{Acknowledgment}
\fi

This work has been funded by the German Federal Ministry of Education and Research and
the Hessen State Ministry for Higher Education, Research and the Arts within their joint support of
the National Research Center for Applied Cybersecurity ATHENE.


% trigger a \newpage just before the given reference
% number - used to balance the columns on the last page
% adjust value as needed - may need to be readjusted if
% the document is modified later
%\IEEEtriggeratref{8}
% The "triggered" command can be changed if desired:
%\IEEEtriggercmd{\enlargethispage{-5in}}

% references section

% can use a bibliography generated by BibTeX as a .bbl file
% BibTeX documentation can be easily obtained at:
% http://mirror.ctan.org/biblio/bibtex/contrib/doc/
% The IEEEtran BibTeX style support page is at:
% http://www.michaelshell.org/tex/ieeetran/bibtex/
\bibliographystyle{IEEEtran}
\balance
\bibliography{IEEEabrv,references}

% that's all folks
\end{document}
