\chapter*{Abstract}

\checkNum{One decade} after its first published version, the Go programming language has become a popular and
widely-used modern programming language.
It aims to achieve thorough memory and thread safety by using measures such as a strict type system and automated memory
management with garbage collection, which prevents invalid memory access.
However, there is also the \unsafe{} package, which allows developers to deliberately circumvent this safety net.
There are a number of legitimate use cases for doing this, for example, an in-place type conversion saving reallocation
costs to improve efficiency, or interacting with C code through the foreign function interface.

Misusing the \unsafe{} API can however lead to security vulnerabilities such as buffer overflow and
\textit{use-after-free} bugs.
This work contributes an analysis of \unsafe{} usage patterns with respect to a security context.
It reveals possible code injection and information leak vulnerabilities in proof-of-concept exploits as well as common
usages from real-world code.

To assess the risk of \unsafe{} code in their applications, this work presents \toolGeiger{}, a novel tool to help
developers quantify \unsafe{} usages not only in their project itself, but including its dependencies.
Using \toolGeiger{}, a study on \unsafe{} usage in the top \projsTotal{} most popular open-source Go projects on
\github{} was conducted, including a manual study of \numberLabeledCodeSnippets{} individual code samples on how
\unsafe{} is used and for what purpose.
The study shows that \percentageUnsafePackages{} of packages imported by the projects using the Go module system use
\unsafe{}.
Furthermore, \percentageUnsafeProjects{} of the projects use \unsafe{} directly, and
\percentageUnsafeTransitiveWithDependencies{} include \unsafe{} usages through any of their dependencies.
A replication and comparison of a concurrent study by Costa et al.~\cite{costa2020} matches these results.

This work further presents \toolSafer{}, a novel static code analysis tool that helps developers to identify
\checkNum{two} dangerous and common misuses of the \unsafe{} API, which were previously undetected with existing tools.
Using \toolSafer{}, \numberBugsFixed{} bugs in real-world code were identified and patches have been submitted to and
accepted by the maintainers.
An evaluation of the tool shows \goSaferEvaluationDatasetGosaferAccuracy{} accuracy on the data set of labeled \unsafe{}
usages, and \goSaferEvaluationPackagesGosaferAccuracy{} accuracy on a set of manually inspected open-source Go packages.


\chapter*{Zusammenfassung}

\checkNum{Ein Jahrzehnt} nach der ersten veröffentlichten Version ist die Programmiersprache Go heute eine beliebte und
weit verbreitete, moderne Sprache.
Sie strebt Speicher- und Threadsicherheit durch Maßnahmen wie ein striktes Typsystem und automatische
Speicherverwaltung, die ungültige Speicherzugriffe verhindert, an.
Es gibt allerdings ebenfalls das \unsafe{} Package, eine API, die es Entwickler*innen erlaubt, diese
Maßnahmen zu umgehen.
In manchen Fällen kann dies gerechtfertigt sein, beispielsweise bei der Konvertierung von Daten in einen anderen Typ,
ohne diese im Speicher zu kopieren, um so die Effizienz des Programms zu steigern, oder um externen C Code über das
Foreign Function Interface zu nutzen.

Eine falsche Benutzung der \unsafe{} API kann jedoch zu Sicherheitsproblemen wie Buffer Overflows und
\textit{Use-After-Frees} führen.
Diese Arbeit analysiert Verwendungsmuster von \unsafe{} Code im Hinblick auf Sicherheitsrisiken.
Dabei werden mögliche Code Injection und Information Leak Verwundbarkeiten sowohl in Proof-of-Concepts als auch in
realem Anwendungscode zu Tage gebracht.

Um die Risiken durch \unsafe{} Code in Anwendungen abzuschätzen, stellt diese Arbeit \toolGeiger{} vor.
Es handelt sich dabei um ein neues Werkzeug, das Entwickler*innen dabei hilft, \unsafe{} Nutzungen in Projekten und
deren Abhängigkeiten zu finden.
Mit \toolGeiger{} wird eine Studie zur Nutzung von \unsafe{} in den \projsTotal{} beliebtesten Open-Source Go Projekten
auf \github{} durchgeführt, inklusive einer manuellen Analyse von \numberLabeledCodeSnippets{} individuellen
Codestücken in Bezug darauf wie und zu welchem Zweck \unsafe{} benutzt wird.
Die Studie zeigt, dass \percentageUnsafePackages{} der Packages, die von Projekten importiert werden, welche das Go
Modules System unterstützen, \unsafe{} verwenden.
Darüber hinaus nutzen \percentageUnsafeProjects{} der Projekte \unsafe{} direkt, und
\percentageUnsafeTransitiveWithDependencies{} enthalten \unsafe{} Code durch ihre Abhängigkeiten.
Eine Replikation sowie ein Vergleich mit einer zeitgleichen Studie von Costa et al.~\cite{costa2020} bestätigt diese
Ergebnisse.

Weiterhin präsentiert diese Arbeit \toolSafer{}, ein neues statisches Analysewerkzeug, das Entwickler*innen hilft,
\checkNum{zwei} gefährliche und häufig vorkommende inkorrekte Verwendungen der \unsafe{} \acrshort{API}, die mit bisher
existierenden Tools nicht gefunden werden, zu identifizieren.
Mittels \toolSafer{} konnten \numberBugsFixed{} Fehler in realem Code gefunden und entsprechende Patches eingereicht
werden, die von den Maintainern bestätigt wurden.
Eine Evaluation des Tool ergibt eine Accuracy von \goSaferEvaluationDatasetGosaferAccuracy{} auf dem Datensatz von
\unsafe{} Codezeilen, und \goSaferEvaluationPackagesGosaferAccuracy{} Genauigkeit auf händisch analysierten Open-Source
Go Packages.
