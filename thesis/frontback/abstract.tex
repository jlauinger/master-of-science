\chapter*{Abstract}

After \checkNum{one decade} since its first published version, the Go programming language has become a popular and
widely used systems programming language.
It aims to achieve thorough memory and thread safety by using compile-time measures such as a strict type system that
prevents invalid memory access.
However, there is also the \unsafe{} package which allows developers to deliberately circumvent this safety net.
There are a number of legitimate use cases for doing this, for example a low-level network protocol implementation that
needs access to the raw byte representation of some data, or an in-place type conversion saving reallocation costs to
improve efficiency.

Misusing the \unsafe{} API can however lead to security vulnerabilities such as buffer overflow and use-after-free bugs.
This work contributes an analysis of \unsafe{} usage patterns with respect to a security context.
It reveals possible code injection and information leak vulnerabilities in proof-of-concept code as well as common code
usages from real-world code.

To assess the risk of \unsafe{} code in their applications, this work presents \toolGeiger{}, a novel tool to help
developers quantify \unsafe{} usages not only in their project itself but including its dependencies.
Using \toolGeiger{}, I conducted a study on \unsafe{} usage in the top \projsTotal{} most popular open-source Go
projects on \github{}, including a manual study of \numberLabeledCodeSnippets{} individual code samples on how \unsafe{}
is used and for what purpose.
The study shows that \percentageUnsafePackages{} of packages imported by the projects using the Go module system use
\unsafe{}.
\percentageUnsafeProjects{} of the projects use \unsafe{} directly, but \percentageUnsafeTransitiveWithDependencies{}
include \unsafe{} usages through any of their dependencies.
A replication and comparison of a concurrent study by Costa et al.~\cite{costa2020} matches these results.

This work further presents \toolSafer{}, a novel static code analysis tool that helps developers to identify
\checkNum{two} dangerous and common misuses of the \unsafe{} API which were previously undetected with existing tools.
Using \toolSafer{}, I identified \numberBugsFixed{} bugs in real-world code and submitted patches to the maintainers.
An evaluation of the tool shows \goSaferEvaluationDatasetGosaferAccuracy{} accuracy on my data set of labeled \unsafe{}
usages, and \goSaferEvaluationPackagesGosaferAccuracy{} accuracy on a set of manually inspected open-source Go packages.


\chapter*{Zusammenfassung}

Nach \checkNum{einem Jahrzehnt} seit der ersten veröffentlichten Version ist die Programmiersprache Go heute eine
beliebte und weit verbreitete Systemprogrammierungssprache.
Sie strebt Speicher- und Threadsicherheit durch Compiler-Maßnahmen wie ein striktes Typsystem, das
ungültige Speicherzugriffe verhindert, an.
Es gibt allerdings ebenfalls das \unsafe{} Package, eine API die es Entwickler*innen erlaubt, diese
Maßnahmen zu umgehen.
In manchen Fällen kann dies gerechtfertigt sein, beispielsweise bei der Implementierung eines
Netzwerkprotokolls, das direkten Zugriff auf die konkreten Daten benötigt, oder zur Konvertierung von Daten in einen
anderen Typ, ohne diese im Speicher zu kopieren, um so die Effizienz des Programms zu steigern.

Eine falsche Benutzung der \unsafe{} API kann jedoch zu Sicherheitsproblemen wie Buffer Overflows und Use-After-Frees
führen.
Diese Arbeit analysiert Verwendungsmuster von \unsafe{} Code im Hinblick auf Sicherheitsrisiken.
Dabei treten mögliche Code Injection und Information Leak Verwundbarkeiten sowohl in Proof-of-Concepts als auch in
realem Anwendungscode zu Tage.

Um die Risiken durch \unsafe{} Code in Anwendungen abzuschätzen, stellt diese Arbeit \toolGeiger{} vor, ein neues
Werkzeug das Entwickler*innen dabei hilft, \unsafe{} Nutzungen nicht nur in ihren Projekten sondern auch in deren
Abhängigkeiten zu finden.
Mit \toolGeiger{} führe ich eine Studie zur Nutzung von \unsafe{} in den Top \projsTotal{} beliebtesten Open-Source Go
Projekten auf \github{}, inklusive einer manuellen Analyse von \numberLabeledCodeSnippets{} individuellen Codestücken
in Bezug darauf wie und zu welchem Zweck \unsafe{} benutzt wird.
Die Studie zeigt, dass \percentageUnsafePackages{} der Packages, die von Projekten die das Go Modules System
unterstützen, importiert werden, \unsafe{} verwenden.
\percentageUnsafeProjects{} der Projekte nutzen \unsafe{} direkt, und \percentageUnsafeTransitiveWithDependencies{}
enthalten \unsafe{} Code durch ihre Abhängigkeiten.
Eine Replikation sowie ein Vergleich mit einer zeitgleichen Studie von Costa et al.~\cite{costa2020} bestätigt diese
Ergebnisse.

Weiterhin präsentiert diese Arbeit \toolSafer{}, ein neues statisches Analysewerkzeug, das Entwickler*innen hilft,
\checkNum{zwei} gefährliche und häufig vorkommende inkorrekte Verwendungen der \unsafe{}, die mit bisher existierenden
Tools nicht gefunden werden, zu identifizieren.
Mittels \toolSafer{} konnte ich \numberBugsFixed{} Fehler in realem Code finden und entsprechende Patches einreichen.
Eine Evaluation des Tool ergibt eine Accuracy von \goSaferEvaluationDatasetGosaferAccuracy{} auf meinem Datensatz von
\unsafe{} Codezeilen, und \goSaferEvaluationPackagesGosaferAccuracy{} Accuracy auf händisch analysierten Open-Source
Go Packages.
