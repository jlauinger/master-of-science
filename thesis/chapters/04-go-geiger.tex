%% ---------------------------------------------------------------------------------------------------------------------

\chapter{\textit{go-geiger}: Identification of Unsafe Usage}\label{ch:go-geiger}

This chapter presents \toolGeiger{}, a tool to find usages of the \unsafe{} \acrshort{API} in Go packages and their
dependencies.
Figure~\ref{fig:outline4} shows an outline of the contents of this chapter.
Using open-source projects from \github{} and \toolGeiger{}, an empirical study on \unsafe{} usage presented which is
used additionally to quantitatively evaluate \toolGeiger{}.
Furthermore, a novel data set of labeled samples of \unsafe{} code is described and used as a qualitative analysis on
\unsafe{} usage in Go.

\begin{figure}[htp!]
    \includegraphics[width=\textwidth]{assets/figures/chapter4/outline4.pdf}
    \caption{Organization of Chapter 4}
    \label{fig:outline4}
\end{figure}



%% ---------------------------------------------------------------------------------------------------------------------

\section{Design}\label{sec:go-geiger:design}

The novel tool \toolGeiger{} is designed to identify usages of the \unsafe{} \acrshort{API} in Go source code.
It is available on \github{}\footnote{\url{https://github.com/jlauinger/go-geiger}}.
The tool includes the dependencies of Go packages in the analysis, which gives a more complete picture of possible
\unsafe{} usages than only looking at an individual package.
It is inspired by \toolCargoGeiger{}\footnote{\url{https://github.com/rust-secure-code/cargo-geiger}}, a similar tool
for detecting the use of unsafe code blocks in Rust programs.
Figure~\ref{fig:go-geiger-architecture} shows the architecture of \toolGeiger{}.

\begin{figure*}[t!]
    \includegraphics[width=\textwidth]{gfx/figures/go-geiger-architecture.png}
    \caption{Go-Geiger Unsafe Detection Tool Design}
    \label{fig:geiger-architecture}
\end{figure*}


First, \toolGeiger{} determines the scope of code that should be analyzed for \unsafe{} usage.
To achieve this, the dependency tree of the packages given to \toolGeiger{} is built.
Then, the sources of all the packages in this tree are parsed, and the resulting abstract syntax tree (\acrshort{AST})
is inspected.
Within the \acrshort{AST}, usages of the \unsafe{} \acrshort{API} are identified.
Each usage is assigned a tuple of labels consisting of \textit{match type} and \textit{context type}.
The match type represents the part of the \unsafe{} \acrshort{API} that is used.
It can be one of the four \unsafe{} package members \textit{Pointer}, \textit{Alignof}, \textit{Offsetof}, and
\textit{Sizeof}, the \textit{reflect} package fields \textit{SliceHeader} and \textit{StringHeader}, or the
\textit{uintptr} keyword.
The context type indicates the functional part of code that the usage is found in.
It can be either an \textit{assignment}, a \textit{call} of a function, a function \textit{parameter} definition, or a
\textit{variable} definition.
Context identification is done based on the abstract syntax tree (\acrshort{AST}), which might lead to edge cases in
some programs that are not handled in \toolGeiger{}.
For example, this could happen if future releases of Go introduce new \acrshort{AST} node types.
To mitigate this possibility, the context is set to \textit{other} in \toolGeiger{} if all four other options can not be
applied.
The context allows to filter the search for \unsafe{} usages in particular functional code positions.
For example, it is possible to find only \unsafe{} usages that are used within parameters of function calls.
After the \unsafe{} usages are identified, they are counted.
For this, it is necessary to take care of package deduplication.
If a particular package exists in the dependency tree multiple times and is reachable on different paths, it must still
be counted only once when calculating the sum of \unsafe{} usages in a package's dependencies.
This is because the package does not get any less safe by including the same code multiple times.
The code is already part of the resulting program.
Finally, the analysis results are shown to the user.
It is possible to output the \unsafe{} usage counts as well as the lines of code containing the usages.

Figure~\ref{fig:go-geiger-screenshot} shows a screenshot of \toolGeiger{}.
It presents the analyis results for the \textit{go-geiger} source code itself.
However, to decrease the space needed to display the figure, parts of the table with \unsafe{} counts have been
excluded.

\begin{figure}[htp!]
    %\vspace{2mm}
    \centering
    \includegraphics[width=\textwidth]{assets/images/chapter4/go-geiger-screenshot.png}
    \caption{Usage example screenshot of \toolGeiger{}}
    \label{fig:go-geiger-screenshot}
    %\vspace{-14pt}
\end{figure}


The output shown in Figure~\ref{fig:go-geiger-screenshot} is composed of three parts.
The first is a table showing the \unsafe{} usage counts for each package in the dependency tree for the packages
given for analysis.
The first column indicates the total number of usages in the package including all its dependencies.
The second column shows the number for the package alone, without the dependencies.
The heading for this column is \textit{local package}.
Next, there are five columns with individual usage counts for the possible usage contexts described above.
These columns add up to the \textit{local package} count.
Finally, the import path for the package is given to identify it.
Lines that are printed in green (Lines~11 and~12 in Figure~\ref{fig:go-geiger-screenshot}) represent packages with no
local \unsafe{} usages.
Red lines (Line~9) are packages that directly contain \unsafe{} code, and white lines (e.g.~Line~3) indicate that the
package does not contain \unsafe{} usages itself but instead introduces them through its dependencies.
After the table, a summary of the number of packages belonging into these three categories is shown (Lines~14--16).
The output concludes with a legend for the colors (Lines~18--20).


%% ---------------------------------------------------------------------------------------------------------------------

\section{Implementation}\label{sec:go-geiger:implementation}

The identification of the dependency trees for the packages to analyze, as well as the parsing of the source code is
done using the standard Go compiler toolchain.
It is accessable using the \textit{packages}
\acrshort{API}\footnote{\url{https://pkg.go.dev/golang.org/x/tools/go/packages}}.
Similarly, the inspection of the \acrshort{AST} is done using the \acrshort{API} available though the \textit{ast}
package in Go\footnote{\url{https://golang.org/pkg/go/ast/}}.

To find \unsafe{} usages corresponding to the different match types shown in Figure~\ref{fig:go-geiger-architecture},
the \acrshort{AST} is filtered for selector expressions (\textit{SelectorExpr} nodes) and identifiers (\textit{Ident}
nodes).
The first indicate possible usages of \textit{unsafe.Pointer}, \textit{Alignof}, \textit{Offsetof}, \textit{Sizeof},
\textit{reflect.SliceHeader}, or \textit{StringHeader}.
The second is used to find instances of \textit{uintptr}.
In both cases the concrete identifier names in the \acrshort{AST} nodes are checked to distinguish \unsafe{} usages from
arbitrary other field accesses.
The context type is determined by going up in the \acrshort{AST} starting at the expression node corresponding to a
given \unsafe{} usage.
For example, an \unsafe{} usage is considered part of the \textit{assignment} context type if it is a descendent of
either an assignment statement (\textit{AssignStmt} node), composite literal (\textit{CompositeLit}), or return
statement (\textit{ReturnStmt}).
The \textit{call}, \textit{parameter}, and \textit{variable} classes correspond to call expression (\textit{CallExpr}),
function declaration (\textit{FuncDecl}), and general declaration (\textit{GenDecl}) nodes, respectively.
To achieve an effective assignment of the context type, the order in which the different types are checked is important.
This is because an \unsafe{} usage could be included in several context types in an example like
\textit{x := f(unsafe.Pointer(y))}.
The \acrshort{AST} for this statement consists of an assignment statement, where the right hand side is a call
expression.
This call expression contains the \unsafe{} usage.
Therefore, by ascending in the \acrshort{AST} and depending on the order of checking, both context types \textit{call}
and \textit{assignment} could be assigned.
Which order should be applied is a design decision, and for \toolGeiger{} it is \textit{assignment}, \textit{call},
\textit{parameter}, and finally \textit{variable}.

The \unsafe{} usage counts are first collected individually for every package.
A cache is used to avoid analyzing the same package multiple times if it is present several times in the dependency
tree.
This cache is aware of possibly different versions of the same package and stores them separately.
When the usage count including dependencies is calculated, \toolGeiger{} starts with the root packages that were
requested for analysis by the user, and recursively calculates the respective counts.
Again, the cache is used to avoid summing up multiple times.
This approach is a depth-first traversal of the dependency tree.

Automated acceptance testing verifies that \toolGeiger{} works as intended.
The test is done by running \toolGeiger{} on a test fixture package with a known number of \unsafe{} usages.
Then, the output is checked for the expected usage counts.


%% ---------------------------------------------------------------------------------------------------------------------

\section{Quantitative Evaluation}\label{sec:go-geiger:quantitative-evaluation}

To evaluate \toolGeiger{} on real-world code, it is used to gather empirical data about the usage of \unsafe{} in
popular open-source Go projects.
This section presents a study which was designed to answer the following research questions:

\begin{enumerate}[left=0.5cm, label={RQ\arabic*}]
    \item How prevalent is \unsafe{} in Go projects? \label{rq:prevalApp}
    \item How deep are \unsafe{} code packages buried in the dependency tree? \label{rq:depsDepth}
    \item Which \unsafe{} keywords are used most? \label{rq:distTypes}
    \item Is there a correlation between \unsafe{} usage and common project metrics such as age or popularity? \label{rq:popularity}
    \item How does the use of \unsafe{} change over time? \label{rq:changeTime}
    \item Does \toolGeiger{} provide additional insights into \unsafe{} usage compared to existing linter tools? \label{rq:linterComparison}
    \item Which \unsafe{} operations are used in practice, and for what purpose? \label{rq:purpose}
\end{enumerate}

The following subsections first discuss how the data for this study was gathered and then answer research
questions~\ref{rq:prevalApp} to~\ref{rq:linterComparison}.
Section~\ref{subsec:go-geiger:qualitative-evaluation:purpose} presents the answer to~\ref{rq:purpose}.
It is answered later, because it is based on a novel, manually labeled data set of \unsafe{} usage purposes, which is
introduced in Section~\ref{subsec:go-geiger:qualitative-evaluation:labeled-dataset}.


%% ---------------------------------------------------------------------------------------------------------------------

\subsection{Data Set}\label{subsec:go-geiger:evaluation:data-set}

To build a data set of \unsafe{} usages, first the top \projsTotal{} most-starred open-source Go projects available on
\github{} were downloaded.
This initial download was done on \checkNum{May 27, 2020}.
These \projsTotal{} projects with their respective revisions are listed in the appendix (Table~\ref{tbl:projects}).
Since \toolGeiger{} is specifically built to analyze the project dependencies, all projects that do not support the Go
modules system were removed from the set.
This is necessary to ensure that all dependencies can be automatically resolved and the respective sources are available
for analysis.
There were \projsWithoutModules{} projects that had no support for modules.
They are indicated in Table~\ref{tbl:projects} by the \textit{nm} label in the last column.
Furthermore, \projsNotCompiled{} projects could not be compiled and thus also had to be excluded from the data set.
The \textit{bf} label is used in the table to mark those projects.
Since \toolGeiger{} analyzes the \acrshort{AST}, it can not work on packages that can not be parsed.
This results in a set of \projsAnalyzed{} Go projects.
These have between \checkNum{3,075} and \checkNum{72,988} stars, with an average of \checkNum{7,860}.

As a next step, the dependency trees for all projects selected for analysis were built, resulting in \packagesAnalyzed{}
unique packages.
These consist of \checkNum{186} packages from the Go standard library, and \checkNum{61,839} which are not.
The average size of the packages \checkNum{1,402} lines of code (\acrshort{LOC}) with a standard deviation of
\checkNum{5,612} \acrshort{LOC} in \checkNum{4.4} Go files on average (standard deviation \checkNum{11.3}).
These packages were analyzed using \toolGeiger{} as well as the existing static analysis tools \toolVet{} and
\toolGosec{} to allow a comparison between the findings of these tools.
The resulting findings were stored in machine-readable \acrshort{CSV} files.
The data set contains \uniqueUnsafeFindings{} unique \unsafe{} usages.
All data files as well as the data acquisition tool used to download the projects and run the analysis tools are
available in a data repository on \github{}\footnote{\url{https://github.com/stg-tud/unsafe_go_study_results}}.
Additionally, the data repository\footnote{\url{https://zenodo.org/record/4107497}} and the raw projects source
code\footnote{\url{https://zenodo.org/record/4001728}} are published on \zenodo{}.


%% ---------------------------------------------------------------------------------------------------------------------

\subsection{Usages in Projects and Dependencies (\ref{rq:prevalApp},~\ref{rq:depsDepth},~\ref{rq:distTypes})}\label{subsec:go-geiger:evaluation:unsafe-usage}

To attribute \unsafe{} usages to either a project (as first-party code) or one of its dependencies (as third-party
code), the root module of the project is used.
Since the data set was constructed such that all projects support the Go module system as described in the previous
section, there is a top-level \textit{go.mod} file present for each of them.
The module specified in that file is stored with the project information as its root module.
Packages that are part of this module are (first-party) project code.
Other packages that are present in the dependency tree but not in the root module are (third-party) dependencies.

By looking at \unsafe{} findings that are part of first-party packages, it is possible to determine how many projects
directly use \unsafe{} in their code.
The data shows that this is the case for \unsafeProjects{} (\percentageUnsafeProjects{}) of the \projsAnalyzed{}
projects.
However, this does not take the dependencies into account.
With respect to these, \unsafeTransitiveWithDependencies{} (\percentageUnsafeTransitiveWithDependencies{}) of the
projects transitively import \unsafe{} usages.
To calculate this percentage, the complete dependency trees of the projects are integrated into the analysis.
This results in \packagesAnalyzed{} unique packages, \unsafePackages{} (\percentageUnsafePackages{}) of which contain at
least one \unsafe{} usage.
This answers research question~\ref{rq:prevalApp} about the prevalence of \unsafe{} in Go projects.

The numbers in the previous paragraph do not include the Go standard library.
All analyzed projects import this library and it contains \unsafe{} usages, which means that with it \checkNum{100\%} of
the projects would transitively use \unsafe{}.
Because the standard library is developed by the Go core team, we assume that it is well audited and safe to use.
Since there is no way to exclude it from a project anyways, the analysis presented in this study is more meaningful
when the standard library is not causing a project to be counted as using \unsafe{}.

\begin{answerToRQ}[\ref{rq:prevalApp}]
    About \percentageUnsafeProjectsRounded{} of projects contain \unsafe{} usages in their first-party code.
    Approximately \percentageUnsafeTransitiveWithDependenciesRounded{} of projects transitively import at least one
    third-party dependency package with \unsafe{} usages.
\end{answerToRQ}

To answer research question~\ref{rq:depsDepth} about how deep in the dependency tree packages using \unsafe{} are
usually located, the import depth of all packages used by a project is determined.
Import depth denotes the minimum depth of a package in the dependency tree of a particular project, which is the
shortest path from the project root module to the package.
Thus, all packages included in the root module have an import depth of \checkNum{zero}.
Packages which are imported by those have a depth of \checkNum{one}, and so on.
The depth is calculated using breadth-first search on the dependency tree, which saves time when packages are imported
many times because the analysis is focused on the mimimum depth.
Figure~\ref{fig:unsafe-import-depth} presents a heatmap plot of the number of packages containing \unsafe{} usages by
their import depth.
The y-axis denotes the depth, the color intensity shows the number of \unsafe{} packages at a given depth, and the
x-axis represents the \projsAnalyzed{} analyzed projects.
On the left hand side, next to the heat map, the horizontal bar chart visualizes the total number of packages at each
import depth summed up over all projects.
Packages that do not contain any \unsafe{} usages are not included in the sum as they are irrelevant for
answering~\ref{rq:depsDepth}.
Thus, the plot presents data about \unsafePackages{} packages.

\begin{figure*}[!t]
    \centering
    \includegraphics[width=\textwidth]{gfx/figures/unsafe-import-depth.png}
    \caption{Import Depth of Unsafe Packages. Answers~\ref{rq:depsDepth}: unsafe packages are around \averageUnsafeImportDepth{} hops away (sd=\stdUnsafeImportDepth), thus manageable to find manually.}
    \label{fig:unsafe-import-depth}
\end{figure*}


Figure~\ref{fig:unsafe-import-depth} shows that most packages containing \unsafe{} usages are imported fairly early,
however not directly at the first level.
The average depth is \averageUnsafeImportDepth{} with a standard deviation of \stdUnsafeImportDepth{}.
The general import depth of all packages, whether they contain \unsafe{} or not, is only slightly lower at
\averageGeneralImportDepth{}.
While the numbers are fairly low, the count of packages at each level of import depth increases exponentially.
Thus, while being possible it is hard for developers to manually audit dependency packages for \unsafe{} usages.
The novel \toolGeiger{} tool helps by quickly identifying the packages containing \unsafe{}, therefore it is possible to
conduct a focused review of the \unsafe{} code without wasting time on identifying the relevant packages for it.
Only the first level of dependencies contains the packages that the project developers added themselves, thus they are
obvious to the developers.
In the data set, \levelOneImportedUnsafePackagesCount{} (\levelOneImportedUnsafePackagesShare{}) of the
\unsafePackages{} packages containing \unsafe{} are imported at level one of the dependency tree.
\levelZeroImportedUnsafePackagesCount{} (\levelZeroImportedUnsafePackagesShare{}) are not imported (level zero).
This further shows that the majority of \unsafe{} code is introduced further down in the dependency tree.

\begin{answerToRQ}[\ref{rq:depsDepth}]
    Most imported packages with at least one \unsafe{} usage are located around a depth of
    \averageUnsafeImportDepthRounded{} in the dependency tree.
\end{answerToRQ}

Research question~\ref{rq:distTypes} is about which \unsafe{} tokens are used the most.
As described in Section~\ref{sec:go-geiger:design}, \toolGeiger{} identifies usages of the four members of the \unsafe{}
package, the \textit{reflect.SliceHeader} and \textit{reflect.StringHeader} types, and \textit{uintptr}.
Figure~\ref{fig:unsafe-tokens-distribution} shows the distribution of these \unsafe{} types in the data set of Go
projects.

\begin{figure}[!t]
    \vspace{-12pt}
    \centering
    \includegraphics[width=0.43\textwidth]{gfx/figures/distribution-unsafe-types-pdf.pdf}
    \caption{Distribution of different types of \unsafe{} tokens}
    \label{fig:unsafe-tokens-distribution}
\end{figure}

The data shows that \textit{uintptr} is the most common \unsafe{} token, with \checkNum{99,741} findings.
Next, \textit{unsafe.Pointer} has a similarly high prevalence of \checkNum{94,431} samples.
These two lead the usage counts by far, with the next being \textit{unsafe.Sizeof} at only \checkNum{3,667} usages and
all other token types found less than \checkNum{1,000} times.
With a mere \checkNum{8} usages found, \textit{unsafe.Alignof} is the most rare.

\todo{add takeaway message?}

\begin{answerToRQ}[\ref{rq:distTypes}]
    In the wild, uintptr and unsafe.Pointer are orders of magnitude more common than other unsafe usages.
\end{answerToRQ}


%% ---------------------------------------------------------------------------------------------------------------------

\subsection{Influence of Age and Popularity (\ref{rq:popularity})}\label{subsec:go-geiger:evaluation:popularity}

This subsection answers research question~\ref{rq:popularity} about whether there is a correlation between \unsafe{}
usage and the common projects metrics age and popularity.
Popularity is measured by the number of stars and number of forks that a project has on \github{}.
Figure~\ref{fig:correlation-popularity} presents a scatter plot showing the number of \unsafe{} usages on the x-axis and
the project metrics stars, forks, and age on the y-axis.
Usages that are part of the Go standard library are not counted in this graph.
Each data point represents one project.
Blue circles indicate a project's number of stars, orange diamonds show the number of forks, and green crosses denote
the age.

\begin{figure}[htp!]
    %\vspace{2mm}
    \centering
    \includegraphics[width=\textwidth]{assets/plots/chapter4/correlation-popularity.pdf}
    \caption{Correlation between \unsafe{} usage and project metrics age and popularity}
    \label{fig:correlation-popularity}
    %\vspace{-10pt}
\end{figure}


The plot overall shows a uniform distribution between \unsafe{} usages and the different project metrics, except for
a gap between a \checkNum{few hundred} and \checkNum{1,000} usages of \unsafe{}.
There are both many projects with fewer and with more usages, but none with about \checkNum{750}.
This could be an indicator that once projects have included more than only a couple of \unsafe{} usages, they tend to
have more of them more easily.
There is no obvious correlation between neither \unsafe{} usage and project age, nor number of stars or forks.

\begin{answerToRQ}[\ref{rq:popularity}]
    There is no significant correlation between the number of \unsafe{} usages and a project's age, number of stars, or
    number of forks.
\end{answerToRQ}


%% ---------------------------------------------------------------------------------------------------------------------

\subsection{Change of Usage over Time (\ref{rq:changeTime})}\label{subsec:go-geiger:evaluation:over-time}

The data set collected in this study contains one version of each analyzed project.
It is not directly possible to measure the change of \unsafe{} usage in projects over time with the data available.
However, there are a number of modules that are included in several versions by different projects, which means that
these modules allow such an analysis of changes in \unsafe{} usage.
To answer research question~\ref{rq:changeTime}, this subsection discusses the differences between different version of
an examplary module, \textit{golang.org/x/sys}.

The module is present in the data set with a total of \sysModuleVersions{} versions.
A subset of the versions with a fair distribution of release dates in \checkNum{2019} and \checkNum{2020}, as well as a
rather old version from \checkNum{2017} is shown in Table~\ref{tbl:unsafe-usage-over-time}.
The table shows the version identifiers alongside their release dates and number of \unsafe{} usages in the particular
version.
The module versions are ordered by their release date.

\begin{table}[htp!]
    \centering
    \caption{Change of unsafe usage over time in the golang.org/x/sys module}
    \label{tbl:unsafe-usage-over-time}
    \begin{tabular}{l|l|r}
    \textbf{Version}                   & \textbf{Release date} & \textbf{Unsafe usage count} \\
    \hline
    v0.0.0-20171012164349-43eea11bc926 & 12.10.2017            & 315                         \\
    v0.0.0-20190502145724-3ef323f4f1fd & 02.05.2019            & 387                         \\
    v0.0.0-20190726091711-fc99dfbffb4e & 26.07.2019            & 392                         \\
    v0.0.0-20191001151750-bb3f8db39f24 & 01.10.2019            & 403                         \\
    v0.0.0-20191128015809-6d18c012aee9 & 28.11.2019            & 403                         \\
    v0.0.0-20200107162124-548cf772de50 & 07.01.2020            & 428                         \\
    v0.0.0-20200302150141-5c8b2ff67527 & 02.03.2020            & 428                         \\
    v0.0.0-20200413165638-669c56c373c4 & 13.04.2020            & 434                         \\
    v0.0.0-20200501145240-bc7a7d42d5c3 & 01.05.2020            & 440                         \\
    \end{tabular}
\end{table}

It is evident that there is a monotonous increase in \unsafe{} usage, with an increase from \sysModuleLeastUnsafe{} to
\sysModuleMostUnsafe{} (\sysModuleUnsafeIncrease) over the \checkNum{2.5 years} period from \checkNum{2017} to
\checkNum{2020}.
A manual analysis of the changes in the module source code shows that the increased usage of \unsafe{} in this case is
caused by additional system call \acrshort{API}s that are supported by the module.
Dispatching to the underlying system call code requires the use of \textit{unsafe.Pointer}.
Therefore, in this case more features provided by a dependency cause more \unsafe{} code to be imported into a project.

\todo{Replace with plot and add more modules}

\begin{answerToRQ}[\ref{rq:changeTime}]
    Changes in \unsafe{} usage in particular modules are motivated, for example, by new \acrshort{API} requirements and
    can be significant, with e.g. about a \sysModuleUnsafeIncreaseRounded{} increase over \checkNum{2.5} years found in
    the \textit{sys} module.
\end{answerToRQ}


%% ---------------------------------------------------------------------------------------------------------------------

\subsection{Comparison with Existing Tools (\ref{rq:linterComparison})}\label{subsec:go-geiger:evaluation:linters-comparison}

To evaluate the benefit \toolGeiger{} provides in comparison to existing static analysis tools for Go, its findings are
put in context with the results of \toolVet{} and \toolGosec{} in this section.
The goal of this comparison is to see whether any of those tools can achieve the same as \toolGeiger{} does.
As described in Section~\ref{sec:background:static-code-analysis}, \toolVet{} is a linter that is included as part of
the standard Go command line tool chain.
It runs a number of analysis passes to identify general problems with the source code.
There is the \textit{unsafeptr} pass, which is designed to find potential misuses of the \textit{unsafe.Pointer} type.
It is however not designed for a general identification of \unsafe{} usages.
On the other hand, \toolGosec{} is a static analysis tool with a design focused around security problems.
It is built from several rules to identify issues, one of which (\textit{\checkNum{G103}}) is simply triggered by the
presence of \unsafe{} package members and generates a warning that those usages should be audited.
Given this design, it is closer to \toolGeiger{} in the sense that it only identifies the presence of \unsafe{} without
using any logic to determine potential misuses.

To conduct the comparison, \toolVet{} and \toolGosec{} are run on the same \packagesAnalyzed{} packages that were
analyzed with \toolGeiger{}.
The results are part of the data set as well.
Then, the findings of the tools are matched using package, file name, and line number information.
Table~\ref{tbl:go-geiger-evaluation-linters} shows the results of this analysis.
The result columns are divided to show the different results for \toolVet{} and \toolGosec{}.
True positives (TP) are lines of code that were both flagged by \toolGeiger{} and the respective linter tool.
False negatives (FN) denote lines that were only flagged by \toolGeiger{}, but not by the linter.
Finally, false positives (FP) indicate that the linter flagged a line that was not identified by \toolGeiger{}.
There is no column for true negatives because the number of these is irrelevant for this analysis.
The table contains two rows that show the number of lines of code, both when counting any message that \toolVet{} or
\toolGosec{} produced, and when only messages related to their \unsafe{} analyses are taken into account.
The latter has more impact because comparing only those messages to the output of \toolGeiger{}, which is solely
designed around \unsafe{} usages, achieves a fairer evaluation.

\begin{table}[htp!]
    \centering
    \caption{Comparison of number of lines with \unsafe{} found by \toolGeiger{} and existing linters \toolVet{} and \toolGosec{}}
    \label{tbl:go-geiger-evaluation-linters}
    \begin{tabular}{l|rr|rr|rr}
        \textbf{Scenario} & \multicolumn{2}{c|}{\textbf{both}} & \multicolumn{2}{c|}{\textbf{only \toolGeiger{}}} & \multicolumn{2}{c}{\textbf{only existing linter}} \\
        {}                & go vet          & gosec            & go vet          & gosec                          & go vet          & gosec                           \\
        \hline
        Any message       & 219             & 36,279           & 76,738          & 40,678                         &  31,224         & 114,306                         \\
        Related message   & 213             & 26,267           & 76,744          & 18,019                         &       0         & 0                               \\
    \end{tabular}
\end{table}

The results show that there is only a very small number of true positives for \toolVet{}, but many false negatives.
This means that for most of the \unsafe{} usages \toolVet{} does not generate a warning.
When comparing any message generated by \toolVet{}, there are many false positives, however those do not exist anymore
when the analysis is restricted to \toolVet{} messages related to \unsafe{}.
This means that while \toolVet{} provides a lot of warnings that are related to other problems, it does not offer any
benefit over \toolGeiger{} for the specific task of identifying \unsafe{} usages.
For \toolGosec{}, there are a lot more true positives, but still a lot of false negatives.
About \checkNum{half} of the \toolGeiger{} results are also found by \toolGosec{}.
This is much better than \toolVet{}, but it is still not accurate.
One reason for this is that \toolGeiger{} identifies not only \textit{unsafe} packages uses, but also \textit{uintptr}
which is common as described in Section~\ref{subsec:go-geiger:evaluation:unsafe-usage}.
Similar to \toolVet{}, there are many false positives when any \toolGosec{} message is counted, but there are none when
only \unsafe{}-related messages are analyzed.
Therefore, \toolGosec{} can not detect any usages that \toolGeiger{} would miss.
It is worth noting that the numbers in Table~\ref{tbl:go-geiger-evaluation-linters} refer to lines of code rather than
\unsafe{} findings, but one line of code can contain several \unsafe{} usages.
Therefore the numbers do not add up to the same count of total findings discussed in
Section~\ref{subsec:go-geiger:evaluation:unsafe-usage}.
\todo{structure comparisons better, revisit feedback from AnKa.}

\begin{answerToRQ}[\ref{rq:linterComparison}]
    The existing tools \toolVet{} and \toolGosec{} do not provide any benefit over \toolGeiger{} for the specific task
    it is designed for.
    Instead, \toolGeiger{} finds all and more of their \unsafe{}-related results.
\end{answerToRQ}


%% ---------------------------------------------------------------------------------------------------------------------

\section{Qualitative Evaluation}\label{sec:go-geiger:qualitative-evaluation}

This section presents an in-depth, qualitative study of the purpose of \unsafe{} usages in
\projsForLabeledCodeSnippets{} selected open-source Go projects.


%% ---------------------------------------------------------------------------------------------------------------------

\subsection{Labeled Data Set of Unsafe Usages}\label{subsec:go-geiger:qualitative-evaluation:labeled-dataset}

To study the purpose of \unsafe{} in applications, a manually labeled data set of \numberLabeledCodeSnippets{} code
samples is presented as a contribution of this thesis.
It is available online in the same data repository\footnote{\url{https://github.com/stg-tud/unsafe_go_study_results}} as
the study results presented in the previous section.
These samples are divided into \numberLabeledCodeSnippetsStd{} standard library samples (\textit{std}) and
\numberLabeledCodeSnippetsApp{} non-standard-library application samples (\textit{app}).
\todo{why these numbers?}
The goal of this data set is to understand what operations are done in practice using the \unsafe{} \acrshort{API},
and for what purpose.
Thus, each sample is labeled in two dimensions by the operation type and its higher-level objective.
The \numberLabeledCodeSnippets{} code samples are drawn from the projects listed in Table~\ref{tbl:dataset-projects}.
They are a subset of the projects used for the empirical study and marked by an asterisk (*) in the appendix
(Table~\ref{tbl:projects}).

\begin{table}[htp!]
    \centering
    \caption{Projects selected for labeled data set}
    \label{tbl:dataset-projects}
    \begin{tabular}{llrrl}
        {} & \textbf{Name} &  \textbf{Stars} &  \textbf{Forks} &    \textbf{Revision} \\ \hline
        \rowcolor{verylightgray}
        1  &         kubernetes/kubernetes &  66,512 &  23,806 &  \texttt{fb9e1946b0} \\
        2  &                 elastic/beats &   8,852 &   3,207 &  \texttt{df6f2169c5} \\
        \rowcolor{verylightgray}
        3  &             gorgonia/gorgonia &   3,373 &    301 &  \texttt{5fb5944d4a} \\
        4  &              weaveworks/scope &   4,354 &    554 &  \texttt{bf90d56f0c} \\
        \rowcolor{verylightgray}
        5  &  mattermost/mattermost-server &  18,277 &   4,157 &  \texttt{e83cc7357c} \\
        6  &               rancher/rancher &  14,344 &   1,758 &  \texttt{56a464049e} \\
        \rowcolor{verylightgray}
        7  &                 cilium/cilium &   5,501 &    626 &  \texttt{9b0ae85b5f} \\
        8  &                     rook/rook &   7,208 &   1,472 &  \texttt{ff90fa7098} \\
        \rowcolor{verylightgray}
        9  &             containers/libpod &   4,549 &    539 &  \texttt{e8818ced80} \\
        10 &                       xo/usql &   5,871 &    195 &  \texttt{bdff722f7b} \\
    \end{tabular}
\end{table}

The projects were selected based on their high number of \unsafe{} usages, and it was taken care of a reasonably large
diversity in their domains of application.
They contain applications based around containers and operations, chat, infrastructure as a service, data storage,
visualization, and machine learning.
When selecting samples to look at, the standard library samples were deliberately separated from application usages.
This decision is based on the results presented in Section~\ref{subsec:go-geiger:evaluation:unsafe-usage}, which show
that all projects must use the standard library, as well as the hypothesis that the standard library uses different
\unsafe{} patterns.

Standard library is defined by the packages that live in the Go \textit{std} and the \textit{golang.org/x/sys} module.
The \textit{sys} module is included because it contains a lot of system call infrastructure and is the replacement for
the deprecated \textit{syscall} package\footnote{\url{https://golang.org/pkg/syscall}}, which is part of the standard
library.
Thus, it is also maintained by the core Go developer team.
Application code samples are taken from packages that are not part of the standard library.
The \numberLabeledCodeSnippets{} code examples were sampled randomly from all packages present in the dependency trees
of the projects shown in Table~\ref{tbl:dataset-projects}.
This increases the variety of code authors contributing to the code examples.
However, they are taken without duplicating lines that are present in different versions of the same module.
If a module is included in multiple versions, than a line of code within it can not be drawn twice from different
versions.
Furthermore, this labeled data set contains only usages of \textit{unsafe.Pointer}.
The goal of this is to allow a better comparison between usage purposes, without distortions from different \unsafe{}
token types such as \textit{Alignof} or \textit{Offsetof}.

\begin{figure}[htp!]
    \includegraphics[width=\textwidth]{assets/figures/chapter4/label-examples.pdf}
    \caption{Labeled data set usage classes}
    \label{fig:label-examples}
\end{figure}


The classes are outlined briefly in the following.
For the \unsafe{} operation type dimension, Figure~\ref{fig:label-examples} shows code examples and descriptions for
each of the classes.
The most common ones are conversions between types.
The \textit{cast-basic}, \textit{cast-bytes}, \textit{cast-struct}, \textit{cast-header}, and \textit{cast-pointer}
classes represent conversions between arbitrary types, where one of them is a basic type such as \textit{int}, a
\textit{[]byte} array, a slice or string header structure, an \textit{unsafe.Pointer} directly, or a structure,
respectively.
\textit{Pointer-arithmetic} is a class containing any form of arithmetic address manipulation such as advancing an
array.
When \unsafe{} is only required to pass it along to another function that expects a parameter of type
\textit{unsafe.Pointer}, the \textit{delegate} label is applied.
The \textit{memory-access} class is used where unsafe.Pointer values are dereferenced, used to manipulate corresponding
memory or for comparison with another address.
\textit{Syscall} represents calls using the Go \textit{syscall} package or \textit{golang.org/x/sys} module.
The \textit{definition} class denotes usages where a field or method of type \textit{unsafe.Pointer} is declared for
later usage.
Thus, for this class the need for \unsafe{} is at a different location in the code.
Finally, \textit{unused} contains instances of \unsafe{} that are not actually used in the analyzed code, such as dead
code or unused parameters.

In the purpose dimension, the \textit{efficiency} class represents cases where unsafe is used only to improve time or
space efficiency of the code.
Usages in this class could be rewritten to not use \unsafe{}.
The \textit{serialization} class includes (un)marshalling and (de)serialization operations like in-place casts between
complex types and bytes.
\textit{Generics} is applied where \unsafe{} is used to build functionality that would be implemented using generics if
they were available in current versions of Go.
Up until release \checkNum{1.15}, there is no support for generics in Go without additional libraries.
The \textit{avoid garbage collection} class contains usages to ask the Go compiler to lock a value while it is used, for
example, when calling a function written in assembly.
\textit{Atomic operations} is a class of usages of the atomic \acrshort{API}, which requires \unsafe{} for some of its
functions.
The \textit{foreign function interface} (\acrshort{FFI}) class includes all cases of interoperability with C code (CGo),
as well as calls to functions that receive their parameters as unsafe pointers.
\textit{Hide from escape analysis} contains instances where \unsafe{} is used to deliberately exclude a value from being
seen by the \acrshort{EA} algorithm.
The \textit{memory layout control} class represents code used for low-level memory management.
\textit{Types} samples are used by the standard library for low-level implementation of the Go type system.
\textit{Reflect} includes instances of type reflection, as well as re-implementations of types contained in the regular
Go \textit{reflect} package, for example, to use \textit{unsafe.Pointer} instead of \textit{uintptr} for slice headers.
Lastly, \textit{unused} is a class for unused occurrences again.


%% ---------------------------------------------------------------------------------------------------------------------

\subsection{Purpose of Unsafe in Practice (\ref{rq:purpose})}\label{subsec:go-geiger:qualitative-evaluation:purpose}

Table~\ref{tbl:dataset-classes} presents the number of samples for each label.
The columns denote the dimension of purpose, while the rows show the operation type dimension.
Columns are divided into separate counts for the \textit{app} and \textit{std} groups of samples.

\input{assets/tables/chapter4/dataset-classes.tex}

The data shows that efficiency is by far the most prevalent reason to use \unsafe{} in real-world Go application code.
While these usages make up about \checkNum{58\%} of the application code samples, they account for only about
\checkNum{5\%} of the standard library usages.
Within the \textit{efficiency} class, casting operations cover most of the usages with \checkNum{97\%} (\textit{app})
and \checkNum{80\%} (\textit{std}) of the samples.
Next, the second most important motivation for \unsafe{} code in the application class is performing serialization or
deserialization operations, including marshalling of structured data to interchangeable formats.
This accounts for \checkNum{28\%} of the \textit{app} usages.
The standard library shows a different most common usage, which is avoiding garbage collection with \checkNum{25\%}.
This purpose is only found in \checkNum{2\%} of the \textit{app} samples.
Next, the \textit{type} (\checkNum{18\%}), \textit{\acrshort{FFI}} (\checkNum{15\%}), and \textit{memory layout}
(\checkNum{13\%}) classes are common in the \textit{std} samples.
A common distribution is the hiding from escape analysis class, which is rare both in the \textit{app}
(\checkNum{0.1\%}) and \textit{std} (\checkNum{2\%}) groups.
The same is true for the \textit{reflection} class (\checkNum{1\%} in both sets).
There are also only few samples (\checkNum{2\%}) that are used to implement generics functionality, which is not
available in current versions of Go.
Some of the findings in the \textit{serialization} class could however be achieved with generics as well, so the classes
overlap slightly.
\todo{add takeaway message}

\begin{answerToRQ}[\ref{rq:purpose}]
    \checkNum{More than half} of the \unsafe{} usages in projects and 3rd party libraries are to improve efficiency via
    \unsafe{} casts.
    In the Go standard library, \checkNum{every third} use of \unsafe{} is to avoid garbage collection.
\end{answerToRQ}


%% ---------------------------------------------------------------------------------------------------------------------

\section{Summary}\label{sec:go-geiger:summary}

\todo{write summary}
