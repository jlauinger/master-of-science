%% -----------------------------------------------------------------------------

\chapter{Related work}\label{ch:related-work}

Put the work in context with current literature.


%% -----------------------------------------------------------------------------

\section{Dependency analysis in Rust with Cargo Geiger}\label{sec:cargo-geiger}

Cargo Geiger is the template for the Go dependency check tool.
It does the same thing for the Rust programming language.


%% -----------------------------------------------------------------------------

\section{OWASP Dependency Checker}\label{sec:owasp-dependency-checker}

Explain OWASP security project.

Show OWASP Dependency Checker program, show experimental Go extension for it.
This tool does no unsafe code analysis though, it checks for known vulnerabilities of the dependencies in the
OWASP database.


%% -----------------------------------------------------------------------------

\section{Work in progress: paper summaries}\label{sec:paper-summaries}


%% -----------------------------------------------------------------------------

\subsection{Understanding Memory and Thread Safety Practices and Issues in Real-World Rust Programs}
\label{subsec:understanding-memory-and-thread-safety-practices-and-issues-in-real-world-rust-programs}

Qin et al.~\cite{qin2020} contribute the first to their knowledge empirical study of bugs related to unsafe code blocks
in Rust.
They mention other empirical studies~\cite{difranco2017, lu2013, chou2001, leesatapornwongsa2016, jin2012, gunawi2014, gu2015}
none of which seem to be related to Rust or Go.
The authors analyze 5 projects, 5 libraries and 2 databases.
They randomly sample 850 unsafe usages out of those.
They analyze the usages, categorize them into classes, analyze the impact of the bugs and develop two new static code
checking tools.
In essence, this is exactly my thesis but done for Rust, and it's probably an excellent example of how to structure a
paper on this.

A main difference to this thesis is that the authors not only look into the current revision code, but explicitly look
through the Git history, filtering for commits that remove unsafe usages.
They further go through reported bugs on the software under analysis and look into the code how those bugs were fixed.
Bug data is retrieved from CVE and RustSec.
This is something I should probably do as well!

Within the 850 unsafe usages, the authors analyze 70 memory-safety issues and 100 concurrency bugs.
It sounds like they found all of those, but this high number is because they look at real-world bugs that were previously
reported.
Using their tools however, they also find about 5 to 10 new bugs that they disclosed to the developers.

Similar to my thesis, they explain the purpose of safe code and reasons why unsafe code might be needed.
They obviously focus on Rust, but some of the points will apply to Go as well.

The reasons to use unsafe code are clustered into these groups: Reuse existing code, convert C-array to Rust slice,
improve performance.
The authors find that often the use of unsafe code has good or unavoidable reasons.
Looking at the commit diffs, they find that unsafe code gets removed to fix memory safety, improve code structure,
improve thread safety, fix bugs, or because it was unnecessary in the first place.

The authors also look into the Rust standard library, which has similar uses of unsafe as the Go standard library.
Unsafe code often requires some preconditions on lifetime or input data to hold, and this can be achieved by encapsulating
it into an interior unsafe function that checks these conditions and might skip the unsafe code if they do not hold.

The areas of problems identified are buffer overflows, null-pointer dereferencing, reading uninitialized memory, invalid
free, use after free, and double free.
Fixing strategies were conditionally skipping the unsafe code, adjusting variable lifetime, or changing unsafe operands.

Rust also has thread safety problems, and problems the authors identified can be grouped into blocking and non-blocking.
Blocking bugs are caused due to incorrect scoping of auto-releasing mutexes.
Channels might also block.
Non-blocking bugs include data races and stem from incorrect scoping of shared data or confused ownership.

If a function's safety depends on how it is used, it might be better put into an encapsulation.

The authors suggest development of IDE extension visualizing scopes and lifetimes, development of Rust-tailored static
analysis tools (which they already contributed two), and dynamic analysis tools.
Due to the study language developers can also learn from design issues concerning the Rust language itself, because it
shows how developers adapt to new language concepts over time.


%% -----------------------------------------------------------------------------

\subsection{Is Rust Used Safely by Software Developers?}
\label{subsec:is-rust-used-safely-by-software-developers?}

Evens et al.~\cite{evans2020} present an empirical study of unsafe usages in Rust crates.
They only include statistical facts such as total number of unsage usages.
This is different from Qin et al.~\cite{qin2020} in that they did not do an in-depth analysis of potential bugs.
Still they got accepted at ICSE!

They found that most unsafe usage is to call other unsafe Rust code, while calling C code is less of a concern.
About a third of libraries contain unsafe code and more than half of them transitively do through dependencies.
The usage count in crates did not change much over the course of 10 months work.
More popular libraries are more likely to contain more unsafe code because they encapsulate more popular C libraries.

The authors conducted a N=20 survey on Reddit, revealing that 10\% of developers used unsafe to make the code compile.
Other popular reasons included performance optimizations, advanced data structures, and unsafe code offering a slicker
and "more elegant" interface.
Most developers employed more testing, static and dynamic analysis, and fuzzing when using unsafe code, but many also
said they would "look carefully on the code".
I'd suspect this is not an effective approach.

The authors developed an augmented Rust CFG and algorithm to detect potentially unsafe functions in all Rust crates that
would compile.


%% -----------------------------------------------------------------------------

\subsection{Escape from Escape Analysis of Golang}
\label{subsec:escape-from-escape-analysis-of-golang}

Wang et al.~\cite{wang2020} propose an approach to make heap memory usage in Go more effective by improving the escape
analysis algorithm.
The current algorithm is very conservative.
In particular, the authors try improve a specific type of escape analysis: passing a pointer into a function call will
make that object escape.
They contribute an optimizer that always replaces such calls by an intermediate cast of the pointer into a uintptr
variable and then back to a pointer, both through the use of unsafe.Pointer.
This breaks the escape analysis chain and will make Go escape only the new pointer to the heap, not the entire previous
data structure.
After the identification of such snippets in the code comes a verification stage that will check whether the function
call is synchronous.
In that case, the underlying variable cannot be freed before the end of the function, and the optimization is correct.
If the call is asynchronous, the optimizer checks if the variable is used in any other Goroutine, in which case the
optimization will not be done.
Otherwise it is deemed valid.

The authors mention escape analysis in other language~\cite{hill2002, hannan1998, choi1999}.

This is an interesting related work because the tool to break escape analysis, the cast to uintptr and back, is exactly
the problem in the common unsafe slice cast that I describe in the other sections.

The authors evaluate performance and correctness on 10 open-source and 10 industrial projects.
Their data set includes kubernetes.

The paper includes advertisement for a company and curiously uses Go 1.9 although the paper is to be published at ICSE
2020.


%% -----------------------------------------------------------------------------

\subsection{Why Can’t Johnny Fix Vulnerabilities: A Usability Evaluation of Static Analysis Tools for Security}
\label{subsec:why-can’t-johnny-fix-vulnerabilities:-a-usability-evaluation-of-static-analysis-tools-for-security}

Smith et al.~\cite{smith2020} conduct a developer survey on usability problems with static code analysis tools.
They contribute insights into why developers can often not get the most beneficial outcome from such tools.
The mainly analyze Java analysis tools such as Find Bugs, CheckStyle and PMD\@.

The problem areas found are code navigation issues, missing or buried information, interface scalability for large
projects (such as a huge control flow graph), inaccuracy of analysis, code disconnect (meaning the proposed fix to a
problem did not resemble the previous problem anymore), and workflow continuity (meaning the developers had to
constantly switch between the tool report and the code editor).

The authors propose as improvements a clear communication of what and how to fix, alerts that are located within the
editable source code, contextualized and meaningful notifications, and a good integration into existing development
workflows such as CI.

Minor relevance, probably cite in the chapter about the linting tools.


%% -----------------------------------------------------------------------------

\subsection{Source Code Vulnerabilities in IoT Software Systems}
\label{subsec:source-code-vulnerabilities-in-iot-software-systems}

Alnaeli et al.~\cite{alnaeli2017} contribute a study of unsafe C code patterns, that is well-known unsafe
functions such as strcpy, in IoT software.
They count how many unsafe and how many safe functions they find, and compare them with each other to discover trends.

Similar to my work, they search for unsafe code patterns in open-source code.
Similar to my work, they also look at changes over time.
In contrast to my work, they analyze C code instead of Go code.

The authors find that the systems under review have neither introduced more nor removed some of the unsafe functions
over time.
They conclude that developers might be unaware of the presence of the functions, their implications, or both.
They suggest better developer education on security-related consequences of these functions.

Minor relevance, cite in survey chapter.


%% -----------------------------------------------------------------------------

\subsection{Statically Detecting Likely Buffer Overflow Vulnerabilities}
\label{subsec:statically-detecting-likely-buffer-overflow-vulnerabilities}

In their (quite old) 2001 paper, Larochelle and David Evans~\cite{larochelle2001} write about static code analysis to
detect likely buffer overflow vulnerabilities in C code.
They cite a paper assuming that buffer overflows would still be relevant in 20 years, that would be 2019.
That would be true but I'll have to see if it makes sense to also cite this and add a comment.

Similar to my work, this is using static analysis to find problems in code.
But contrary to my work, it's for C code, not Go.
They also talk about simple overflow exploitation techniques, in the context of C\@.
I can and probably should do the same, with a focus on Go.

The authors exploit semantic comments to enable local checking of interprocedural properties, they focus on lightweight
analysis that does not add much overhead, and they use heuristics.
They conduct their study using the LClint annotation tool also developed by Evans.
They note that annotating the standard library would bring a lot of security even without annotating actual programs,
because most vulnerabilities come from using insecure or improperly used functions.
This I can use too to argue for better auditing of standard libraries and developer education as well.

The annotations available include data constraints, such as x > y, and control flow constraints such as branch-specific
annotations.


%% -----------------------------------------------------------------------------

\subsection{Vulnerable Open Source Dependencies: Counting Those That Matter}
\label{subsec:vulnerable-open-source-dependencies:-counting-those-that-matter}

Pashchenko et al.~\cite{pashchenko2018} in a quite recent study propose a way of counting relevant dependencies in a
project and identifying whether the dependencies are vulnerable.
The study is done for Java, and specifically for Maven dependency management.
The authors note that failure of identifying relevant dependencies might lead to bad allocation of development resources.

A central point is dependency scoping (production versus testing) which many studies did not incorporate in the past.
If a vulnerability is found in a dependency only relevant for testing, it might not be worth to put resources onto its
mitigation.
This is extremely relevant for me!
As of now, I also do not distinguish between testing and production library.
This is probably not even possible with the Go dependency management system.
A key insight for distiguishing this related work from my work would be to highlight the relative instability of the
incredibly new Golang dependency system compared to the very seasoned Maven system.
The authors find that around 20 \% of dependencies are for testing only (not deployed).

Furthermore, it is very important to transitively look at dependencies of dependencies because they can just as well
introduce problems into the main project.
This is something my study already perfectly does.

The authors contribute a method of determining whether a dependency is actively maintained or halted.
It is done by looking at the release cycle with an exponential smoothing model.
There are only a few vulnerabilities in halted libraries that they found, but those are especially important because
mitigation cannot be done by a simple upgrade.

The authors introduce the important concept of reliability for bug fixing.
This is, the developers are directly reliable for fixing their main package and own dependencies, and they need to
make sure their direct dependencies are up to date but cannot fix them.
The indirect, or transitive, depencies are out of responsiblity for the developers since they cannot even be upgraded
manually.
A major finding is that grouping the dependencies of a project into these reliability groups gives a better picture into
how much the developers can actually do, and the authors find that for more than 80 \% of the problems the developers
can mitigate themselves through an upgrade or fixing their own code.

Identification of vulnerabilities is done by matching code patterns to vulnerability databases both in a manual and an
automated fashion.
The developers contribute a tool that can provide annotations to the code as to which vulnerablity it might belong.

A very relevant insight for me: the authors first did an incorrect dependency popularity measurement.
They counted the times the dependency gets imported, possibly distorting the popularity if a project is decomposed into
many small own dependencies~\cite{sajnani2014}.
A better way is to count the projects that include the dependency.
I already use the second approach in Table~\ref{tbl:pull-requests} for the supplied pull requests for fixes.
However, I need to make sure in the plot generation that I use the same metrics.

The authors compare to other ecosystems such as Pip and NPM, but not Go.
For NPM, they cite a relevant study on Javascript dependency vulnerabilites~\cite{lauinger2017}, which showed that
transitive project dependencies are more vulnerable than direct dependencies.
They reason that this might be the case because developers are less aware of the existence of those dependencies, and
have less control about them.
The authors highlight that this is a finding specific to Javascript because NPM allows several versions of the same
library to be used in the same project, while Maven does not.
I need to find out if this is possible with Go modules too, I think it is not.
I think Go behaves like Maven.
However, I need to highlight that in my data survey there are of course different versions of the same library, and this
is even necessary to analyze unsafe usage over time.

Within threats to validity, the authors note that the selection of libraries was potentially biased, that the
vulnerability database may not cover all vulnerabilites, the study only used Maven, and that project IDs were approximated.

This is an extremely valuable related work that I definitely need to cite both in the background on Go dependencies
and the survey as well.


%% -----------------------------------------------------------------------------

\subsection{Do developers update their library dependencies?}
\label{subsec:do-developers-update-their-library-dependencies?}

Kula et al.~\cite{kula2017} conduct a study on Github projects and developer survey on how dependencies are updated.
They find that developers often think library updates are extra work and deprioritize it.
The study is done on Maven and Java and finds that libraries are rarely updated.
Specifically, security updates are included less often because developers were unaware of them.
Upon getting the news, developers promptly updated away from vulnerable dependencies.

Available updates are traditionally announced on the homepage of the library, through change logs and semantic versioning.
Other possibilities include security advisories.

When selecting vulnerabilities for analyzing security advisories, the authors only use denial of service and man in the
middle vulnerabilities.
This means that the survey is both rather shallow and biased towards web projects.

This work is minor to medium relevant.
It's to include in a list of examples to cite, but on its own it is less effective as a related work than the others
I read.


%% -----------------------------------------------------------------------------

\subsection{Can automated pull requests encourage software developers to upgrade out-of-date dependencies?}
\label{subsec:can-automated-pull-requests-encourage-software-developers-to-upgrade-out-of-date-dependencies?}

Mirhosseini et al.~\cite{mirhosseini2017} answer the research question whether automated pull requests such as greenkeeper
or pyup can encourage developers to update dependencies.
They find that badges provide a good incentive to developers to keep them green by updating.
Automated pull requests offer an actionable solution to update, but notification fatigue can work against update
discipline.
That is, maintaining many open source projects can lead to a lot of pull request notifications, especially if there is
one opened for every single dependency.

The authors analyze about 7,400 Github projects, grouped into badges, automated pull requests, and a baseline.
Automated pull requests lead to 1.6 times, badges to 1.4 higher probability to update.
Pull requests also mean faster updates.
On purpose, they use a broad sample of Github projects instead of the most popular ones.
This is something I should talk about too in my work!
In summary, both versions of update helpers had a positive impact, with pull requests being in the lead.

A surprising finding was that only about 30 \% of automated PRs was merged, compared to about 80 \% of general PRs.
Some merged update PRs were also downgraded again after some days, hinting incompatibilities introduced.
Using CI correlated with marginally higher merge rate.

They also conduct a developer survey.
Developers had mixed feelings about badges versus automated pull requests.
They suggested to use batched updates, resulting in fewer pull requests and less fatigue.

The authors identified through survey different updating strategies: quick, scheduled, reactive.
A developer said, reasoned through possible new errors in libraries and minor benefits, that in the reality of commercial
software engineering, staying on bleeding edge library versions usually costs more than it benefits, and the better
approach is to update infrequently.
My work should act on this and argue that this is true for feature releases but very bad for security releases.
Maybe one could do something with semver here.

The survey also showed that a single bad update of one library can very strongly cause developers to be reluctant from
updating in the future.
From this, we can argue for high responsible in library releasing to not destroy the necessary environment for timely
ships of security fixes.

The authors further cite another work stating that vulnerabilities are often contained in dependencies~\cite{xia2014}.
CI is not always a guarantee that an update does not break the software.
Badges can counteract fatigue, and more importantly they rely on a different mechanism: social pressure.

This has very relevance for me because of the survey on update preferences.
It is interesting to cite this within the argument that libraries must be updated in order to remove already fixed
vulnerabilities from dependent projects.


%% -----------------------------------------------------------------------------

\subsection{Understanding the Origins of Mobile App Vulnerabilities: A Large-scale Measurement Study of Free and Paid Apps}
\label{subsec:understanding-the-origins-of-mobile-app-vulnerabilities:-a-large-scale-measurement-study-of-free-and-paid-apps}

Watanabe et al.~\cite{watanabe2017} conduct a study on the origins of mobile vulnerabilities.
They inspect both free and paid Android apps.
They find that 70 \% of bugs come from dependencies, 50 \% from third-party dependencies.
More expensive and/or popular apps tended to have more vulnerabilities.

An interesting finding for me: 50 \% of vulnerabilities were found in unreachable code.
As I also already found one manual vulnerability that was unreachable, it might be interesting to argue that there is a
very feasible risk that the studies are distorted due to failure to identify dead code.

The authors claim that a limitation of static analysis is that obfuscated, runtime-dynamic code might introduce
vulnerabilities that the analysis missed.
This might be true, but such code is also to be treated as dangerous in my opinion.

In contrast to my work, this is based on Android and Java at best, not Go.

This is only minor relevant as another study.


%% -----------------------------------------------------------------------------

\subsection{Understanding Real-World Concurrency Bugs in Go}
\label{subsec:understanding-real-world-concurrency-bugs-in-go}

Tu et al.~\cite{tu2019} present a very recent (2019) study on Go bugs. This is almost definitely an extremely relevant
related work. READ IT!
