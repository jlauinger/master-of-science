%% -----------------------------------------------------------------------------

\chapter{Background on unsafe code}\label{ch:unsafe-code}


%% -----------------------------------------------------------------------------

\section{Common vulnerabilities in systems languages}\label{sec:common-vulnerabilities}

Systems languages have many pitfalls: buffer overflows, use-after-frees, heap overflows,
...

Huge security implications, tons of bugs especially with cryptographic C code that gets
used in the internet.


%% -----------------------------------------------------------------------------

\section{Safeness guarantees in Go and Rust}\label{sec:safeness-feature}

Compiler features to statically prove upon compilation that a program does not contain any
buffer overflows etc.

Usually more verbose programming style, but enables to automatically check the program
on security vulnerabilities.

How is it syntactically achieved in Go?

How is it achieved in Rust?


%% -----------------------------------------------------------------------------

\section{Unsafe code}\label{sec:unsafe-code}

Sometimes the programmer might need to bypass the safe guards, e.g. for optimizations or
interoperability with native unsafe code, that is C libraries.

Explain language features to achieve unsafe code. How does such code look like?


%% -----------------------------------------------------------------------------

\section{Unsafe pointers in Go}\label{sec:unsafe-pointers}

In the case of Go there are unsafe pointers from the unsafe package. Explain how they
can be used, and which reasons there might be to use some.
