%% -----------------------------------------------------------------------------

\chapter{Survey of unsafe code usages in popular Go projects}\label{ch:survey}

A major contribution of this thesis is an analysis of unsafe code usage in the wild:
the top 500 most popular open source Go projects.


%% -----------------------------------------------------------------------------

\section{Most popular open source Go projects}\label{sec:most-popular-projects}

I downloaded the top 500 most popular (by number of stars) Go projects from Github.
They can be found through Github search.

Using the \texttt{github.com/google/go-github/github}~\cite{NEEDED} and \texttt{github.com/go-git/go-git/v5}~\cite{NEEDED}
Go libraries, I found, saved, downloaded, and checkout out these projects using a Go program.

The projects, including the revision I checked out, and some meta data is shown in Table~\ref{tbl:projects}.


%% -----------------------------------------------------------------------------

\section{Methods of generating data points}\label{sec:survey-acquisition-methods}

Iteration of three data acquisition programs, getting faster and faster: Bash, Python, Go.

Loading packages used by the program by running \texttt{go list -deps -json ./...}, then parsing the JSON output.

CSV is used to write data to disk.
Interop with Go structure types is achieved using the \texttt{github.com/gocarina/gocsv}~\cite{NEEDED} library.

Program uses the following steps:

\begin{enumerate}
    \item Read projects to analyze from CSV file
    \item Get package list for a given project
    \item Run requested analysis type
    \item Parse output and match findings to package and file
    \item Append findings for this project to CSV file
    \item Continue with step 2 for the next project, until done
\end{enumerate}

There are five different analysis types:

\begin{itemize}
    \item Grep
    \item Go Vet
    \item Gosec
    \item Abstract Syntax Tree
    \item My linter (go-safer)
\end{itemize}

The time needed to run these analysis on the 500 projects described in Section~\ref{sec:most-popular-projects} is shown
in Table~\ref{tbl:survey-analysis-wallclocktime}.

\begin{table}[h]
    \centering
    \caption{Wall-clock runtime for different analysis types used for the data survey}
    \label{tbl:survey-analysis-wallclocktime}
    \begin{tabular}{ll}
        \toprule
        Analysis Type & Wall-Clock Time \\
        \midrule
        (Project Download) & 4 hours \\
        Grep & 15 minutes \\
        Go Vet & 3 hours \\
        Gosec & 24 hours \\
        Abstract Syntax Tree & 1 hour \\
        Go-Safer Linter & 3.5 hours \\
        \bottomrule
    \end{tabular}
\end{table}


%% -----------------------------------------------------------------------------

\section{Description and structure of data set}\label{sec:survey-dataset}

In the data set I obtained, there are the following objects: projects, packages, modules,  Grep findings, Go Vet findings,
Gosec findings, Go-safer (linter) findings, AST unsafe findings, AST functions, AST statements, and error conditions.

The data set size in terms of object count is shown in Table~\ref{tbl:survey-dataset-size}

\begin{table}[h]
    \centering
    \caption{Survey data set size in terms of object count}
    \label{tbl:survey-dataset-size}
    \begin{tabular}{lr}
        \toprule
        Object type & Data point count \\
        \midrule
        Projects & 495 \\
        Packages & 40,384 \\
        Modules & 3,202 \\
        Grep findings & 2,850,140 \\
        Go Vet findings & 120,416 \\
        Gosec findings & 124,165 \\
        Go-safer (linter) findings & 321 \\
        AST unsafe findings & 2,590,346 \\
        AST functions & 782,760 \\
        AST statements & 1,205,214 \\
        \bottomrule
    \end{tabular}
\end{table}

A thing worth mentioning is that there a bit less AST unsafe findings compared to the Grep unsafe findings. This is
because the Grep run includes comments containing \texttt{unsafe.Pointer}, while the AST run does not.


%% -----------------------------------------------------------------------------

\section{Methods of analyzing the data}\label{sec:survey-analysis-methods}

How did I process the data set?
Approach to build results etc.


%% -----------------------------------------------------------------------------

\subsection{Jupyter Notebooks}\label{subsec:survey-jupyter}

Describe purpose, deployment and notebook structure.


%% -----------------------------------------------------------------------------

\subsection{Python Flask application for manual classification}\label{subsec:survey-classification}

Describe purpose, development and deployment.


%% -----------------------------------------------------------------------------

\section{Survey Results}\label{sec:survey-results}

These sections go through the results of the data survey.
Unsafe usages include usages of \texttt{unsafe.Pointer}, \texttt{uintptr}, \texttt{reflect.SliceHeader} and more.


%% -----------------------------------------------------------------------------

\subsection{Usage statistics in projects, modules, packages, and registries}\label{subsec:results-stats}

\begin{figure}[ht]
    \begin{center}
    {\scriptsize \includegraphics[width=\textwidth, height=9cm]{assets/plots/unsafe-usages-by-project-n30.tikz}}
    \end{center}
    \caption{unsafe usages by project for the top 30 projects with most usages}
    \label{fig:unsafe-usages-by-project-n30}
\end{figure}

\begin{figure}[ht]
    \begin{center}
    {\scriptsize \includegraphics[width=\textwidth, height=9cm]{assets/plots/unsafe-usages-by-module-n30.tikz}}
    \end{center}
    \caption{unsafe usages by module for the top 30 modules with most usages}
    \label{fig:unsafe-usages-by-module-n30}
\end{figure}

\begin{figure}[ht]
    \begin{center}
    {\scriptsize \includegraphics[width=\textwidth, height=9cm]{assets/plots/unsafe-usages-by-package-n30.tikz}}
    \end{center}
    \caption{unsafe usages by package for the top 30 packages with most usages}
    \label{fig:unsafe-usages-by-package-n30}
\end{figure}

\begin{figure}[ht]
    \begin{center}
    {\scriptsize \includegraphics[width=\textwidth, height=7cm]{assets/plots/unsafe-usages-by-registry.tikz}}
    \end{center}
    \caption{unsafe usages by registry}
    \label{fig:unsafe-usages-by-registry}
\end{figure}


%% -----------------------------------------------------------------------------

\subsection{Distribution of different unsafe token types}\label{subsec:results-tokens}


%% -----------------------------------------------------------------------------

\subsection{Module and package popularity}\label{subsec:results-popularity}


%% -----------------------------------------------------------------------------

\subsection{Fluctuation of unsafe usages count over time}\label{subsec:results-time-change}

\begin{figure}[ht]
    \begin{center}
    {\scriptsize \includegraphics[width=\textwidth, height=10cm]{assets/plots/fluctuation-in-versions-of-sys.tikz}}
    \end{center}
    \caption{Number of unsafe usages in different versions (every other) of sys over time}
    \label{fig:fluctuation-in-versions-of-sys}
\end{figure}

Figure~\ref{fig:fluctuation-in-versions-of-sys} shows the number of \texttt{unsafe.Pointer} usages in different versions
of the \texttt{golang.org/x/sys} package.
We can see that the number changes over time.
In this case, it ranges between 315 and 440.
This is because different versions of the same package can contain different code, and therefore a different number of
\texttt{unsafe.Pointer} usages.
Here the reason for this is that syscall stubs get added or removed.

For \texttt{sys}, the number of unsafe usages gradually increases over time.
Todo: include an example commit compare.

We can take away two important insights from this.

\begin{enumerate}
    \item It is important to pin the version of the module / package when doing analysis on the code, e.g.\ running a
    survey on the usage of unsafe code patterns.
    \item Even when a vulnerability is fixed in the latest version of a library, projects that use that library will
    still be affected until they upgrade the library.
    Different projects can use different versions of the same library.
\end{enumerate}

%% -----------------------------------------------------------------------------

\subsection{Correlations of unsafe tokens used together in a file or statement}\label{subsec:results-correlation-together}


%% -----------------------------------------------------------------------------

\subsection{Go Vet and Gosec tools coverage of unsafe snippets}\label{subsec:results-vet-gosec}


%% -----------------------------------------------------------------------------

\subsection{Correlation of unsafe usages with project features}\label{subsec:results-correlation-project}


%% -----------------------------------------------------------------------------

\section{Publication of data set}\label{sec:survey-publication}

To obtain the data set, do this:

Clone it from Github?
