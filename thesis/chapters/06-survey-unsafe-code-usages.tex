%% -----------------------------------------------------------------------------

\chapter{Survey of unsafe code usages in popular Go projects}\label{ch:survey}

A major contribution of this thesis is an analysis of unsafe code usage in the wild:
the top 500 most popular open source Go projects.

Threat to validity: unable to differentiate dead code in analysis.
This is similar to many studies as summarized in the related work chapter.

%% -----------------------------------------------------------------------------

\section{Most popular open source Go projects}\label{sec:most-popular-projects}

I downloaded the top 500 most popular (by number of stars) Go projects from Github.
They can be found through Github search.

Using the \texttt{github.com/google/go-github/github}~\cite{gogithub} and \texttt{github.com/go-git/go-git/v5}~\cite{gogit}
Go libraries, I found, saved, downloaded, and checkout out these projects using a Go program.

The projects, including the revision I checked out, and some meta data is shown in Table~\ref{tbl:projects}.


%% -----------------------------------------------------------------------------

\section{Methods of generating data points}\label{sec:survey-acquisition-methods}

Iteration of three data acquisition programs, getting faster and faster: Bash, Python, Go.

Loading packages used by the program by running \texttt{go list -deps -json ./...}, then parsing the JSON output.

CSV is used to write data to disk.
Interop with Go structure types is achieved using the \texttt{github.com/gocarina/gocsv}~\cite{gocsv} library.

Program uses the following steps:

\begin{enumerate}
    \item Read projects to analyze from CSV file
    \item Get package list for a given project
    \item Run requested analysis type
    \item Parse output and match findings to package and file
    \item Append findings for this project to CSV file
    \item Continue with step 2 for the next project, until done
\end{enumerate}

There are five different analysis types:

\begin{itemize}
    \item Grep
    \item Go Vet
    \item Gosec
    \item Abstract Syntax Tree
    \item My linter (go-safer)
\end{itemize}

The time needed to run these analysis on the 500 projects described in Section~\ref{sec:most-popular-projects} is shown
in Table~\ref{tbl:survey-analysis-wallclocktime}.

\begin{table}[h]
    \centering
    \caption{Wall-clock runtime for different analysis types used for the data survey}
    \label{tbl:survey-analysis-wallclocktime}
    \begin{tabular}{ll}
        \toprule
        Analysis Type & Wall-Clock Time \\
        \midrule
        (Project Download) & 4 hours \\
        Grep & 15 minutes \\
        Go Vet & 3 hours \\
        Gosec & 24 hours \\
        Abstract Syntax Tree & 1 hour \\
        Go-Safer Linter & 3.5 hours \\
        \bottomrule
    \end{tabular}
\end{table}


%% -----------------------------------------------------------------------------

\section{Description and structure of data set}\label{sec:survey-dataset}

In the data set I obtained, there are the following objects: projects, packages, modules,  Grep findings, Go Vet findings,
Gosec findings, Go-safer (linter) findings, AST unsafe findings, AST functions, AST statements, and error conditions.

The data set size in terms of object count is shown in Table~\ref{tbl:survey-dataset-size}

\begin{table}[h]
    \centering
    \caption{Survey data set size in terms of object count}
    \label{tbl:survey-dataset-size}
    \begin{tabular}{lr}
        \toprule
        Object type & Data point count \\
        \midrule
        Projects & 495 \\
        Packages & 40,384 \\
        Modules & 3,202 \\
        Grep findings & 2,850,140 \\
        Go Vet findings & 120,416 \\
        Gosec findings & 124,165 \\
        Go-safer (linter) findings & 321 \\
        AST unsafe findings & 2,590,346 \\
        AST functions & 782,760 \\
        AST statements & 1,205,214 \\
        \bottomrule
    \end{tabular}
\end{table}

A thing worth mentioning is that there a bit less \acrshort{ast} unsafe findings compared to the Grep unsafe findings.
This is because the Grep run includes comments containing \texttt{unsafe.Pointer}, while the AST run does not.

The different data set objects have several fields that are included in the CSV data.
Appendix~\ref{ch:data-structure} shows the data structure and data types of each object types in multiple tables.


%% -----------------------------------------------------------------------------

\section{Methods of analyzing the data}\label{sec:survey-analysis-methods}

How did I process the data set?
Approach to build results etc.


%% -----------------------------------------------------------------------------

\subsection{Jupyter Notebooks}\label{subsec:survey-jupyter}

Describe purpose, deployment and notebook structure.


%% -----------------------------------------------------------------------------

\subsection{Python Flask application for manual classification}\label{subsec:survey-classification}

Describe purpose, development and deployment.


%% -----------------------------------------------------------------------------

\section{Survey Results}\label{sec:survey-results}

These sections go through the results of the data survey.
Unsafe usages include usages of \texttt{unsafe.Pointer}, \texttt{uintptr}, \texttt{reflect.SliceHeader} and more.


%% -----------------------------------------------------------------------------

\subsection{Usage statistics in projects, modules, packages, and registries}\label{subsec:results-stats}

First is some statistics about how many unsafe usages were found using Grep.
Unsafe usages are occurrences of the following tokens:

\begin{itemize}
    \item \texttt{unsafe.Pointer}
    \item \texttt{unsafe.Sizeof}
    \item \texttt{unsafe.Alignof}
    \item \texttt{unsafe.Offsetof}
    \item \texttt{uintptr}
    \item \texttt{reflect.SliceHeader}
    \item \texttt{reflect.StringHeader}
\end{itemize}

Figure~\ref{fig:unsafe-usages-by-project-n30} shows the top 30 projects with the most unsafe usages.
We see that \texttt{kubernetes/kubernetes} and \texttt{rancher/k3s} lead the list with more than 10,000 usages.

\begin{figure}[ht]
    \centering
    {\scriptsize \includegraphics[width=\textwidth, height=9cm]{assets/plots/stats/unsafe-usages-by-project-n30.tikz}}
    \caption{Unsafe usages by project for the top 30 projects with most usages}
    \label{fig:unsafe-usages-by-project-n30}
\end{figure}

Looking at individual modules, we see from Figure~\ref{fig:unsafe-usages-by-module-n30} that most usages are contained
in the Go standard library.
Second place with still a lot of unsafe usages is the \texttt{k8s/kubernetes} module, followed by
\texttt{golang.org/x/sys}.
The \texttt{sys} module already has some space to the \texttt{kubernetes} one, and after that the modules have much
fewer unsafe usages.

\begin{figure}[ht]
    \centering
    {\scriptsize \includegraphics[width=\textwidth, height=9cm]{assets/plots/stats/unsafe-usages-by-module-n30.tikz}}
    \caption{Unsafe usages by module for the top 30 modules with most usages}
    \label{fig:unsafe-usages-by-module-n30}
\end{figure}

Figure~\ref{fig:unsafe-usages-by-package-n30} shows the top 30 packages with the most unsafe usages.
Keeping in mind that we just learned that the standard library has the most usages, we can now see that they are mostly
in the \texttt{runtime}, \texttt{syscall}, and \texttt{reflect} packages.
We can identify some more packages from this figure which are analyzed in more detail in Chapter~\ref{ch:code-vulnerabilities}.

\begin{figure}[ht]
    \centering
    {\scriptsize \includegraphics[width=\textwidth, height=9cm]{assets/plots/stats/unsafe-usages-by-package-n30.tikz}}
    \caption{Unsafe usages by package for the top 30 packages with most usages}
    \label{fig:unsafe-usages-by-package-n30}
\end{figure}

Looking at Figure~\ref{fig:unsafe-usages-by-registry}, we see that if we skip the standard library and the Kubernetes
situation, almost all unsafe usages are contained in packages that are hosted on github.com.
However, the data also shows (without plot) that github.com is by far the largest registry, therefore this is expected.
The other big registry is gorgonia.org, which hosts a machine learning framework that will be part of the discussion in
Chapter~\ref{ch:code-vulnerabilities}.

\begin{figure}[ht]
    \centering
    {\scriptsize \includegraphics[width=\textwidth, height=7cm]{assets/plots/stats/unsafe-usages-by-registry-n12.tikz}}
    \caption{Unsafe usages by registry for the top 12 registries with most usages}
    \label{fig:unsafe-usages-by-registry}
\end{figure}

Finally, we see from Figure~\ref{fig:fraction-of-unsafe-modules-and-packages} that only about 20 percent of modules, and
less than 5 percent of packages contain unsafe usages.
We conclude that they are clustered in a small group of packages.
In other words, if a package contains \textit{any} unsafe usage then it will probably contain \textit{many}.

\begin{figure}[ht]
    \centering
    \subfigure[Modules]{
        {\scriptsize \includegraphics[width=0.4\textwidth]{assets/plots/stats/pie-modules.tikz}}
        \label{subfig:fraction-of-unsafe-modules}
    }
    \subfigure[Packages]{
        {\scriptsize \includegraphics[width=0.4\textwidth]{assets/plots/stats/pie-packages.tikz}}
        \label{subfig:fraction-of-unsafe-packages}
    }
    \caption{Fraction of modules and packages containing unsafe usages}
    \label{fig:fraction-of-unsafe-modules-and-packages}
\end{figure}


%% -----------------------------------------------------------------------------

\subsection{Distribution of different unsafe token types}\label{subsec:results-tokens-distribution}

As stated above, the Grep analysis searched for different types of unsafe tokens.
Figure~\ref{fig:unsafe-tokens-distribution} gives an overview of the distribution.
We see that \texttt{unsafe.Pointer} and \texttt{uintptr} are by far the most common, with \texttt{uintptr} being only
slightly in the lead.
Looking at the types that are less common might be even more interesting than the most common ones.

\begin{figure}[ht]
    \centering
    {\scriptsize \includegraphics[width=0.5\textwidth, height=9cm]{assets/plots/tokens-distribution/distribution-different-unsafe-token-types.tikz}}
    \caption{Distribution of different types of unsafe token types}
    \label{fig:unsafe-tokens-distribution}
\end{figure}


%% -----------------------------------------------------------------------------

\subsection{Module and package popularity}\label{subsec:results-popularity}

To understand the severity of unsafe usage findings, we correlate with the popularity of modules and packages.
Popularity is measured in terms of number of projects that include the module / package.

Figure~\ref{fig:popularity-module} shows the most popular modules.
The Go standard library as well as some almost-standard, official utility modules are most popular.
The first real non-standard module is \texttt{github.com/golang/protobuf}.

\begin{figure}[ht]
    \centering
    {\scriptsize \includegraphics[width=\textwidth, height=9cm]{assets/plots/popularity/popularity-module-n30.tikz}}
    \caption{Module popularity by number of importing projects, N=30}
    \label{fig:popularity-module}
\end{figure}

Figure~\ref{fig:popularity-package} shows the most popular packages:

\begin{figure}[ht]
    \centering
    {\scriptsize \includegraphics[width=\textwidth, height=9cm]{assets/plots/popularity/popularity-package-without-std-n30.tikz}}
    \caption{Package popularity by number of importing projects, excluding standard library, N=30}
    \label{fig:popularity-package}
\end{figure}


%% -----------------------------------------------------------------------------

\subsection{Fluctuation of unsafe usages count over time}\label{subsec:results-time-change}

\begin{figure}[ht]
    \centering
    {\scriptsize \includegraphics[width=\textwidth, height=10cm]{assets/plots/time-change/fluctuation-in-versions-of-sys.tikz}}
    \caption{Number of unsafe usages in different versions (every other) of sys over time}
    \label{fig:fluctuation-in-versions-of-sys}
\end{figure}

Figure~\ref{fig:fluctuation-in-versions-of-sys} shows the number of \texttt{unsafe.Pointer} usages in different versions
of the \texttt{golang.org/x/sys} package.
We can see that the number changes over time.
In this case, it ranges between 315 and 440.
This is because different versions of the same package can contain different code, and therefore a different number of
\texttt{unsafe.Pointer} usages.
Here the reason for this is that syscall stubs get added or removed.

For \texttt{sys}, the number of unsafe usages gradually increases over time.
Todo: include an example commit compare.

We can take away two important insights from this.

\begin{enumerate}
    \item It is important to pin the version of the module / package when doing analysis on the code, e.g.\ running a
    survey on the usage of unsafe code patterns.
    \item Even when a vulnerability is fixed in the latest version of a library, projects that use that library will
    still be affected until they upgrade the library.
    Different projects can use different versions of the same library.
\end{enumerate}

%% -----------------------------------------------------------------------------

\subsection{Correlations of unsafe tokens used together in a file or statement}\label{subsec:results-correlation-together}

The data shows that 1061 files contain at least one \texttt{unsafe.Pointer}.
1147 files contain at least one \texttt{uintptr}.
There are 496 files that contain each at least one of both.

In terms of line, there are 12135 distinct lines of code that contain \texttt{unsafe.Pointer}.
There are 11951 lines that contain \texttt{uintptr}.
We can already see that since there are more \texttt{uintptr} occurrences in total, but less lines, that \texttt{uintptr}
is clustered more within the same line.

There are 6846 usages of \texttt{unsafe.Pointer} that are within files where at least one other \texttt{uintptr} occurs.
On the other hand, 5939 usages of \texttt{uintptr} are in files where at least one other \texttt{unsafe.Pointer} occurs.

Of about 21,000 unsafe usages, 11,738 are in a function that contains another usage, while 8,841 are not.

There are 13,757 usages in statements that contain no further unsafe usage.
7,246 usages are within statements where at least one other unsafe usage occurs within that same statement.
Those are distributed among 2,123 statements, making an average of 3.4 unsafe usages per statement for the statements
that contain at least one usage.

The important part is that unsafe usages may occur in clusters.
Those are interesting to look into.

\begin{figure}[ht]
    \centering
    {\scriptsize \includegraphics[width=\textwidth, height=3cm]{assets/plots/correlation-together/use-in-same-line.tikz}}
    \caption{Usage of unsafe together in the same line}
    \label{fig:correlations-unsafe-usage-same-line}
\end{figure}

\begin{figure}[ht]
    \centering
    \subfigure[Functions]{
        {\scriptsize \includegraphics[width=0.4\textwidth]{assets/plots/correlation-together/use-in-same-function.tikz}}
        \label{subfig:correlations-unsafe-usage-same-function}
    }
    \subfigure[Statements]{
        {\scriptsize \includegraphics[width=0.4\textwidth]{assets/plots/correlation-together/use-in-same-statement.tikz}}
        \label{subfig:correlations-unsafe-usage-same-statement}
    }
    \caption{Usage of unsafe in the same function or statement}
    \label{fig:correlations-unsafe-usage-same-function-and-statement}
\end{figure}


%% -----------------------------------------------------------------------------

\subsection{Go Vet and Gosec tools coverage of unsafe snippets}\label{subsec:results-vet-gosec}

The Gosec coverage is not interesting because Gosec only has a single rule concerning unsafe usages which is the
existence of \texttt{unsafe.Pointer} in the code.
This is already identified by the Grep analysis.

Go Vet has an analysis pass for possible misuses of \texttt{unsafe.Pointer}.
Figure~\ref{fig:vet-findings-per-project-n30} shows the Vet findings per project for the top 30 projects with the most
findings.
We see that the projects with the most findings are \texttt{gohugoio/hugo}, \texttt{cli/cli},
and \texttt{sql-machine-learning/sqlflow}.
They are different projects than the ones with the highest numbers of unsafe usages.

\begin{figure}[ht]
    \centering
    {\scriptsize \includegraphics[width=\textwidth, height=9cm]{assets/plots/vet-gosec/vet-findings-per-project-n30.tikz}}
    \caption{Go Vet findings per projects for the top 30 projects with the most findings}
    \label{fig:vet-findings-per-project-n30}
\end{figure}

From the correlation between findings shown in Figure~\ref{fig:correlation-vet-grep-findings}, we see that Vet findings
are almost completely unrelated to the unsafe findings.
There are only about 200 lines where an unsafe usage was found and Go Vet issued a warning.
Of those not even all warnings were related to a possible misuse of \texttt{unsafe.Pointer}.
We can conclude that either almost all projects use \texttt{unsafe.Pointer} only in a safe way, or Go Vet is not very
good at detecting possible misuses.
In the following chapters, we will see that the second case is true.
I will also present a new linter tool that detects some of the unsafe patterns identified in this thesis.

\begin{figure}[ht]
    \centering
    {\scriptsize \includegraphics[width=\textwidth, height=4cm]{assets/plots/vet-gosec/vet-findings-correlated-unsafe-findings.tikz}}
    \caption{Correlation of Go Vet and Grep unsafe findings}
    \label{fig:correlation-vet-grep-findings}
\end{figure}

Table~\ref{tbl:vet-misuse-findings} shows the packages that were found to produce Go Vet pointer misuse warnings.
The number of Vet findings is shown before and after deduplication, that is not counting the same line of source code
more than once.
We see that the runtime has incredibly many findings before deduplication.
It still has by far the most warnings after deduplication, but with less than 200 its two orders of magnitude less.
These two orders come exactly from the fact that I analysed around 500 projects, all of which use the \texttt{runtime}
package and all of which therefore add to the non-deduplicated count.

\begin{table}
    \centering
    \caption{Vet possible misuse of unsafe.Pointer findings per package}
    \label{tbl:vet-misuse-findings}

    \subtable[Raw findings before deduplication]{
        \label{subtbl:vet-findings-duplicates}
        \begin{tabular}{lr}
            \toprule
            Package                                 & Number of findings          \\
            \midrule
            runtime                                 &                      60164 \\
            sync/atomic                             &                        338 \\
            strings                                 &                        335 \\
            github.com/modern-go/reflect2           &                         46 \\
            golang.org/x/sys/unix                   &                         16 \\
            github.com/spaolacci/murmur3            &                         11 \\
            gorgonia.org/tensor                     &                          8 \\
            github.com/apache/arrow/go/arrow/math   &                          6 \\
            github.com/minio/simdjson-go            &                          4 \\
            github.com/apache/arrow/go/arrow/memory &                          4 \\
            github.com/AndreasBriese/bbloom         &                          4 \\
            github.com/coocood/bbloom               &                          2 \\
            github.com/segmentio/encoding/json      &                          1 \\
            \bottomrule
        \end{tabular}
    }

    \subtable[After deduplication]{
        \label{subtbl:vet-findings-deduplicated}
        \begin{tabular}{lr}
            \toprule
            Package                                 & Number of findings          \\
            \midrule
            runtime                                 &                         175 \\
            gorgonia.org/tensor                     &                           8 \\
            github.com/apache/arrow/go/arrow/math   &                           6 \\
            github.com/minio/simdjson-go            &                           3 \\
            github.com/apache/arrow/go/arrow/memory &                           3 \\
            github.com/spaolacci/murmur3            &                           2 \\
            github.com/coocood/bbloom               &                           2 \\
            github.com/AndreasBriese/bbloom         &                           2 \\
            sync/atomic                             &                           1 \\
            strings                                 &                           1 \\
            golang.org/x/sys/unix                   &                           1 \\
            github.com/segmentio/encoding/json      &                           1 \\
            github.com/modern-go/reflect2           &                           1 \\
            \bottomrule
        \end{tabular}
    }
\end{table}


%% -----------------------------------------------------------------------------

\subsection{Correlation of unsafe usages with project features}\label{subsec:results-correlation-project}

Figure~\ref{fig:correlations-project-imports-grep-vet} shows a dot for each project.
The x and y axis show the number of modules and packages imported by the project.
We see a linear correlation which is expected.
More modules mean more packages as each module must have at least one package.

The dot color shows the number of unsafe usages found by Grep, and we see that the more packages a project imports, the
more unsafe usages it has.

The size of the dot represents the number of Vet findings for the project.
Here we see something interesting:
The size of the dot is inverse correlated to the number of imports.
One could expect that the more unsafe usages a project has (dot is more red), the more Vet findings it might have (dot
is bigger).
However the opposite is true.
We could conclude that projects that use more unsafe code have a higher overall code quality.
More on this will be elaborated in the discussion in Chapter~\ref{ch:conclusions}.

\begin{figure}[ht]
    \centering
    {\scriptsize \includegraphics[width=\textwidth, height=9cm]{assets/plots/correlation-project/correlation-imports-grep-vet.tikz}}
    \caption{Correlation of number of imports, Grep unsafe, and Vet findings of projects}
    \label{fig:correlations-project-imports-grep-vet}
\end{figure}

Figure~\ref{fig:correlations-project-unsafe-stars} and Figure~\ref{fig:correlations-project-unsafe-age} correlate a
project's unsafe usages count with its number of stars, and its age, respectively.

We see no particular correlation between number of stars and unsafe count.
We can however see that there seem to be two clusters of projects, one with around 5000 usages, and one with at least
6000 usages.

There does not seem to be a correlation between project age and number of unsafe usages either.
The dots look very equally distributed.

\begin{figure}[ht]
    \centering
    {\scriptsize \includegraphics[width=\textwidth, height=9cm]{assets/plots/correlation-project/correlation-unsafe-stars.tikz}}
    \caption{Correlation of number of unsafe usages and number of stars of projects}
    \label{fig:correlations-project-unsafe-stars}
\end{figure}

\begin{figure}[ht]
    \centering
    {\scriptsize \includegraphics[width=\textwidth, height=9cm]{assets/plots/correlation-project/correlation-unsafe-age.tikz}}
    \caption{Correlation of number of unsafe usages and age of projects}
    \label{fig:correlations-project-unsafe-age}
\end{figure}

Finally, Figure~\ref{fig:correlations-module-unsafe-popularity} shows a correlation between popularity and number of
unsafe usages for a module.
It is the only plot in this subsection that is about modules, not projects.
We can clearly see the modules with many unsafe usages tend to get imported rather rarely compared to the other ones.
This has a good and a bad side:
There are no high-impact-high-probability modules, which would contain lots of unsafe usages and are very popular at the
same time.
These would be excellent targets for an attacker.
On the other hand, the modules with many unsafe usages are by default audited less because they are less popular,
meaning that potential bugs are more likely to be missed due to the lower number of users of the module.

\begin{figure}[ht]
    \centering
    {\scriptsize \includegraphics[width=\textwidth, height=9cm]{assets/plots/correlation-project/correlation-unsafe-module-popularity.tikz}}
    \caption{Correlation of number of unsafe usages and popularity of modules}
    \label{fig:correlations-module-unsafe-popularity}
\end{figure}


%% -----------------------------------------------------------------------------

\subsection{Most interesting modules for analysis}\label{subsec:results-most-interesting-modules}

To determine which modules are the most interesting ones for analysis in Chapter~\ref{ch:code-vulnerabilities}, I use
a score.
It is the product of unsafe usage count in a module and the import count of that module, a good indication of its
popularity.
Table~\ref{tbl:most-interesting-modules-by-score} shows the top 20 most interesting modules.

\begin{table}
    \centering
    \caption{Most interesting modules for analysis by score, N=20}
    \label{tbl:most-interesting-modules-by-score}
    \begin{tabular}{rlrrr}
        \toprule
        {} &                           Module & Unsafe usages count & Import count &    Score \\
        \midrule
        1  &                              std &                6676 &          495 &  3304620 \\
        2  &                 golang.org/x/sys &                1163 &          235 &   273305 \\
        3  &                 golang.org/x/net &                  91 &          217 &    19747 \\
        4  &      github.com/json-iterator/go &                 225 &           71 &    15975 \\
        5  &    github.com/modern-go/reflect2 &                 215 &           71 &    15265 \\
        6  &                k8s.io/kubernetes &                1887 &            8 &    15096 \\
        7  &       github.com/golang/protobuf &                  65 &          168 &    10920 \\
        8  &         github.com/gogo/protobuf &                 103 &          105 &    10815 \\
        9  &  github.com/hashicorp/go-msgpack &                 792 &           12 &     9504 \\
        10 &       github.com/ugorji/go/codec &                 823 &           11 &     9053 \\
        11 &               golang.org/x/tools &                 178 &           42 &     7476 \\
        12 &              golang.org/x/crypto &                  28 &          191 &     5348 \\
        13 &             github.com/ugorji/go &                 804 &            6 &     4824 \\
        14 &       github.com/davecgh/go-spew &                  38 &          106 &     4028 \\
        15 &                 go.etcd.io/bbolt &                 148 &           24 &     3552 \\
        16 &   k8s.io/apiextensions-apiserver &                 124 &           26 &     3224 \\
        17 &               gonum.org/v1/gonum &                 416 &            5 &     2080 \\
        18 &         github.com/docker/docker &                  42 &           47 &     1974 \\
        19 &   github.com/vishvananda/netlink &                 158 &           12 &     1896 \\
        20 &                gvisor.dev/gvisor &                1846 &            1 &     1846 \\
        \bottomrule
    \end{tabular}
\end{table}

Based on the table, we can identify interesting modules that are worth looking into:

\begin{itemize}
    \item \texttt{k8s.io/kubernetes} because it has a lot of unsafe usages
    \item \texttt{github.com/json-iterator/go} because it is getting imported quite often and has unsafe usages
    \item \texttt{github.com/modern-go/reflect2}, and
    \item \texttt{github.com/golang/protobuf} due to the same reason
\end{itemize}


%% -----------------------------------------------------------------------------

\subsection{Number of reflect.SliceHeader and reflect.StringHeader usages}\label{subsec:results-sliceheader}

SliceHeader and StringHeader count: 341 and XX (StringHeader todo) after deduplication.

These get analyzed in depth in Chapter~\ref{ch:code-vulnerabilities}, as slices and thus strings provide a very good
attack surface for vulnerabilites coming from misuse of unsafe code.


%% -----------------------------------------------------------------------------

\section{Publication of data set}\label{sec:survey-publication}

To obtain the data set, do this:

Clone it from Github?
