%% ---------------------------------------------------------------------------------------------------------------------

\chapter{Discussion}\label{ch:discussion}

This chapter discusses the results and findings of this thesis.
First, the dangers and practical exploitability of real-world Go applications due to usages of the \unsafe{} API are
elaborated.
Then, the patches that were sent to open-source code maintainers are described in detail, and suggestions towards a
world with safer usage practices of \unsafe{} code are given.
Finally, improvements in upcoming versions of the Go programming language are presented, and threats to validity are
discussed.


%% ---------------------------------------------------------------------------------------------------------------------

\section{Practical Exploitability of \textit{Unsafe} Code}\label{sec:discussion:exploitability}

This thesis took a deep look into the \unsafe{} API of the Go programming language.
It was shown that it is effectively used by a large fraction (\percentageUnsafeTransitiveWithDependencies{}) of the
\projsAnalyzed{} analyzed top-starred open-source projects, either directly or by including \unsafe{} usages through
third-party dependencies.
While there are many legitimate use cases for \unsafe{} code, such as efficient conversions between otherwise
incompatible types without allocating additional memory, or accessing the foreign function interface (\acrshort{FFI}),
a thorough manual study also showed that there are dangers and possible vulnerabilities that can come from it.
The data collected when creating the labeled data set of annotated \unsafe{} usage examples shows that the  majority of
the usages are safe and legitimate, but there were also several bugs that could lead to severe vulnerabilities.

A contribution of this thesis is novel evidence of possible misuses of the \unsafe{} \acrshort{API} with concrete proof
showing how a vulnerability is introduced.
The changes needed to avoid the misuses are often minor and subtle, which underlines the fact that \unsafe{} is a
fragile and dangerous feature of the Go programming language.
It is complicated to understand the implications that a given piece of code has, and could have in the future,
especially if there is interaction with other parts of the code base and other developers.
If a bug is introduced, it can easily slip through code review and be part of a zero day exploit some time later.
The bugs that were found during the course of this work, and to which patches were submitted as described in the next
section, may seem few and minor at first glance, but it only takes one \textit{use-after-free} bug at a critical
position to open up the possibility of a major remote code execution attack.

In my opinion, the general direction that programming languages like Go or Rust take is the right one.
By enforcing a strict safety net in most cases and still allowing escape hatches to circumvent these measures when it is
absolutely necessary, they combine the best of two worlds.
In large parts of the code base, buffer overflows, race conditions etc. are mostly prevented at the small cost of a
slightly more rigid coding workflow and developer training.
Still, there is the option of complete flexibility and total control over the memory if it should be needed.
This allows organizations to focus auditing and review efforts on the code parts that use the \unsafe{} \acrshort{API},
which is a fraction of the total code.


%% ---------------------------------------------------------------------------------------------------------------------

\section{Patching Open-Source Projects}\label{sec:discussion:patches}

In the process of analyzing \unsafe{} usages in real-world Go code, and using the novel static analysis tool
\toolSafer{}, \numberBugsFixed{} unsafe-related bugs were found that can lead to security vulnerabilities.
Most of them (\checkNum{63}) are instances of incorrect constructions of slice header values, which are used with
in-place conversions between slices of different types.
These can cause \textit{use-after-free} vulnerabilities due to the garbage collector freeing a value that is still
in use or an error in the escape analysis causing a value to be placed on the stack incorrectly, as described in
Sections~\ref{subsec:unsafe-security-problems:slice-casts:gc-race}
and~\ref{subsec:unsafe-security-problems:slice-casts:escape-analysis}.
Furthermore, there is the bug causing incorrect length information in a slice in the \goFuse{} library described in
Section~\ref{subsec:unsafe-security-problems:slice-casts:incorrect-length}.

During the course of this work, \numberPRs{} pull requests with patches to these bugs were submitted to the authors of
the affected libraries on \github{}.
These pull requests contain fixes to one or more bugs, adding up to a total number of \checkNum{63} resolved
bugs.
Furthermore, \checkNum{one} additional pull request with \checkNum{one} more bug already existed.
They are listed in Table~\ref{tbl:pull-requests} along with their respective \github{} projects.
The Popularity column shows how many projects in the data set of popular Go projects use the respective library,
which is related to the impact of the bugs.
Furthermore, the table indicates which of the pull requests have been merged so far, and how many bugs they contain
individually.

\begin{table}[htp!]
    \centering
    \caption{Pull requests on \github{} to fix vulnerable Go libraries}
    \label{tbl:pull-requests}
    {\footnotesize
        \begin{tabularx}{\textwidth}{rlXrrll}
        {} & \textbf{Project}             & \textbf{PR-Name}                                                & \textbf{Popularity} & \textbf{Bugs} & \textbf{PR Number}                                                     & \textbf{Merged}  \\
        \hline
        1  & hanwen/go-fuse               & batch forget: fix missing return to handle malformed input data &  3 projects         &  1            & \href{https://www.github.com/hanwen/go-fuse/pull/363}{PR \#363}        & no               \\
        \rowcolor{verylightgray}
        2  & buger/jsonparser             & Fix possible memory confusion in unsafe slice cast              &  4 projects         &  1            & \href{https://www.github.com/buger/jsonparser/pull/204}{PR \#204}      & yes              \\
        3  & elastic/go-structform        & Fix possible memory\ldots                                       &  1 project~         &  2            & \href{https://github.com/elastic/go-structform/pull/21}{PR \#21}       & yes              \\
        \rowcolor{verylightgray}
        4  & go-fiber/utils               & Fix possible memory\ldots                                       &  1 project~         &  1            & \href{https://github.com/gofiber/utils/pull/7}{PR \#7}                 & yes              \\
        5  & influxdata/influxdb          & fix: possible memory\ldots                                      &  8 projects         &  1            & \href{https://github.com/influxdata/influxdb/pull/18307}{PR \#18307}   & no               \\
        \rowcolor{verylightgray}
        6  & influxdata/influxdb1-client  & fix: possible memory\ldots                                      &  4 projects         &  1            & \href{https://github.com/influxdata/influxdb1-client/pull/40}{PR \#40} & no               \\
        7  & modern-go/reflect2           & Fix possible memory\ldots                                       & 71 projects         &  1            & \href{https://github.com/modern-go/reflect2/pull/13}{PR \#13}          & yes              \\
        \rowcolor{verylightgray}
        8  & savsgio/gotils               & Fix possible memory\ldots                                       &  1 project~         &  2            & \href{https://github.com/savsgio/gotils/pull/2}{PR \#2}                & yes              \\
        9  & valyala/fasttemplate         & Fix possible memory\ldots                                       & 10 projects         &  1            & \href{https://github.com/valyala/fasttemplate/pull/21}{PR \#21}        & yes              \\
        \rowcolor{verylightgray}
        10 & weaveworks/ps                & Fix possible memory\ldots                                       &  1 project~         &  1            & \href{https://github.com/weaveworks/ps/pull/3}{PR \#3}                 & no               \\
        11 & yuin/goldmark                & Fix possible memory\ldots                                       &  5 projects         &  1            & \href{https://github.com/yuin/goldmark/pull/134}{PR \#134}             & yes              \\
        \rowcolor{verylightgray}
        12 & yuin/gopher-lua              & Fix possible memory\ldots                                       &  6 projects         &  1            & \href{https://github.com/yuin/gopher-lua/pull/287}{PR \#287}           & yes              \\
        13 & gorgonia/tensor              & Fix possible memory\ldots                                       &  1 project~         &  1            & \href{https://github.com/gorgonia/tensor/pull/68}{PR \#68}             & yes              \\
        \rowcolor{verylightgray}
        14 & gorgonia/tensor              & Fix 48 more possible\ldots                                      &  1 project~         & 48            & \href{https://github.com/gorgonia/tensor/pull/72}{PR \#72}             & yes              \\
        15 & mailru/easyjson              & fix unsafe unsafe usage\newline \textit{(previously existing)}  & 42 projects         &  1            & \href{https://github.com/mailru/easyjson/pull/258}{PR \#258}           & no               \\
        \hline
        {} & \multicolumn{3}{l}{\textbf{Total}}                                                                                   & \textbf{64}   & {}                                                                     & \textbf{10 / 15} \\
        \end{tabularx}
    }
\end{table}

Sending pull requests on \github{} is a public disclosure procedure, although the bugs have not been announced on any
news pages.
Given that there are currently no actual exploits that use the bugs to inject code or leak data, this was a good choice
compared to other procedures like a responsible disclosure, for example.
Submitting \github{} pull requests allows the code authors to easily merge the necessary changes.
So far, \numberPRsMerged{} of the pull requests have been reviewed, acknowledged, and accepted by the authors.
Thus, \numberBugsMerged{} (\fractionBugsMerged{}) of the bugs have been fixed.
The remaining were not rejected either, but received no attention due to a generally high volume of open pull requests
in the project.


%% ---------------------------------------------------------------------------------------------------------------------

\section{Suggestions for Safer Go Code}\label{sec:discussion:safer-go-code}

Usages of \unsafe{} that get imported through third-party dependencies are dangerous, because they can be hidden at
first glance, but still get compiled into the resulting application binary.
On top of the fact that external dependencies tend to get updated rather late, if at all, as was discussed in
Section~\ref{sec:related-work:dependency-issues}, there is a possibility that there is simply nobody even aware of the
potential danger.
The novel tool \toolGeiger{} is a step in the journey of improving this situation, as it allows to quickly and
effectively differentiate the libraries that need auditing from those that do not.
An effective next step would be to use this information even before deciding which library to use.
If there are multiple external libraries that achieve the same goal, choosing which one to use should incorporate the
data about how many \unsafe{} usages each library contains, if any.
Libraries with less \unsafe{} code should be preferred over those with a lot.
Previous work has proposed various code metrics to help choosing a specific library~\cite{delamora2018}.
They include security and fault-proneness, both of which could be influenced by the use of \unsafe{}.
It would be beneficial to contribute towards a situation where library maintainers use the number of \unsafe{} usages
in their library as a feature or part of its advertisement.
Similar to code quality report badges on the \github{} repository of the project, there could be a badge showing if the
library uses any \unsafe{} code, and if it does how much.
However, it is important to remember that the number of \unsafe{} usages is not itself an indicator for the safety of
a library or project.
First, security problems can also be introduced in code that does not use \unsafe{} at all.
Second, there could be many well audited \unsafe{} usages in one project, and a single one that causes a vulnerability
in another one.
Although the number of \unsafe{} usages in the second project is smaller, it might be much less secure.


%% ---------------------------------------------------------------------------------------------------------------------

\section{Improvements in Upcoming Go Releases}\label{sec:discussion:changes-in-go}

Since the study data for this thesis has been collected, the most recent Go release \checkNum{1.15} from
\checkNum{August 2020} has enabled additional static checking of \textit{unsafe.Pointer} usages with the
\textit{-d=checkptr} compiler flag\footnote{\url{https://golang.org/doc/go1.14\#compiler}}.
The development of the new check had been started with Go release \checkNum{1.14} already.
It checks that the alignment of the target type matches the alignment of the pointer when converting \unsafe{} pointers
to an arbitrary type.
Furthermore, the result of pointer arithmetic can not be a completely arbitrary heap value anymore, instead at least one
argument of the arithmetic expression must already be a pointer into the resulting value.
Both checks prevent some misuses of \unsafe{} that can occur if the developer took a false assumption over the types
involved in a conversion, and therefore they are a valuable step towards a safer usage of \unsafe{} in Go.
Architecture-dependent types as covered in Section~\ref{sec:unsafe-security-problems:architecture-dependent-types} are
less dangerous with this new check, because it detects a mismatch in memory alignment or byte order when the application
is compiled for an unsuitable architecture.
However, the remaining exploits discussed in Chapter~\ref{ch:unsafe-security-problems} remain possible with the new
check.
Also, it can not find the misuses that are detected by \toolSafer{}, which underlines the need of continuous improvement
of the developer tools and continued awareness of potentially still unknown dangers with the use of \unsafe{}.

There is an open proposal for introducing a new type \textit{unsafe.Slice} into the Go \unsafe{}
package\footnote{\url{https://github.com/golang/go/issues/19367}}.
It is supposed to replace the \textit{reflect.SliceHeader} type and differs in the type of the \textit{Data} field.
While that field is a \textit{uintptr} in the \textit{reflect} package, the proposed type uses \textit{unsafe.Pointer}.
Thus, the new representation for a slice stores the reference to the underlying array in a type that is treated as a
pointer by the garbage collector and escape analysis.
This solves the slice vulnerabilities described in Sections~\ref{subsec:unsafe-security-problems:slice-casts:gc-race}
and~\ref{subsec:unsafe-security-problems:slice-casts:escape-analysis}, which are rather common in real Go applications.
Therefore, it would provide a significantly safer way of writing such code.
By the time this thesis is submitted, the proposal is actively discussed and patches to the Go compiler have already
been accepted.
The feature is currently planned to be included in the \checkNum{1.17} release of Go, which would be expected to be
finished around \checkNum{August, 2021}.

Finally, some \unsafe{} usages in the analyzed open-source projects were necessary to achieve functionality that could
have been implemented with generics, if they were available in Go.
Generics are in fact one of the most widely requested features for the Go programming language, with \checkNum{79\%} of
participants in the \checkNum{2019} Go Developer Survey\footnote{\url{https://blog.golang.org/survey2019-results}},
who answered the particular question, stating that they think it is a critical missing feature.
Since generics have now been officially announced\footnote{\url{https://blog.golang.org/go2-next-steps}} for the
upcoming Go release \checkNum{2.0}, these \unsafe{} usages can then be replaced by the new generics support.
Similarly, other language improvements such as a new foreign function interface could make other usages of \unsafe{},
which are necessary at the moment, obsolete, decreasing the room for errors made by the developers and increasing
safety.


%% ---------------------------------------------------------------------------------------------------------------------

\section{Threats to Validity}\label{sec:discussion:threats-to-validity}

This section discusses potential threats to the validity of this work.
Internal threats concern the quality of execution, or whether any results could be explained by other factors that were
not covered in this thesis.
The empirical study and quantitative evaluation presented in Section~\ref{sec:go-geiger:quantitative-evaluation}
focused on code for the \textit{amd64} architecture.
This is due to a limitation of \toolGeiger{}, which currently only finds \unsafe{} usages in code targeted for the
architecture that \toolGeiger{} is executed on.
There could possibly be a different prevalence of \unsafe{}, or a contradictory distribution of \unsafe{} token types,
on other architectures.
This threat is mitigated first by the fact that \textit{amd64} is by far the most commonly used architecture in the
data set presented in this thesis, therefore code for other architectures should not have a large influence on the
statistics.
Secondly, an analysis using \textit{grep} showed similar results to the study using \toolGeiger{}.
While \textit{grep} can not exclude unreachable code, it includes code for all architectures.

Furthermore, the manual analysis of \unsafe{} code patterns could have missed some vulnerabilities.
In fact, it is almost guaranteed that there are more ways to exploit \unsafe{} usages.
However, the objective of this thesis is not a necessarily exhaustive search for all possible exploit vectors.
Instead, the contributions of this work include an analysis that did reveal multiple vulnerabilities, thus providing a
step towards safer Go applications.

On the other hand, external threats to validity concern the ability to generalize the study results to new data.
First, the raw data for the empirical study in Section~\ref{sec:go-geiger:quantitative-evaluation}, as well as the
labeled data set described in Section~\ref{sec:go-geiger:qualitative-evaluation}, was gathered from the most popular
projects on \github{}, measured by their number of stars.
Furthermore, projects without support for the Go module system and projects that would not compile on our machines were
excluded.
It is possible that the usage of \unsafe{} is different in less popular projects.
This is especially true for projects with fewer stars, where there might not be as many developers reviewing \unsafe{}
usages due to a smaller public interest.
Projects which have not yet adopted the module system or contain build errors could have a generally lower code quality,
which means there could be a bias in the study data towards higher-quality projects.
However, choosing projects by popularity mitigates possible bias towards specific domains of projects, such as
virtualization or databases, because the number of stars is an easy and rather neutral metric.
Therefore, the project set enables a good overview on the usage of \unsafe{} in a broad selection of projects.
