%% ---------------------------------------------------------------------------------------------------------------------

\chapter{Discussion}\label{ch:discussion}

This chapter puts the results of this thesis in context and discusses the findings.
First, the dangers and practical exploitability of real-world Go applications due to usages of the \unsafe{} API is
elaborated.
Then, we describe in detail the patches that we sent to open-source code maintainers and give suggestions towards safer
usage best practices for \unsafe{} code.
Finally, threats to validity and future work are presented.


%% ---------------------------------------------------------------------------------------------------------------------

\section{Practical Exploitability of Unsafe}\label{sec:discussion:exploitability}

In this thesis, we took a deep look into the \unsafe{} API of the Go programming language.
We showed that it is effectively used by almost all of the \projsAnalyzed{} top-starred open-source projects
(\percentageUnsafeTransitiveWithDependencies{}).


%% ---------------------------------------------------------------------------------------------------------------------

\section{Patching Open-Source Projects}\label{sec:discussion:patches}

In the process of analyzing \unsafe{} usages in real-world Go code, and using our novel static analysis tool
\toolSafer{}, we found \numberBugsFixed{} unsafe-related bugs that can lead to security vulnerabilities.
Most of them are instances of incorrect constructions of slice header values, which are used with in-place conversions
between slices of different types.
These can cause the \textit{use-after-free} vulnerabilities due to the garbage collector freeing a value that is still
in use or an error in the escape analysis causing a value to be placed on the stack incorrectly, as described in
Sections~\ref{subsec:unsafe-security-problems:slice-casts:gc-race}
and~\ref{subsec:unsafe-security-problems:slice-casts:escape-analysis}.
Furthermore, there is the bug causing incorrect length information in a slice in the \goFuse{} library as described in
Section~\ref{subsec:unsafe-security-problems:slice-casts:incorrect-length}.

We submitted \numberPRs{} pull requests with patches to these bugs to the authors of the affected libraries on
\github{}.
This is a public disclosure procedure, although we did not announce the bugs on any news pages.
Given that there are currently no actual exploits that use the bugs to inject code or leak data, this was a good choice
compared to other procedures like a responsible disclosure for example.
Submitting \github{} pull requests allows the code authors to easily merge the necessary changes.

\begin{table}[htp!]
    \centering
    \caption{Pull requests on \github{} to fix vulnerable Go libraries}
    \label{tbl:pull-requests}
    {\footnotesize
        \begin{tabularx}{\textwidth}{rlXrrll}
        {} & \textbf{Project}             & \textbf{PR-Name}                                                & \textbf{Popularity} & \textbf{Bugs} & \textbf{PR Number}                                                     & \textbf{Merged}  \\
        \hline
        1  & hanwen/go-fuse               & batch forget: fix missing return to handle malformed input data &  3 projects         &  1            & \href{https://www.github.com/hanwen/go-fuse/pull/363}{PR \#363}        & no               \\
        \rowcolor{verylightgray}
        2  & buger/jsonparser             & Fix possible memory confusion in unsafe slice cast              &  4 projects         &  1            & \href{https://www.github.com/buger/jsonparser/pull/204}{PR \#204}      & yes              \\
        3  & elastic/go-structform        & Fix possible memory\ldots                                       &  1 project~         &  2            & \href{https://github.com/elastic/go-structform/pull/21}{PR \#21}       & yes              \\
        \rowcolor{verylightgray}
        4  & go-fiber/utils               & Fix possible memory\ldots                                       &  1 project~         &  1            & \href{https://github.com/gofiber/utils/pull/7}{PR \#7}                 & yes              \\
        5  & influxdata/influxdb          & fix: possible memory\ldots                                      &  8 projects         &  1            & \href{https://github.com/influxdata/influxdb/pull/18307}{PR \#18307}   & no               \\
        \rowcolor{verylightgray}
        6  & influxdata/influxdb1-client  & fix: possible memory\ldots                                      &  4 projects         &  1            & \href{https://github.com/influxdata/influxdb1-client/pull/40}{PR \#40} & no               \\
        7  & modern-go/reflect2           & Fix possible memory\ldots                                       & 71 projects         &  1            & \href{https://github.com/modern-go/reflect2/pull/13}{PR \#13}          & yes              \\
        \rowcolor{verylightgray}
        8  & savsgio/gotils               & Fix possible memory\ldots                                       &  1 project~         &  2            & \href{https://github.com/savsgio/gotils/pull/2}{PR \#2}                & yes              \\
        9  & valyala/fasttemplate         & Fix possible memory\ldots                                       & 10 projects         &  1            & \href{https://github.com/valyala/fasttemplate/pull/21}{PR \#21}        & yes              \\
        \rowcolor{verylightgray}
        10 & weaveworks/ps                & Fix possible memory\ldots                                       &  1 project~         &  1            & \href{https://github.com/weaveworks/ps/pull/3}{PR \#3}                 & no               \\
        11 & yuin/goldmark                & Fix possible memory\ldots                                       &  5 projects         &  1            & \href{https://github.com/yuin/goldmark/pull/134}{PR \#134}             & yes              \\
        \rowcolor{verylightgray}
        12 & yuin/gopher-lua              & Fix possible memory\ldots                                       &  6 projects         &  1            & \href{https://github.com/yuin/gopher-lua/pull/287}{PR \#287}           & yes              \\
        13 & gorgonia/tensor              & Fix possible memory\ldots                                       &  1 project~         &  1            & \href{https://github.com/gorgonia/tensor/pull/68}{PR \#68}             & yes              \\
        \rowcolor{verylightgray}
        14 & gorgonia/tensor              & Fix 48 more possible\ldots                                      &  1 project~         & 48            & \href{https://github.com/gorgonia/tensor/pull/72}{PR \#72}             & yes              \\
        15 & mailru/easyjson              & fix unsafe unsafe usage\newline \textit{(previously existing)}  & 42 projects         &  1            & \href{https://github.com/mailru/easyjson/pull/258}{PR \#258}           & no               \\
        \hline
        {} & \multicolumn{3}{l}{\textbf{Total}}                                                                                   & \textbf{64}   & {}                                                                     & \textbf{10 / 15} \\
        \end{tabularx}
    }
\end{table}

Table~\ref{tbl:pull-requests} lists the pull requests with their respective \github{} project.
The Popularity column indicates how many projects in our data set of the top \projsTotal{} most popular Go projects use
the respective library, showing the impact of bugs.
For the bug in \textit{mailu/easyjson}, there was already an existing open pull request.
So far \numberPRsMerged{} of the pull requests have been accepted by the authors.
The remaining were not rejected either, but received no attention due to a generally high volume of open pull requests
in the project.


%% ---------------------------------------------------------------------------------------------------------------------

\section{Suggestions for Safer Go Code}\label{sec:discussion:safer-go-code}

With our two novel static analysis tools \toolGeiger{} and \toolSafer{}, we contributed valuable developer tools that
can assist in managing the dangers associated with the use of \unsafe{} code.
Looking at the identified reasons to use the \unsafe{} API as shown by our manually labeled data set of classified
usages as described in Section~\ref{sec:go-geiger:labeled-dataset}, it is clear that there are in fact many legitimate
and safe use cases for it.
It is therefore important to further improve the safety of such operations.

Since the study data for this thesis has been collected, the most recent Go release \checkNum{1.15} from
\checkNum{August 2020} has introduced dynamic checking of unsafe pointers both into the runtime and the \toolVet{} tool.


%% ---------------------------------------------------------------------------------------------------------------------

\section{Threats to Validity}\label{sec:discussion:threats-to-validity}

This section discusses potential threats to validity to our empirical study of \unsafe{} usages in popular open-source
Go projects.
Internal threats concern the quality of execution for our study.
They include a bias towards bigger projects because with a higher number of lines of code, they could be
overrepresented in the data set of manually labeled \unsafe{} usages.
Next, we only considered projects that use the Go module system and could compile on our machines.

On the other hand, external threats to validity concern the ability to generalize our study results to new data.
They include a bias towards certain projects as well, as the distribution of \unsafe{} usages might be correlated to the
project size, activity or number of contributors, or others.
Furthermore, when analyzing \unsafe{} code examples for security, we might have excluded usage patterns that are
uncommon in the top \projsTotal{} most popular projects, but frequent in projects of a specific problem domain.


%% ---------------------------------------------------------------------------------------------------------------------

\section{Future Work}\label{sec:discussion:future-work}

Based on the contributions of this thesis, future research might look into the following areas.
First, our novel data set of manually labeled \unsafe{} usages, containing code samples along labels on what is being
done and for what purpose, could be used as a training and verification data set in machine learning applications.
It could be used for example to evaluate possible future static analysis tools with a focus on \unsafe{} Go code,
to train an automated classification tool that could possibly predict the danger of new \unsafe{} code patterns.
Furthermore, it can be used as a seed for further studies on the design and adoption of \unsafe{} APIs not only in Go
but also across other languages.

Next, our novel developer tools \toolGeiger{} and \toolSafer{} presented in this thesis could be integrated into
developer \acrshort{IDE}s.
While the current command line interface works well for the initial development of the tools, is a preferred user
interface for some developers anyway, and is especially suitable to be used as part of a continuous integration
(\acrshort{CI}) automation, an \acrshort{IDE} integration has the benefit of being able to unify the display of warnings
with the editor that the developer uses to fix them.
For example, \toolSafer{} warnings can be displayed next to the affected line directly in the editor, and the code that
violates the safety rules could be highlighted to effectively draw the attention of the developer.
As described in Section~\ref{sec:related-work:static-code-analysis}, this can lead to a higher fraction of warnings
actually being resolved.
Such an \acrshort{IDE} integration could be implemented using the language server
protocol~\footnote{\url{https://langserver.org/}}, as this allows for easy compatibility with most \acrshort{IDE}s
without the need to develop individual plugins for each of them.

Finally, future work could integrate the static code analysis tool \toolSafer{} into \toolVet{}, the official linter
for Go code that ships as part of the standard compiler toolchain.
This would be fairly easy because \toolSafer{} is build using the \toolVet{} infrastructure.
Therefore, the analysis passes provided by \toolSafer{} could directly be added to the set of analysis shipped with
\toolVet{}.
Achieving this would reduce the effort to use \toolSafer{} a lot and could therefore increase the adoption of the tool,
even only passively as part of using \toolVet{} with a \acrshort{CI} workflow, leading to safer Go applications overall.
