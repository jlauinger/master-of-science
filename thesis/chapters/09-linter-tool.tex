%% -----------------------------------------------------------------------------

\chapter{Linter tool for unsafe patterns}\label{ch:go-safer}

Another major contribution of this thesis is the developement of a Go Vet-style, open-source linter tool. It can
identify some of the unsafe code patterns discussed in Chapter~\ref{ch:code-vulnerabilities}.


%% -----------------------------------------------------------------------------

\section{Vulnerability patterns found by the linter}\label{sec:go-safer-scope}


%% -----------------------------------------------------------------------------

\section{Implementation of a linter tool}\label{sec:go-safer-implementation}

How is the linter implemented?


\subsection{Go analysis infrastructure}\label{subsec:go-safer-analysis-infrastucture}

Go Vet is built using the Go analysis infrastructure. This linter uses the same.


\subsection{Design of the sliceheader pass}\label{subsec:go-safer-passes-sliceheader}

It uses the following \acrshort{ast} patterns.


\subsection{Design of the structcast pass}\label{subsec:go-safer-passes-structcast}

It uses the following AST patterns.


\subsection{Open-sourcing to Github}\label{subsec:go-safer-github}

Available as open-source software.


%% -----------------------------------------------------------------------------

\section{Verification and evaluation of the linter tool}\label{sec:go-safer-evaluation}

Automatic testing

Amount of snippets manually found in Chapter~\ref{ch:survey} are detected by the linter tool:


%% -----------------------------------------------------------------------------

\section{Distribution in IDE plugin}\label{sec:go-safer-ide-plugin}

The tool is available as plugins for the common developer IDEs JetBrains GoLand and Eclipse.

Explain how the plugin development was done, and how to install the program. Include
screenshots.