%% ---------------------------------------------------------------------------------------------------------------------

\chapter{\textit{go-safer}: Detecting Unsafe Misuses}\label{ch:go-safer}

Another major contribution of this thesis is the development of a Go Vet-style, open-source linter tool.
It can identify some of the unsafe code patterns.

\begin{figure}[ht]
    \includegraphics[width=\textwidth]{assets/figures/chapter5/outline5.pdf}
    \caption{Role of Chapter 5 in Thesis Outline}
    \label{fig:outline5}
\end{figure}



%% ---------------------------------------------------------------------------------------------------------------------

\section{Design}\label{sec:go-safer:design}

Top-level approach
Usage example
Publication

\begin{figure}[htp!]
    %\vspace{2mm}
    \centering
    \includegraphics[width=0.7\textwidth]{assets/figures/chapter5/go-safer-architecture.pdf}
    \caption{Architecture of the \toolSafer{} static code analysis tool}
    \label{fig:safer-architecture}
    %\vspace{-14pt}
\end{figure}



%% ---------------------------------------------------------------------------------------------------------------------

\section{Implementation}\label{sec:go-safer:implementation}

Go vet analysis pass infrastructure
Low-level details
Verification with tests


%% ---------------------------------------------------------------------------------------------------------------------

\section{Evaluation}\label{sec:go-safer:evaluation}


%% ---------------------------------------------------------------------------------------------------------------------

\subsection{Labeled Usages}\label{subsec:go-safer:evaluation:labeled-usages}

Precision, Recall, calculated using manually labeled usages data set


%% ---------------------------------------------------------------------------------------------------------------------

\subsection{Case Studies}\label{subsec:go-safer:evaluation:case-studies}

Manual inspection of some projects, used to calculate precision / recall of go-safer


%% ---------------------------------------------------------------------------------------------------------------------

\subsection{Comparison with Existing Tools}\label{subsec:go-safer:evaluation:linters-comparison}

Go vet / Gosec
