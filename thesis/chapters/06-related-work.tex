%% ---------------------------------------------------------------------------------------------------------------------

\chapter{Related Work}\label{ch:related-work}

This chapter discusses related and concurrent work relevant to this thesis.
First, a similar, concurrent study on \unsafe{} code usages in the Go programming language is presented, replicated, and
compared in detail to this work.
Then, related studies and papers are discussed in groups identified by their general topic.


%% ---------------------------------------------------------------------------------------------------------------------

\section{Concurrent Study on Unsafe Go}\label{sec:related-work:concurrent-study}

Costa et al. submitted a similar study to this work on usages of the \unsafe{} \acrshort{API} in open-source Go code to
the IEEE Transactions on Software Engineering journal.
At the time this thesis is written, their paper is not yet accepted, thus it is cited from ArXiV~\cite{costa2020}.
While I personally could not find substantial errors in the paper, it has not yet been peer-reviewed.
Since it is a very relevant concurrent work to this thesis, I nevertheless reproduced their empirical study and compared
it to the results of this thesis.

The authors present a study of \checkNum{2,438} popular Go projects which are downloaded on \checkNum{October 2nd, 2019}.
Their project selection was done by taking the top \checkNum{3,000} most-starred open-source projects on \github{},
filtering out archived, inactive, and young projects with less than \checkNum{10} commits.
Furthermore, they removed educational projects such as books or learning material, as well as projects that failed to
be downloaded for some reason.
Then, they counted the number of \unsafe{} usages in each project using a parser to generate the abstract syntax tree
(\acrshort{AST}).
They find that \checkNum{24\%} of the studied projects use \unsafe{} code.
In contrast, the study presented in this thesis in Chapter~\ref{ch:go-geiger} finds that \percentageUnsafeProjects{}
use \unsafe{}.
This difference is due to the different choice of projects, which was verified by comparing the set of analyzed projects
available through the published replication package\footnote{\url{https://zenodo.org/record/3871931}}.
While Costa et al. analyzed more than \checkNum{2,400} projects, I only analyzed \projsAnalyzed{}, indicating that
\unsafe{} usage is more prevalent in more popular projects.
Furthermore, a fundamental difference of this work to~\cite{costa2020} is that the study by Costa et al. did not look
at dependencies, instead they only analyzed the projects directly.
My analysis on the projects including their dependencies in contrast showed that
\percentageUnsafeTransitiveWithDependencies{} of the projects use \unsafe{} either directly or through some included
library.
To compare the study to~\cite{costa2020}, it is therefore necessary to focus on the \unsafe{} usages contained in the
top-level projects, which is done by taking into account only usages that are detected within the main project module
indicated by the \textit{go.mod} file in the root directory.
However, this can be inaccurate if there are multiple modules contained within the same project repository, because in
that case the study by Costa et al. would include code for that project that is seen as dependency in my study.
There are \checkNum{25} projects for which~\cite{costa2020} reports \unsafe{} usages, but my study does not, however
they are all included in the \percentageUnsafeTransitiveWithDependencies{} of projects for which my study indicates that
they use \unsafe{} code through their dependencies.
For this reason, I conclude that the difference in the fraction of projects using \unsafe{} is due to a different
definition of project and dependency code.

There are \checkNum{86} projects for which my study and~\cite{costa2020} reported different absolute numbers of
\unsafe{} usage sites, while the numbers matched exactly for all remaining projects.
To find the reason for the difference, I replicated the study to the best of my possibilities given that Costa et al.
did neither include the exact revisions of the projects under analysis nor their parsing tool to count the \unsafe{}
usages in their replication package.
Table~\ref{tbl:costa-counts-comparison} shows the top \checkNum{10} projects with highest difference, including their
respective absolute counts found in the studies and the reason for the difference.

\begin{table}[h]
    \centering
    \caption[Comparison of \unsafe{} usage counts Costa et al. and this work]%
    {Comparison of \unsafe{} usage counts in Costa et al. and this work~\newline \tiny ~\newline \footnotesize
    Reasons: \underline{a} change over time, \underline{b} multiple architecture, \underline{c} dependency vendoring,
    \underline{d} submodules \tiny ~\newline}
    \label{tbl:costa-counts-comparison}
    \begin{tabular}{l|r|r|r|l}
        \textbf{Project}       & \textbf{Count in~\cite{costa2020}} & \textbf{Count in this work} & \textbf{Difference} & \textbf{Reason} \\
        \hline
        jetstack/cert-manager  &                                140 &                         374 &               + 234 &  a              \\
        \rowcolor{verylightgray}
        kubernetes/kubernetes  &                              2,058 &                       1,885 &              -- 173 &  b              \\
        golang/mobile          &                                211 &                          93 &              -- 118 &  b              \\
        \rowcolor{verylightgray}
        TykTechnologies/tyk    &                                 79 &                         187 &               + 108 &  c              \\
        elastic/beats          &                                164 &                          70 &               -- 94 &  c              \\
        \rowcolor{verylightgray}
        golang/tools           &                                113 &                          44 &               -- 69 &  b              \\
        peterq/pan-light       &                                 64 &                           0 &               -- 64 &  d              \\
        \rowcolor{verylightgray}
        cilium/cilium          &                                290 &                         345 &               -- 55 &  c              \\
        go-delve/delve         &                                 72 &                          20 &               -- 52 &  b              \\
        \rowcolor{verylightgray}
        ethereum/go-ethereum   &                                 42 &                          91 &                + 49 &  c              \\
    \end{tabular}
\end{table}

For the \textit{jetstack/cert-manager} project, the difference is due to the changes in the code between
\checkNum{October 2019} and \checkNum{Mai 2020}, which was when the project data set for this thesis was collected.
This is indicated as \textit{Reason~a} in the table.
For \textit{peterq/pan-light}, there are multiple Go modules included in the project repository, so for this project
the difference is due to varying definitions of dependency code as described previously.
\textit{Reason~d} corresponds to this situation.
Finally, there are four projects that did not yet support the Go module system in \checkNum{October 2019} and included
vendored copies of their dependencies in the repository (\textit{Reason~c}), and four projects which contained separate
specific code for different architectures, which are not counted by \toolGeiger{} and thus not included in this work's
study (\textit{Reason~b}).
Therefore, the counts associated with \textit{Reason~b} are higher in~\cite{costa2020}.
The only significant methodical difference therefore is the different counting of code that is available separately for
different architectures, which is a limitation of the \toolGeiger{} tool.

By looking at \unsafe{} usage counts for each week since \checkNum{2015} in the history of \checkNum{270} projects
selected from the total projects data set, Costa et al. showed that while the share of packages containing \unsafe{}
code did not change significantly, the number of individual \unsafe{} usages increased over time.
The study by Costa et al. also includes a set of manually labeled purposes of \unsafe{} usages, however the authors
labeled one entire file from each project with a single label, while the labeled data set presented in this thesis is
built with a granularity of a label for a single line of code.
Comparing the labels was possible on only \checkNum{25} code samples that are included in both sets.
Furthermore, the data set presented in~\cite{costa2020} is incomplete at least in the version published as replication
package.
\checkNum{20} of the \checkNum{25} mutual code samples did not actually have a label in the data set
by~\cite{costa2020}.
The remaining \checkNum{five} matched reasonably well at least, as shown in Table~\ref{tbl:costa-labels-comparison}.

\begin{table}[htp!]
    \centering
    \caption{Comparison of \unsafe{} usage labels in Costa et al. and this work}
    \label{tbl:costa-labels-comparison}
    \begin{tabularx}{\textwidth}{X|l|l}
        \multicolumn{3}{l}{\footnotesize\textit{Source Code}} \\
        \textbf{Labels in~\cite{costa2020}} & \textbf{My labels}     & \textbf{Match} \\
        \hline
        \multicolumn{3}{l}{\footnotesize\textit{*(*uintptr)(unsafe.Pointer(uintptr(unsafe.Pointer(\&b[0])) + uintptr(i))) \^{}= kw}} \\
        Performance Optimization   & pointer-arithmetic / efficiency     & match  \\
        \hline
        \rowcolor{verylightgray}
        \multicolumn{3}{l}{\footnotesize\textit{return Pointer\{unsafe.Pointer(v.Pointer()), v.Type()\}}} \\
        \rowcolor{verylightgray}
        Reflect                    & cast-pointer / reflect              & match  \\
        \hline
        \multicolumn{3}{l}{\footnotesize\textit{raw := (*unix.InotifyEvent)(unsafe.Pointer(\&buf[offset]))}} \\
        System Call                & cast-struct / serialization         & mine more fine-grained \\
        \hline
        \rowcolor{verylightgray}
        \multicolumn{3}{l}{\footnotesize\textit{entryHdr := (*entryHdr)(unsafe.Pointer(\&entryHdrBuf[0]))}} \\
        \rowcolor{verylightgray}
        Convert from byte          & cast-bytes / efficiency             & match  \\
        \hline
        \multicolumn{3}{l}{\footnotesize\textit{if n := int(uintptr(unsafe.Pointer(\&b[0]))) \% wordSize; n != 0 \{}} \\
        Performance Optimization   & pointer-arithmetic / serialization  & mine more fine-grained \\
    \end{tabularx}
\end{table}

\checkNum{Three} of the labels were exact matches of the label provided by~\cite{costa2020} and one of the two labels
assigned to the sample in my data set presented in this thesis.
The second label is lost because~\cite{costa2020} only did a one-dimensional labelling.
For the remaining \checkNum{two} samples, the labels did not match, however my labels are more fine-grained with respect
to the specific line of code.
The labels assigned by~\cite{costa2020} are appropriate for their scope, which is the complete file containing the line
of code shown in the table.
Overall, the usage purposes identified by~\cite{costa2020} matched the ones identified in this thesis with most of them
being used for the foreign function interface (\acrshort{FFI}) and efficient type casting, and few of them to achieve
reflection and direct pointer manipulations.

Additional contributions of~\cite{costa2020} include a manual correlation of \unsafe{} usages to project domains, which
showed that networking projects are the biggest group, followed by development tools, container virtualization,
databases, and projects that offer bindings to other projects.
Bindings and networking projects include the most projects with heavy use of \unsafe{} (more than \checkNum{100} calls
in a project).


%% ---------------------------------------------------------------------------------------------------------------------

\section{Unsafe APIs Across Languages}\label{sec:related-work:unsafe-across-languages}

Similar to the Go \unsafe{} \acrshort{API}, there are other programming languages that offer ways to circumvent their
respective measures for memory safety.
Rust offers a sophisticated concept of value ownership that prevents invalid memory accesses like
\textit{use-after-free} bugs~\cite{matsakis2014}.
To allow developers to circumvent this safety measure when necessary, for example to implement low-level functions, it
provides the \unsafe{} keyword to mark a function or code block that is allowed to do five additional, potentially
unsafe operations.
They include dereferencing raw pointers, calling \unsafe{} functions, and accessing mutable static
variables~\cite{matsakis2014}.
This achieves the same level of possibilities as offered by the \unsafe{} \acrshort{API} in Go.
Recently, two studies have analyzed the usage of \unsafe{} code blocks in open-source Rust libraries and applications.
Evans et al.~\cite{evans2020} presented an empirical study revealing that \checkNum{less than 30\%} of analyzed
libraries directly contained \unsafe{} code, however \checkNum{more than half} did when their dependencies were included
in the analysis.
Over the course of their \checkNum{10 months} study, they found that these numbers did not change significantly.
Most of the \unsafe{} Rust code is used to call other Rust functions, while interoperability with external C code was a
smaller concern.
The authors conducted an N=\checkNum{20} survey about the reasons to use \unsafe{} on Reddit, which showed that the most
popular reasons were performance optimizations, advanced data structures, or a more elegant interface made possible by
using \unsafe{} \acrshort{API}s.
\checkNum{10\%} of developers indicated that they have used \unsafe{} just to make the code compile in the past.
\jl{evtl. rausnehmen.}
Qin et al.~\cite{qin2020} studied bug reports that were related to \unsafe{} code usage in Rust, which revealed that the
most common bug classes are buffer overflows, null-pointer dereferencing, reading uninitialized memory, and
\textit{use-after-free}.
Often, the cause of the bug was an incorrect check for specific edge cases, which could be fixed by conditionally
skipping the execution of \unsafe{} code.
The authors note that it is very dangerous to have hidden \unsafe{} code in regular functions, which can be called from
safe Rust code, because developers then often need to make sure the input data is in a proper state which is not
immediately obvious.
Furthermore, Almohri et al.~\cite{almohri2018} presented a system to ensure memory safety while executing \unsafe{} code
in Rust by limiting access to the program's memory during the times \unsafe{} functions run.
This is done by splitting the memory address space into regions with different trust levels and thus creates a sandbox
that the \unsafe{} code runs in.
Since it is possible to compile Rust to WebAssembly, a binary intermediate code representation that is shipped to web
browsers~\cite{rourke2018}, usage of \unsafe{} blocks in Rust might lead vulnerabilities that occur in the virtual
machine environment when executing the WebAssembly code.
Lehmann et al.~\cite{lehmann2020} presented a study on this possibility.

Java also contains an \unsafe{} API with the \textit{sun.misc.Unsafe} library, which can cause security vulnerabilities
when misused~\cite{mastrangelo2019}.
Mastrangelo et al.~\cite{mastrangelo2015} showed that \checkNum{25\%} of the artifacts analyzed in an empirical study
used the \unsafe{} library.
Huang et al.~\cite{huang2019} analyzed the causes and consequences of misuses of that library, and showed different
patterns in which affected programs can crash due to such programming errors.

Finally, for the non memory-safe languages C and C++ there is previous work on achieving partial memory safety at
least~\cite{burow2018, nagarakatte2009}.
Song et al.~\cite{song2019} present tools that help with the process of identifying vulnerabilities in applications.
Furthermore, there have been comprehensive studies in the past about vulnerabilities related to memory
safety~\cite{szekeres2013,alnaeli2017,larochelle2001}.


%% ---------------------------------------------------------------------------------------------------------------------

\section{Other Go Vulnerabilities}\label{sec:related-work:other-go-vulnerabilities}

Concurrency, channels, escape analysis \cite{wang2020, sibiryov2017, tu2019, giunti2020, lange2017, zhou2017, dilley2019, hill2002, hannan1998, choi1999}


%% ---------------------------------------------------------------------------------------------------------------------

\section{Static Code Analysis}\label{sec:related-work:static-code-analysis}

cryptolint, CogniCryptSAST, DLint \cite{egele2013, kruger2018, gong2015, smith2020, gabet2020, wickert2019}
C++\cite{song2019}


%% ---------------------------------------------------------------------------------------------------------------------

\section{Security Issues with Dependencies}\label{sec:related-work:dependency-issues}

counting, updating, introducing vulnerabilities \cite{xia2014, mirhosseini2017, kula2017, watanabe2017, pashchenko2018}



%%% -----------------------------------------------------------------------------
%
%\section{Dependency analysis in Rust with Cargo Geiger}\label{sec:cargo-geiger}
%
%Cargo Geiger is the template for the Go dependency check tool.
%It does the same thing for the Rust programming language.
%
%
%%% -----------------------------------------------------------------------------
%
%\section{OWASP Dependency Checker}\label{sec:owasp-dependency-checker}
%
%Explain OWASP security project.
%
%Show OWASP Dependency Checker program, show experimental Go extension for it.
%This tool does no unsafe code analysis though, it checks for known vulnerabilities of the dependencies in the
%OWASP database.
%
%
%%% -----------------------------------------------------------------------------
%
%\section{Work in progress: paper summaries}\label{sec:paper-summaries}
%
%
%%% -----------------------------------------------------------------------------
%
%\subsection{Understanding Memory and Thread Safety Practices and Issues in Real-World Rust Programs}
%\label{subsec:understanding-memory-and-thread-safety-practices-and-issues-in-real-world-rust-programs}
%
%Qin et al.~\cite{qin2020} contribute the first to their knowledge empirical study of bugs related to unsafe code blocks
%in Rust.
%They mention other empirical studies~\cite{difranco2017, lu2013, chou2001, leesatapornwongsa2016, jin2012, gunawi2014, gu2015}
%none of which seem to be related to Rust or Go.
%The authors analyze 5 projects, 5 libraries and 2 databases.
%They randomly sample 850 unsafe usages out of those.
%They analyze the usages, categorize them into classes, analyze the impact of the bugs and develop two new static code
%checking tools.
%In essence, this is exactly my thesis but done for Rust, and it's probably an excellent example of how to structure a
%paper on this.
%
%A main difference to this thesis is that the authors not only look into the current revision code, but explicitly look
%through the Git history, filtering for commits that remove unsafe usages.
%They further go through reported bugs on the software under analysis and look into the code how those bugs were fixed.
%Bug data is retrieved from CVE and RustSec.
%This is something I should probably do as well!
%
%Within the 850 unsafe usages, the authors analyze 70 memory-safety issues and 100 concurrency bugs.
%It sounds like they found all of those, but this high number is because they look at real-world bugs that were previously
%reported.
%Using their tools however, they also find about 5 to 10 new bugs that they disclosed to the developers.
%
%Similar to my thesis, they explain the purpose of safe code and reasons why unsafe code might be needed.
%They obviously focus on Rust, but some of the points will apply to Go as well.
%
%The reasons to use unsafe code are clustered into these groups: Reuse existing code, convert C-array to Rust slice,
%improve performance.
%The authors find that often the use of unsafe code has good or unavoidable reasons.
%Looking at the commit diffs, they find that unsafe code gets removed to fix memory safety, improve code structure,
%improve thread safety, fix bugs, or because it was unnecessary in the first place.
%
%The authors also look into the Rust standard library, which has similar uses of unsafe as the Go standard library.
%Unsafe code often requires some preconditions on lifetime or input data to hold, and this can be achieved by encapsulating
%it into an interior unsafe function that checks these conditions and might skip the unsafe code if they do not hold.
%
%The areas of problems identified are buffer overflows, null-pointer dereferencing, reading uninitialized memory, invalid
%free, use after free, and double free.
%Fixing strategies were conditionally skipping the unsafe code, adjusting variable lifetime, or changing unsafe operands.
%
%Rust also has thread safety problems, and problems the authors identified can be grouped into blocking and non-blocking.
%Blocking bugs are caused due to incorrect scoping of auto-releasing mutexes.
%Channels might also block.
%Non-blocking bugs include data races and stem from incorrect scoping of shared data or confused ownership.
%
%If a function's safety depends on how it is used, it might be better put into an encapsulation.
%
%The authors suggest development of IDE extension visualizing scopes and lifetimes, development of Rust-tailored static
%analysis tools (which they already contributed two), and dynamic analysis tools.
%Due to the study language developers can also learn from design issues concerning the Rust language itself, because it
%shows how developers adapt to new language concepts over time.
%
%
%%% -----------------------------------------------------------------------------
%
%\subsection{Is Rust Used Safely by Software Developers?}
%\label{subsec:is-rust-used-safely-by-software-developers?}
%
%Evens et al.~\cite{evans2020} present an empirical study of unsafe usages in Rust crates.
%They only include statistical facts such as total number of unsage usages.
%This is different from Qin et al.~\cite{qin2020} in that they did not do an in-depth analysis of potential bugs.
%Still they got accepted at ICSE!
%
%They found that most unsafe usage is to call other unsafe Rust code, while calling C code is less of a concern.
%About a third of libraries contain unsafe code and more than half of them transitively do through dependencies.
%The usage count in crates did not change much over the course of 10 months work.
%More popular libraries are more likely to contain more unsafe code because they encapsulate more popular C libraries.
%
%The authors conducted a N=20 survey on Reddit, revealing that 10\% of developers used unsafe to make the code compile.
%Other popular reasons included performance optimizations, advanced data structures, and unsafe code offering a slicker
%and "more elegant" interface.
%Most developers employed more testing, static and dynamic analysis, and fuzzing when using unsafe code, but many also
%said they would "look carefully on the code".
%I'd suspect this is not an effective approach.
%
%The authors developed an augmented Rust CFG and algorithm to detect potentially unsafe functions in all Rust crates that
%would compile.
%
%
%%% -----------------------------------------------------------------------------
%
%\subsection{Escape from Escape Analysis of Golang}
%\label{subsec:escape-from-escape-analysis-of-golang}
%
%Wang et al.~\cite{wang2020} propose an approach to make heap memory usage in Go more effective by improving the escape
%analysis algorithm.
%The current algorithm is very conservative.
%In particular, the authors try improve a specific type of escape analysis: passing a pointer into a function call will
%make that object escape.
%They contribute an optimizer that always replaces such calls by an intermediate cast of the pointer into a uintptr
%variable and then back to a pointer, both through the use of unsafe.Pointer.
%This breaks the escape analysis chain and will make Go escape only the new pointer to the heap, not the entire previous
%data structure.
%After the identification of such snippets in the code comes a verification stage that will check whether the function
%call is synchronous.
%In that case, the underlying variable cannot be freed before the end of the function, and the optimization is correct.
%If the call is asynchronous, the optimizer checks if the variable is used in any other Goroutine, in which case the
%optimization will not be done.
%Otherwise it is deemed valid.
%
%The authors mention escape analysis in other language~\cite{hill2002, hannan1998, choi1999}.
%
%This is an interesting related work because the tool to break escape analysis, the cast to uintptr and back, is exactly
%the problem in the common unsafe slice cast that I describe in the other sections.
%
%The authors evaluate performance and correctness on 10 open-source and 10 industrial projects.
%Their data set includes kubernetes.
%
%The paper includes advertisement for a company and curiously uses Go 1.9 although the paper is to be published at ICSE
%2020.
%
%
%%% -----------------------------------------------------------------------------
%
%\subsection{Why Can’t Johnny Fix Vulnerabilities: A Usability Evaluation of Static Analysis Tools for Security}
%\label{subsec:why-can’t-johnny-fix-vulnerabilities:-a-usability-evaluation-of-static-analysis-tools-for-security}
%
%Smith et al.~\cite{smith2020} conduct a developer survey on usability problems with static code analysis tools.
%They contribute insights into why developers can often not get the most beneficial outcome from such tools.
%The mainly analyze Java analysis tools such as Find Bugs, CheckStyle and PMD\@.
%
%The problem areas found are code navigation issues, missing or buried information, interface scalability for large
%projects (such as a huge control flow graph), inaccuracy of analysis, code disconnect (meaning the proposed fix to a
%problem did not resemble the previous problem anymore), and workflow continuity (meaning the developers had to
%constantly switch between the tool report and the code editor).
%
%The authors propose as improvements a clear communication of what and how to fix, alerts that are located within the
%editable source code, contextualized and meaningful notifications, and a good integration into existing development
%workflows such as CI.
%
%Minor relevance, probably cite in the chapter about the linting tools.
%
%
%%% -----------------------------------------------------------------------------
%
%\subsection{Source Code Vulnerabilities in IoT Software Systems}
%\label{subsec:source-code-vulnerabilities-in-iot-software-systems}
%
%Alnaeli et al.~\cite{alnaeli2017} contribute a study of unsafe C code patterns, that is well-known unsafe
%functions such as strcpy, in IoT software.
%They count how many unsafe and how many safe functions they find, and compare them with each other to discover trends.
%
%Similar to my work, they search for unsafe code patterns in open-source code.
%Similar to my work, they also look at changes over time.
%In contrast to my work, they analyze C code instead of Go code.
%
%The authors find that the systems under review have neither introduced more nor removed some of the unsafe functions
%over time.
%They conclude that developers might be unaware of the presence of the functions, their implications, or both.
%They suggest better developer education on security-related consequences of these functions.
%
%Minor relevance, cite in survey chapter.
%
%
%%% -----------------------------------------------------------------------------
%
%\subsection{Statically Detecting Likely Buffer Overflow Vulnerabilities}
%\label{subsec:statically-detecting-likely-buffer-overflow-vulnerabilities}
%
%In their (quite old) 2001 paper, Larochelle and David Evans~\cite{larochelle2001} write about static code analysis to
%detect likely buffer overflow vulnerabilities in C code.
%They cite a paper assuming that buffer overflows would still be relevant in 20 years, that would be 2019.
%That would be true but I'll have to see if it makes sense to also cite this and add a comment.
%
%Similar to my work, this is using static analysis to find problems in code.
%But contrary to my work, it's for C code, not Go.
%They also talk about simple overflow exploitation techniques, in the context of C\@.
%I can and probably should do the same, with a focus on Go.
%
%The authors exploit semantic comments to enable local checking of interprocedural properties, they focus on lightweight
%analysis that does not add much overhead, and they use heuristics.
%They conduct their study using the LClint annotation tool also developed by Evans.
%They note that annotating the standard library would bring a lot of security even without annotating actual programs,
%because most vulnerabilities come from using insecure or improperly used functions.
%This I can use too to argue for better auditing of standard libraries and developer education as well.
%
%The annotations available include data constraints, such as x > y, and control flow constraints such as branch-specific
%annotations.
%
%
%%% -----------------------------------------------------------------------------
%
%\subsection{Vulnerable Open Source Dependencies: Counting Those That Matter}
%\label{subsec:vulnerable-open-source-dependencies:-counting-those-that-matter}
%
%Pashchenko et al.~\cite{pashchenko2018} in a quite recent study propose a way of counting relevant dependencies in a
%project and identifying whether the dependencies are vulnerable.
%The study is done for Java, and specifically for Maven dependency management.
%The authors note that failure of identifying relevant dependencies might lead to bad allocation of development resources.
%
%A central point is dependency scoping (production versus testing) which many studies did not incorporate in the past.
%If a vulnerability is found in a dependency only relevant for testing, it might not be worth to put resources onto its
%mitigation.
%This is extremely relevant for me!
%As of now, I also do not distinguish between testing and production library.
%This is probably not even possible with the Go dependency management system.
%A key insight for distiguishing this related work from my work would be to highlight the relative instability of the
%incredibly new Golang dependency system compared to the very seasoned Maven system.
%The authors find that around 20 \% of dependencies are for testing only (not deployed).
%
%Furthermore, it is very important to transitively look at dependencies of dependencies because they can just as well
%introduce problems into the main project.
%This is something my study already perfectly does.
%
%The authors contribute a method of determining whether a dependency is actively maintained or halted.
%It is done by looking at the release cycle with an exponential smoothing model.
%There are only a few vulnerabilities in halted libraries that they found, but those are especially important because
%mitigation cannot be done by a simple upgrade.
%
%The authors introduce the important concept of reliability for bug fixing.
%This is, the developers are directly reliable for fixing their main package and own dependencies, and they need to
%make sure their direct dependencies are up to date but cannot fix them.
%The indirect, or transitive, depencies are out of responsiblity for the developers since they cannot even be upgraded
%manually.
%A major finding is that grouping the dependencies of a project into these reliability groups gives a better picture into
%how much the developers can actually do, and the authors find that for more than 80 \% of the problems the developers
%can mitigate themselves through an upgrade or fixing their own code.
%
%Identification of vulnerabilities is done by matching code patterns to vulnerability databases both in a manual and an
%automated fashion.
%The developers contribute a tool that can provide annotations to the code as to which vulnerablity it might belong.
%
%A very relevant insight for me: the authors first did an incorrect dependency popularity measurement.
%They counted the times the dependency gets imported, possibly distorting the popularity if a project is decomposed into
%many small own dependencies~\cite{sajnani2014}.
%A better way is to count the projects that include the dependency.
%I already use the second approach in Table~\ref{tbl:pull-requests} for the supplied pull requests for fixes.
%However, I need to make sure in the plot generation that I use the same metrics.
%
%The authors compare to other ecosystems such as Pip and NPM, but not Go.
%For NPM, they cite a relevant study on Javascript dependency vulnerabilites~\cite{lauinger2017}, which showed that
%transitive project dependencies are more vulnerable than direct dependencies.
%They reason that this might be the case because developers are less aware of the existence of those dependencies, and
%have less control about them.
%The authors highlight that this is a finding specific to Javascript because NPM allows several versions of the same
%library to be used in the same project, while Maven does not.
%I need to find out if this is possible with Go modules too, I think it is not.
%I think Go behaves like Maven.
%However, I need to highlight that in my data survey there are of course different versions of the same library, and this
%is even necessary to analyze unsafe usage over time.
%
%Within threats to validity, the authors note that the selection of libraries was potentially biased, that the
%vulnerability database may not cover all vulnerabilites, the study only used Maven, and that project IDs were approximated.
%
%This is an extremely valuable related work that I definitely need to cite both in the background on Go dependencies
%and the survey as well.
%
%
%%% -----------------------------------------------------------------------------
%
%\subsection{Do developers update their library dependencies?}
%\label{subsec:do-developers-update-their-library-dependencies?}
%
%Kula et al.~\cite{kula2017} conduct a study on Github projects and developer survey on how dependencies are updated.
%They find that developers often think library updates are extra work and deprioritize it.
%The study is done on Maven and Java and finds that libraries are rarely updated.
%Specifically, security updates are included less often because developers were unaware of them.
%Upon getting the news, developers promptly updated away from vulnerable dependencies.
%
%Available updates are traditionally announced on the homepage of the library, through change logs and semantic versioning.
%Other possibilities include security advisories.
%
%When selecting vulnerabilities for analyzing security advisories, the authors only use denial of service and man in the
%middle vulnerabilities.
%This means that the survey is both rather shallow and biased towards web projects.
%
%This work is minor to medium relevant.
%It's to include in a list of examples to cite, but on its own it is less effective as a related work than the others
%I read.
%
%
%%% -----------------------------------------------------------------------------
%
%\subsection{Can automated pull requests encourage software developers to upgrade out-of-date dependencies?}
%\label{subsec:can-automated-pull-requests-encourage-software-developers-to-upgrade-out-of-date-dependencies?}
%
%Mirhosseini et al.~\cite{mirhosseini2017} answer the research question whether automated pull requests such as greenkeeper
%or pyup can encourage developers to update dependencies.
%They find that badges provide a good incentive to developers to keep them green by updating.
%Automated pull requests offer an actionable solution to update, but notification fatigue can work against update
%discipline.
%That is, maintaining many open source projects can lead to a lot of pull request notifications, especially if there is
%one opened for every single dependency.
%
%The authors analyze about 7,400 Github projects, grouped into badges, automated pull requests, and a baseline.
%Automated pull requests lead to 1.6 times, badges to 1.4 higher probability to update.
%Pull requests also mean faster updates.
%On purpose, they use a broad sample of Github projects instead of the most popular ones.
%This is something I should talk about too in my work!
%In summary, both versions of update helpers had a positive impact, with pull requests being in the lead.
%
%A surprising finding was that only about 30 \% of automated PRs was merged, compared to about 80 \% of general PRs.
%Some merged update PRs were also downgraded again after some days, hinting incompatibilities introduced.
%Using CI correlated with marginally higher merge rate.
%
%They also conduct a developer survey.
%Developers had mixed feelings about badges versus automated pull requests.
%They suggested to use batched updates, resulting in fewer pull requests and less fatigue.
%
%The authors identified through survey different updating strategies: quick, scheduled, reactive.
%A developer said, reasoned through possible new errors in libraries and minor benefits, that in the reality of commercial
%software engineering, staying on bleeding edge library versions usually costs more than it benefits, and the better
%approach is to update infrequently.
%My work should act on this and argue that this is true for feature releases but very bad for security releases.
%Maybe one could do something with semver here.
%
%The survey also showed that a single bad update of one library can very strongly cause developers to be reluctant from
%updating in the future.
%From this, we can argue for high responsible in library releasing to not destroy the necessary environment for timely
%ships of security fixes.
%
%The authors further cite another work stating that vulnerabilities are often contained in dependencies~\cite{xia2014}.
%CI is not always a guarantee that an update does not break the software.
%Badges can counteract fatigue, and more importantly they rely on a different mechanism: social pressure.
%
%This has very relevance for me because of the survey on update preferences.
%It is interesting to cite this within the argument that libraries must be updated in order to remove already fixed
%vulnerabilities from dependent projects.
%
%
%%% -----------------------------------------------------------------------------
%
%\subsection{Understanding the Origins of Mobile App Vulnerabilities: A Large-scale Measurement Study of Free and Paid Apps}
%\label{subsec:understanding-the-origins-of-mobile-app-vulnerabilities:-a-large-scale-measurement-study-of-free-and-paid-apps}
%
%Watanabe et al.~\cite{watanabe2017} conduct a study on the origins of mobile vulnerabilities.
%They inspect both free and paid Android apps.
%They find that 70 \% of bugs come from dependencies, 50 \% from third-party dependencies.
%More expensive and/or popular apps tended to have more vulnerabilities.
%
%An interesting finding for me: 50 \% of vulnerabilities were found in unreachable code.
%As I also already found one manual vulnerability that was unreachable, it might be interesting to argue that there is a
%very feasible risk that the studies are distorted due to failure to identify dead code.
%
%The authors claim that a limitation of static analysis is that obfuscated, runtime-dynamic code might introduce
%vulnerabilities that the analysis missed.
%This might be true, but such code is also to be treated as dangerous in my opinion.
%
%In contrast to my work, this is based on Android and Java at best, not Go.
%
%This is only minor relevant as another study.
%
%
%%% -----------------------------------------------------------------------------
%
%\subsection{Understanding Real-World Concurrency Bugs in Go}
%\label{subsec:understanding-real-world-concurrency-bugs-in-go}
%
%Tu et al.~\cite{tu2019} present a very recent (2019) study on Go concurrency bugs.
%They summarize how Go was designed to be an especially safe and easy-to-use language for concurrency.
%Every Go major release has improved on concurrency.
%Go offers both shared memory (through use of Mutex, RWMutex, Cond, atomic, Once, WaitGroup) and message passing(through
%chan).
%Channels can be buffered and unbuffered.
%New additions include the select statement, context, and Pipe.
%
%The authors analyzed six open-source Go applications, and inspected about 170 bugs.
%They used commit logs and keywords to find concurrency-related bugs and fixes.
%Then they grouped them on two dimensions into blocking / non-blocking, and message passing / shared memory.
%They answered the research question of how often goroutines get used in the real world, statically and dynamically.
%
%Similar to my work, their static analysis just involved finding instances of the go keyword.
%Dynamic analysis is more complicated though.
%They found that goroutines are used very often in the sample programs.
%Similar to my work, they compared the frequency of the two methods of concurrency synchronization.
%I can do a similar argument when comparing different keywords of unsafe code.
%The authors find that usage trends were stable over time.
%
%The authors analyzed commit messages and filtered for the following keywords: race, deadlock, synchronization,
%concurrency, lock, mutex, atomic, compete, context, once, goroutine leak.
%They randomly sampled, filtered for actual bug fixings, and then manually analzed and grouped them.
%This is something I should do as well.
%Bugs were split about 50/50 between message passing and shared memory.
%
%The conclusion is that new Go mechanisms that should make concurrency easier and safer can actually introduce new bugs.
%
%This is an extremely relevant related work.
%Cite this both in related work, background, and survey.
%
%
%%% -----------------------------------------------------------------------------
%
%\subsection{A Dataset of Parametric Cryptographic Misuses}
%\label{subsec:a-dataset-of-parametric-cryptographic-misuses}
%
%Wickert et al.~\cite{wickert2019} present a labeled data set of 201 crypto API misuses.
%They focus on Java crypto API JCA parametric misuses, that is the usage of insecure configuration and/or wrong
%parameters to the crypto API.
%They initially downloaded more than 1300 Java projects from Github, but only about 130 of them contained the crypto
%APIs, 53 of them could be built, and 39 of them could be analyzed successfully by a static analysis tool.
%Still, in the remaining 39 projects plus some manual investigation they found 201 misuses in 44 projects.
%
%The authors sorted the misuses by two dimensions: API used (e.g. Cipher, MessageDigest, \ldots) and misuse category (e.g.
%transformation algorithm, incorrect randomization, key, initialization vector, or iteration count).
%
%The authors included their findings into MUBench and performed a benchmark on precision and recall on the FindBugs tool.
%Furthermore, they iterate on how their data set can be used as a foundation for further research, e.g. benchmarking new
%static analysis tools or training classifiers.
%
%Similar to my work, they mined projects from Github and looked for security problems.
%In contrast to my work, they focused on Java and crypto APIs, not Go and unsafe pointers.
%They state that there exists possible bias in project selection, and that approaches that are secure today might become
%insecure in the future.
%
%Medium relevant, cite in related work and both survey chapters I think.
