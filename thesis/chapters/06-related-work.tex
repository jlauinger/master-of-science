%% ---------------------------------------------------------------------------------------------------------------------

\chapter{Related Work}\label{ch:related-work}

This chapter discusses related and concurrent work relevant to this thesis.
First, a similar, concurrent study on \unsafe{} code usages in the Go programming language is presented, replicated, and
compared in detail to this work.
Then, related studies and papers are discussed in groups identified by their general topic.


%% ---------------------------------------------------------------------------------------------------------------------

\section{Concurrent Study on \textit{Unsafe} in Go}\label{sec:related-work:concurrent-study}

Costa et al. submitted a similar study to this work on usages of the \unsafe{} \acrshort{API} in open-source Go code to
the IEEE Transactions on Software Engineering journal.
At the time this thesis is written, their paper is not yet accepted, thus, it is cited from arXiv~\cite{costa2020}.
While I personally could not find substantial errors in the paper, it has not yet been peer-reviewed.
Since it is a very relevant concurrent work to this thesis, I nevertheless reproduced their empirical study and compared
it to the results of this thesis.

The authors present a study of \checkNum{2,438} popular Go projects, which were downloaded on
\checkNum{October~2, 2019}.
Their project selection was done by taking the top \checkNum{3,000} most-starred open-source projects on \github{},
filtering out archived, inactive, and young projects with less than \checkNum{10} commits.
Furthermore, they removed educational projects such as books or learning material, as well as projects that could not be
downloaded for some reason.
Then, they counted the number of \unsafe{} usages in each project using a parser to generate the abstract syntax tree
(\acrshort{AST}).
They found that \checkNum{24\%} of the studied projects use \unsafe{} code.
In contrast, the study presented in Chapter~\ref{ch:go-geiger} in this thesis finds that \percentageUnsafeProjects{}
use \unsafe{}.
This difference is partly due to the different choice of projects, which was verified by comparing the set of analyzed
projects available through the published replication package\footnote{\url{https://zenodo.org/record/3871931}}.
While Costa et al. analyzed more than \checkNum{2,400} projects, this thesis only analyzed \projsAnalyzed{}, indicating
that \unsafe{} usage is more prevalent in more popular projects.
Furthermore, a fundamental difference between this work and~\cite{costa2020} is that the study by Costa et al. did not
look at dependencies, instead they only analyzed the projects directly.
My analysis on the projects including their dependencies in contrast showed that
\percentageUnsafeTransitiveWithDependencies{} of the projects use \unsafe{} either directly or through some included
library.
To compare the study to~\cite{costa2020}, it is therefore necessary to focus on the \unsafe{} usages contained in the
top-level projects, which is done by taking into account only usages that are detected within the main project module
indicated by the \textit{go.mod} file in the root directory.
However, this can be inaccurate if there are multiple modules contained within the same project repository, because in
that case the study by Costa et al. would attribute code directly to a project, but it is seen as a dependency in my
study.
There are \checkNum{25} projects for which~\cite{costa2020} report \unsafe{} usages  but my study does not.
However, they are all included in the \percentageUnsafeTransitiveWithDependencies{} of projects for which my study
indicates that they use \unsafe{} code through their dependencies.
For this reason, it can be concluded that the difference in the fraction of projects using \unsafe{} is due to a
different definition of project and dependency code.

There are \checkNum{86} projects for which the study in this thesis and~\cite{costa2020} report different absolute
numbers of \unsafe{} usages, while the numbers match exactly for all remaining projects.
To find the reason for the difference, their study was replicated in the best possible way, given that Costa et al.
did neither include the exact revisions of the projects under analysis nor their parsing tool to count the \unsafe{}
usages in their replication package.
Table~\ref{tbl:costa-counts-comparison} shows the top \checkNum{10} projects with highest difference, including their
respective absolute counts found in the studies and the reason for the difference.

\begin{table}[h]
    \centering
    \caption[Comparison of \unsafe{} usage counts Costa et al. and this work]%
    {Comparison of \unsafe{} usage counts in Costa et al. and this work~\newline \tiny ~\newline \footnotesize
    Reasons: \underline{a} change over time, \underline{b} multiple architecture, \underline{c} dependency vendoring,
    \underline{d} submodules \tiny ~\newline}
    \label{tbl:costa-counts-comparison}
    \begin{tabular}{l|r|r|r|l}
        \textbf{Project}       & \textbf{Count in~\cite{costa2020}} & \textbf{Count in this work} & \textbf{Difference} & \textbf{Reason} \\
        \hline
        jetstack/cert-manager  &                                140 &                         374 &               + 234 &  a              \\
        \rowcolor{verylightgray}
        kubernetes/kubernetes  &                              2,058 &                       1,885 &              -- 173 &  b              \\
        golang/mobile          &                                211 &                          93 &              -- 118 &  b              \\
        \rowcolor{verylightgray}
        TykTechnologies/tyk    &                                 79 &                         187 &               + 108 &  c              \\
        elastic/beats          &                                164 &                          70 &               -- 94 &  c              \\
        \rowcolor{verylightgray}
        golang/tools           &                                113 &                          44 &               -- 69 &  b              \\
        peterq/pan-light       &                                 64 &                           0 &               -- 64 &  d              \\
        \rowcolor{verylightgray}
        cilium/cilium          &                                290 &                         345 &               -- 55 &  c              \\
        go-delve/delve         &                                 72 &                          20 &               -- 52 &  b              \\
        \rowcolor{verylightgray}
        ethereum/go-ethereum   &                                 42 &                          91 &                + 49 &  c              \\
    \end{tabular}
\end{table}

For the \textit{jetstack/cert-manager} project, the difference is due to the changes in the code between
\checkNum{October 2019} and \checkNum{May 2020}, which is when the project data set for this thesis was collected.
This is indicated as \textit{Reason~a} in the table.
For \textit{peterq/pan-light}, there are multiple Go modules included in the project repository, so for this project
the difference is due to varying definitions of dependency code as described previously.
\textit{Reason~d} corresponds to this situation.
Finally, there are four projects that did not yet support the Go module system in \checkNum{October 2019} and included
vendored copies of their dependencies in the repository (\textit{Reason~c}), and four projects which contained separate
specific code files for different architectures, which are not counted by \toolGeiger{} and thus are not included in
this work's study (\textit{Reason~b}).
Because of this, the counts associated with \textit{Reason~b} are higher in~\cite{costa2020}.
The only significant methodical difference therefore is the different counting of code that is available separately for
different architectures, which is a limitation of the \toolGeiger{} tool.

By looking at \unsafe{} usage counts for each week since \checkNum{2015} in the history of \checkNum{270} projects
selected from the total projects data set, Costa et al. showed that while the share of packages containing \unsafe{}
code did not change significantly, the absolute number of individual \unsafe{} usages increased over time.
The study by Costa et al. also includes a set of manually labeled purposes of \unsafe{} usages, however the authors
labeled one entire file from each project with a single label, while the labeled data set presented in this thesis is
built with a granularity of a pair of labels for a single line of code.
Comparing the labels was possible on only \checkNum{25} code samples that are included in both sets.
Furthermore, the data set presented in~\cite{costa2020} is incomplete, or at least the version published as replication
package is.
In the data set by~\cite{costa2020}, \checkNum{20} of the \checkNum{25} mutual code samples did not actually have a
label.
The remaining \checkNum{five} matched reasonably well at least, as is shown in Table~\ref{tbl:costa-labels-comparison}.
The table contains both the source code snippet and the labels assigned by~\cite{costa2020} and this work, as well as
their concordance in the third column.

\begin{table}[htp!]
    \centering
    \caption{Comparison of \unsafe{} usage labels in Costa et al. and this work}
    \label{tbl:costa-labels-comparison}
    \begin{tabularx}{\textwidth}{X|l|l}
        \multicolumn{3}{l}{\footnotesize\textit{Source Code}} \\
        \textbf{Labels in~\cite{costa2020}} & \textbf{My labels}     & \textbf{Match} \\
        \hline
        \multicolumn{3}{l}{\footnotesize\textit{*(*uintptr)(unsafe.Pointer(uintptr(unsafe.Pointer(\&b[0])) + uintptr(i))) \^{}= kw}} \\
        Performance Optimization   & pointer-arithmetic / efficiency     & match  \\
        \hline
        \rowcolor{verylightgray}
        \multicolumn{3}{l}{\footnotesize\textit{return Pointer\{unsafe.Pointer(v.Pointer()), v.Type()\}}} \\
        \rowcolor{verylightgray}
        Reflect                    & cast-pointer / reflect              & match  \\
        \hline
        \multicolumn{3}{l}{\footnotesize\textit{raw := (*unix.InotifyEvent)(unsafe.Pointer(\&buf[offset]))}} \\
        System Call                & cast-struct / serialization         & mine more fine-grained \\
        \hline
        \rowcolor{verylightgray}
        \multicolumn{3}{l}{\footnotesize\textit{entryHdr := (*entryHdr)(unsafe.Pointer(\&entryHdrBuf[0]))}} \\
        \rowcolor{verylightgray}
        Convert from byte          & cast-bytes / efficiency             & match  \\
        \hline
        \multicolumn{3}{l}{\footnotesize\textit{if n := int(uintptr(unsafe.Pointer(\&b[0]))) \% wordSize; n != 0 \{}} \\
        Performance Optimization   & pointer-arithmetic / serialization  & mine more fine-grained \\
    \end{tabularx}
\end{table}

\checkNum{Three} of the labels were exact matches of the label provided by~\cite{costa2020} and one of the two labels
assigned to the sample in the data set presented in this thesis.
The second label is lost, because~\cite{costa2020} only did a one-dimensional labeling.
For the remaining \checkNum{two} samples, the labels did not match, however my labels are more fine-grained with respect
to the specific line of code.
The labels assigned by~\cite{costa2020} are appropriate for their scope, which is the complete file containing the line
of code shown in the table.
Overall, the usage purposes identified by~\cite{costa2020} match the ones identified in this thesis with most of them
being used for the foreign function interface (\acrshort{FFI}) and efficient type casting, and few of them to achieve
reflection and direct pointer manipulations.

Additional contributions of~\cite{costa2020} include a manual correlation of \unsafe{} usages to project domains, which
showed that networking projects are the biggest group, followed by development tools, container virtualization,
databases, and projects that offer bindings to other libraries or applications.
Bindings and networking projects include the most projects with heavy use of \unsafe{} (more than \checkNum{100} calls
in a project).


%% ---------------------------------------------------------------------------------------------------------------------

\section{\textit{Unsafe} APIs Across Languages}\label{sec:related-work:unsafe-across-languages}

Similar to the Go \unsafe{} \acrshort{API}, there are other programming languages that offer ways to circumvent their
respective measures for memory safety.
Rust offers a sophisticated concept of value ownership that prevents invalid memory accesses like
\textit{use-after-free} bugs~\cite{matsakis2014}.
To allow developers to circumvent this safety measure when necessary, for example, to implement low-level functions, it
provides the \unsafe{} keyword to mark a function or code block that is allowed to do five additional, potentially
unsafe operations.
They include dereferencing raw pointers, calling \unsafe{} functions, and accessing mutable static
variables~\cite{matsakis2014}.
This achieves the same level of possibilities as offered by the \unsafe{} \acrshort{API} in Go.
Recently, two studies have analyzed the usage of \unsafe{} code blocks in open-source Rust libraries and applications.
Evans et al.~\cite{evans2020} presented an empirical study revealing that \checkNum{less than 30\%} of analyzed
libraries directly contained \unsafe{} code.
However \checkNum{more than half} did when their dependencies were included in the analysis.
Over the course of their \checkNum{ten months}' study, they found that these numbers did not change significantly.
Most of the \unsafe{} Rust code is used to call other Rust functions, while interoperability with external C code was a
smaller concern.
The authors conducted an N=\checkNum{20} survey about the reasons to use \unsafe{} on Reddit, which showed that the most
popular reasons were performance optimizations, advanced data structures, or a more elegant interface made possible by
using \unsafe{} \acrshort{API}s.
About \checkNum{10\%} of developers indicated that they have used \unsafe{} just to make the code compile in the past.

Qin et al.~\cite{qin2020} studied bug reports that were related to \unsafe{} code usage in Rust, which revealed that the
most common bug classes are buffer overflows, null-pointer dereferencing, reading uninitialized memory, and
\textit{use-after-free} bugs.
Often, the cause of the bug was an incorrect check for specific edge cases, which could be fixed by conditionally
skipping the execution of \unsafe{} code.
The authors note that it is very dangerous to have hidden \unsafe{} code in regular functions, which can be called from
safe Rust code, because developers often need to make sure that the input data is in a proper state, which is not
immediately obvious.

Furthermore, Almohri et al.~\cite{almohri2018} presented a system to ensure memory safety while executing \unsafe{} code
in Rust by limiting access to the program's memory during the times \unsafe{} functions run.
This is done by splitting the memory address space into regions with different trust levels and, thus, creates a sandbox
which the \unsafe{} code runs in.
RustBelt~\cite{jung2017} is a formal proof of safety-related properties of a subset of the Rust language with \unsafe{}.
Since it is possible to compile Rust to WebAssembly, a binary intermediate code representation that is shipped to web
browsers~\cite{rourke2018}, usage of \unsafe{} blocks in Rust might lead to vulnerabilities that occur in the virtual
machine environment when executing the WebAssembly code.
Lehmann et al.~\cite{lehmann2020} presented a study on this possibility.

Java also contains an \unsafe{} API with the \textit{sun.misc.Unsafe} library, which can cause security vulnerabilities
when misused~\cite{mastrangelo2019}.
Mastrangelo et al.~\cite{mastrangelo2015} showed that \checkNum{25\%} of the artifacts analyzed in an empirical study
used this \textit{Unsafe} library.
Huang et al.~\cite{huang2019} analyzed the causes and consequences of misuses of \textit{Unsafe}, and showed different
patterns in which affected programs can crash due to such programming errors.

Finally, for the non-memory-safe languages C and C++, there is previous work on achieving partial memory safety at
least~\cite{burow2018, nagarakatte2009}.
Song et al.~\cite{song2019} presented tools that help with the process of identifying vulnerabilities in applications.
Furthermore, there have been comprehensive studies in the past about vulnerabilities related to memory
safety~\cite{szekeres2013,alnaeli2017,larochelle2001}.


%% ---------------------------------------------------------------------------------------------------------------------

\section{Go Vulnerabilities Unrelated to \textit{Unsafe}}\label{sec:related-work:other-go-vulnerabilities}

There has been a lot of previous research on security vulnerabilities in Go programs that are not related to the
\unsafe{} \acrshort{API}~\cite{zhou2017, hill2002, hannan1998, choi1999}.
One of the features Go is most known for is its excellent support for concurrent program execution~\cite{donovan2015}.
To synchronize concurrent threads, it offers both message passing through channels and coordinating memory access by
using mutually exclusive locking of variables through mutexes.
Tu et al.~\cite{tu2019} presented a study of about \checkNum{170} concurrency-related bugs in \checkNum{six} open-source
Go applications.
To do this, they identified commits that are related to fixing such bugs by searching the project history for related
keywords.
They found that there is an even distribution between bugs related to message passing and shared memory, and that the
usage of both techniques did not change significantly over time.
They conclude that adding new features to make a language safer is not necessarily sufficient, because new bug classes
can be introduced, especially if developers are not familiar with the new concepts.

Other related work on concurrency in Go includes a detailed analysis of message passing in open-source
projects~\cite{dilley2019}, which found that most projects use channels for synchronous message passing, and that common
models to organize concurrent threads are poorly supported by static analysis tools.
Lange et al.~\cite{lange2017} presented work on detecting dead locks and infinite loops that are caused by incorrect
usages of channels.
To do this, they built a model of possible communication patterns in Go applications, and applied bounded verification
to find incorrect ones.
Giunti~\cite{giunti2020} contributed a framework that can generate concurrent Go code that is free from data races and
deadlocks.
This is achieved using a formal communication model expressed in a special calculus.

Wang et al.~\cite{wang2020} discussed how the escape analysis of the Go compiler misses the connection between the
underlying data arrays of slices and strings when they are incorrectly converted into each other by creating a composite
literal header value as described in Section~\ref{subsec:unsafe-security-problems:slice-casts:escape-analysis}.
The authors, however, used this property to optimize the heap memory consumption of Go programs.
They presented a transpiler that modifies the intermediate binary representation created by the Go compiler, which is
converted to architecture-dependent specific assembly afterwards.
When a local variable is passed by reference to another function, it is often seen as escaped, because the Go compiler
uses a conservative approach to detect escaped values.
Therefore the variable will be allocated on the heap.
The transpiler checks whether there is any possible concurrent access to that variable, and if there is none, it might
be possible to allocate it on the stack of the calling function instead.
To achieve this, the reference passed to the function is converted into a \textit{uintptr} value and back to a reference
to break the connection between the values as seen by the escape analysis on purpose.
Doing this can improve heap usage, because some values are not unnecessarily allocated there.


%% ---------------------------------------------------------------------------------------------------------------------

\section{Static and Dynamic Code Analysis}\label{sec:related-work:static-code-analysis}

Previous work presented static code analysis tools that are designed with a focus on security vulnerabilities in mind,
similar to the \toolSafer{} linter introduced in this work.
For Android projects built in Java, there is \toolCryptolint{}~\cite{egele2013}.
It is a static analysis tool that detects misuses of cryptographic \acrshort{API}s in Android apps, such as the use of
the electronic code book (\acrshort{ECB}) cipher mode.
Its authors Egele et al. found that \checkNum{88\%} of the \checkNum{11,748} Android applications that they analyzed
contained at least one misuse.
The specification language \textit{CrySL}~\cite{kruger2018} can be used to formally describe rules that ensure
cryptographic \acrshort{API}s are used correctly.
Static analysis tools like \textit{CogniCryptSAST}~\cite{kruger2017} can check whether those rules are followed by an
application and, thus, find cryptographic misuses that are potential security vulnerabilities.
Wickert et al.~\cite{wickert2019} presented a labeled data set of \checkNum{201} cryptographic misuses in Java projects.
It is organized in two dimensions: the specific cryptographic \acrshort{API} that is used, and the category of misuse.
The authors integrated their novel data set into \textit{MuBench}~\cite{amann2016}, which is both a collection of
\acrshort{API} misuses as well as a framework to evaluate precision and recall of static analysis tools on the data
sets.

For JavaScript, Gong et al.~\cite{gong2015} presented \textit{DLint}, which is both a static and dynamic linter.
On top of common problems with JavaScript source code that are also found by previous static-only linters, it adds a
dynamic analysis approach that is able to check \checkNum{28} additional potential problems with the code.
In an empirical study on \checkNum{more than 200} popular websites, the authors found on average \checkNum{49} problems
per site that were missed with static analysis.

Song et al.~\cite{song2019} provided a detailed overview of available static and dynamic analysis tools for the C and C++
languages, including a taxonomy of the tools and security vulnerabilities detected by them.
They find that although there has been decades of research into analysis tools for these languages, there is still room
for improvement, in particular with respect to incorrect usages of pointer types and security vulnerabilities that are
caused by them.
These include, among others, the detection of dangling pointers, bounds checking of buffer accesses using pointers, and
finding invalid conversions between incompatible types.
For the Go programming language, Bodden et al.~\cite{bodden2016} presented an analysis of information flow and possible
data leaks through taint analysis.
Gabet et al.~\cite{gabet2020} contributed work on the static detection of race conditions and safe access of shared
memory through mutexes.
Similar to \textit{CrySL}, it is based on a formal model of communication in Go, and rules to detect incorrect instances.

Smith et al.~\cite{smith2020} conducted a study on the usability of static analysis tools with a focus on security
problems.
They found that often problems that could have been found by the tools are not fixed by the developers, because the user
interfaces are too complex and the tool gets abandoned.
The authors recommend four design guidelines to improve the usability of static analysis tools.
They include suggestions on how to fix a specific bug rather than just pointing out its existence, integrating the tools
within the existing workflows of developers, showing results within the code editor near the relevant code, and
contextualizing warning messages to the concrete incorrect code, such as including specific variable names.


%% ---------------------------------------------------------------------------------------------------------------------

\section{Security Issues with Dependencies}\label{sec:related-work:dependency-issues}

Concerning the dependencies or imports of projects, previous work has analyzed different aspects.
Pashchenko et al.~\cite{pashchenko2018} discussed how to focus on the most important dependencies when looking for
security vulnerabilities.
The study was done for Java, specifically for the Maven dependency management system.
A central point is to distinguish the different scopes for testing and production, as vulnerabilities that exist in
dependencies that are only relevant for testing have less impact.
This is because they are not shipped to customers as part of the final product, and therefore do not run on actual
production systems.
The authors did not claim that security vulnerabilities in such libraries are unimportant, but when companies have a
limited budget of development resources to spend on auditing dependencies, it is a good choice to focus on the libraries
with the most customer-facing impact.
The study found that in the analyzed Java projects, about \checkNum{20\%} of dependencies are required for development
and testing purposes only.
The Go module system, which compared to Maven is very new and only recently reached a stable version, does not have a
distinction between development and production dependencies.
Therefore, it is not possible to account for this difference in the study presented in this thesis.
Pashchenko et al. also proposed an algorithm for deciding whether a dependency is still actively maintained, or has been
abandoned.
They found few vulnerabilities in libraries that are no longer maintained, however those have a big impact, because
there are no updates to fix them.
Therefore, it is important to replace such dependencies with actively supported alternatives.

A study by Watanabe et al.~\cite{watanabe2017} analyzed the origins of vulnerabilities in mobile applications available
for Android.
They found that \checkNum{70\%} of them were introduced through vulnerable dependencies.
Furthermore, \checkNum{50\%} of the bugs came from third-party dependencies, where developers can not easily fix them,
and instead have to rely on the maintainers or replace the library with an alternative one.
Xia et al.~\cite{xia2014} also presented work on the prevalence of outdated libraries in open-source project
dependencies across different languages.

When insecure dependencies have been identified, it is important to update or replace the affected libraries as soon as
possible.
Kula et al.~\cite{kula2017} conducted a study on \github{} projects, as well as a developer survey, to find out how the
update process is done in practice.
They found that developers often think maintaining up-to-date versions for their set of dependencies is extra work that
can be deprioritized.
Thus, dependencies are not updated in a timely manner.
Specifically, the authors found that developers are often unaware of security vulnerabilities in dependencies.
As soon as they find out about them, dependencies are however usually updated quickly.
Mirhosseini et al.~\cite{mirhosseini2017} answered the question whether automated pull requests and badges that indicate
out-of-date dependencies on platforms like \github{} can encourage developers to update dependencies.
The authors found that badges provide a good incentive for developers to update, because there is a psychological urge
to keep the badges green.
Automated pull requests offer an easy and directly actionable way to update, but when there are many of them,
notification fatigue can work against update discipline.
In summary, badges account for a \checkNum{1.4} times higher probability to update, while automated pull requests
increase the probability by a factor of \checkNum{1.6} according to~\cite{mirhosseini2017}.
A developer survey done by the authors also showed that a single bad update of one library can cause developers to be
strongly reluctant from updating in the future.
Therefore, releasing library versions comes with a high responsibility to not break software, in order to keep an
environment where developers update their projects regularly to fix security vulnerabilities.
